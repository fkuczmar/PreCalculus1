\documentclass{ximera}
\title{Circles in a Corner}


\newcommand{\pskip}{\vskip 0.1 in}

\begin{document}
\begin{abstract}
Finding equations of all circles through a given point tangent to the coordinate axes.
\end{abstract}
\maketitle


\begin{question} \label{QMFfefef333}

\begin{enumerate}

\item Drag the center of each circle below so that it lies in the first quadrant and is tangent to both the $x$ and $y$ axes.

\begin{onlineOnly}
    \begin{center}
\desmos{bo49rg0xpe}{900}{600}
\end{center}
\end{onlineOnly}

\href{https://www.desmos.com/calculator/bo49rg0xpe}{141: Circles tangent to coordinate axes}

\item What can you say about the center of a circle in the first quadrant that is tangent to the coordinate axes?

\item Find an equation of the center set of circles in the first quadrant that are tangent to the coordinate axes. This means to find an equation of the set of all centers of circles in the first quadrant tangent to the coordinate axes.

\item Find an equation of the circle in the first quadrant with radius $6$ that is tangent to the coordinate axes. Enter this equation in Line 8.

\item Find an equation of the circle in the \emph{second} quadrant with radius $1/2$ that is tangent to the coordinate axes. Enter this equation in Line 10.


\end{enumerate}
\end{question}



\begin{question} \label{QLFerr3r355}
\begin{enumerate}
\item How many circles of radius $4$ are tangent to the coordinate axes? 

\item Find equations of two such circles.

\item Find an equation of the circle of radius $r > 0$ in the first quadrant tangent to the coordinate axes. Enter this equation in Line 3 of the worksheet below. Check that you are correct by dragging the slider $r$.


\begin{onlineOnly}
    \begin{center}
\desmos{3prz8y0fja}{900}{600}
\end{center}
\end{onlineOnly}

\href{https://www.desmos.com/calculator/32734b6500}{141: Circles in a Corner}

\item Drag the slider $r$ above to determine how many circles tangent to the coordinate axes pass through the point $(5,3)$. 

\item Use algebra to find equations of all the circle(s) in part (d). Enter these equations on Lines 4 and 5 of the desmos worksheet.

\end{enumerate}

\end{question}


\begin{question}  \label{QLfe9f8ZNM}
Let $a,b\geq 0$ be constants.

Find equations of all circles tangent to the coordinate axes thta pass through the point $Q$ with coordinates $(a,b)$.

Use the worksheet below to check your work.

\begin{onlineOnly}
    \begin{center}
\desmos{9yip8ilgly}{900}{600}
\end{center}
\end{onlineOnly}

\href{https://www.desmos.com/calculator/9yip8ilgly}{141: Circles in a Corner 2}

\end{question}

\end{document}

