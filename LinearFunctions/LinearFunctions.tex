\documentclass{ximera}
\title{Linear Functions}

\newcommand{\pskip}{\vskip 0.1 in}

\begin{document}
\begin{abstract}
The point-slope equation of a line.
\end{abstract}
\maketitle

Linear functions are those that change at a constant rate. For example, between 12:10pm and 12:20pm a balloon might ascends at a constant rate of 50 ft/min or a car might travel at a constant speed of 60 miles/hour. In the first case, the height of the balloon is a linear function of time and in the second the car's odometer reading is also a linear function of time.

It can be misleading to say that the height of the balloon increases 50 feet every minute as this could be interpreted as meaning \emph{only} that the height increases only over each one minute interval starting at 12:10pm (ie. from 12:10pm to 12:11pm, from 12:11pm to 12:12 pm, etc.), whereas in fact the height increases at the average rate of 50 ft/min over \emph{every} time interval between 12:10pm and 12:20pm. Furthermore, saying that the height increases 50 feet every minuteo obscures the units of the rate of change as being ft/min. So it's best to stick with the original wording, that the balloon ascends at a constant rate of 50 ft/min.

Now most functions are decidedly not linear, even over short periods of time. It is unlikely for a balloon to ascend at a contant rate for very long, nor is it really possible to drive at a constant speed for very long. Nevertheless, most functions are locally linear, or at least approximately so, and it is for primarily this reason that they are very useful. For example, over a short period of time we would expect a balloon to ascend or descend at a nearly constant rate and the speed of a car to be almost constant. 

Over long periods of time, some populations tend to grow approximately exponentially. But over a short period of time, the growth is nearly linear. For a dramatic way to see this, take a look at the website below to see how the U.S. and world populations are growing:

\href{https://www.census.gov/popclock/}{Population Clock}

We can also see this same phenomenon graphically. Consider, for example, the graph of the function
\[
    T = f(m) , 0\leq m \leq 20 ,
\]
that expresses the temperature (in Celsius degrees) of a cup of coffee in terms of the number of minutes past noon shown below. The function is not linear. But zoom in close to the point $P$ on the graph.

\begin{exploration}\label{exp:angles2}
Zoom in close to point $P$ by scrolling with your mouse. 
\begin{question} \label{Q1:LF}
What do you notice about the graph of the function very close to $P$?
\end{question}

\pdfOnly{
Access Desmos interactives through the online version of this text at
 
\href{https://www.desmos.com/calculator/3bvwxb95es}.
}
 
\begin{onlineOnly}
    \begin{center}
\desmos{3bvwxb95es}{900}{600}
\end{center}
\end{onlineOnly}
\end{exploration}







\end{document}
