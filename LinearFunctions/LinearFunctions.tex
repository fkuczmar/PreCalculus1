\documentclass{ximera}
\title{Linear Functions}

\newcommand{\pskip}{\vskip 0.1 in}

\begin{document}
\begin{abstract}
The point-slope equation of a line.
\end{abstract}
\maketitle

Linear functions are those that change at a constant rate. For example, between 12:10pm and 12:20pm a balloon might ascends at a constant rate of 50 ft/min or a car might travel at a constant speed of 60 miles/hour. In the first case, the height of the balloon is a linear function of time and in the second the car's odometer reading is also a linear function of time.

Saying that the height of a balloon increases 50 feet every minute is \emph{not} the same as saying the balloon ascends at a constant rate of $50$ ft/min. The former does \emph{not} imply that the balloon ascends at a constant rate. Furthermore, saying that the height increases 50 feet every minute obscures the units of the rate of change as being ft/min. So it's best to stick with the original wording, that the balloon ascends at a constant rate of 50 ft/min.


\begin{question}  \label{Q343rfgg}
The graph of the function
\[
    h = g(t) , 0\leq t \leq 5
\]
expressing the height (in feet) of a balloon in terms of the number of minutes past noon is shown below. Drag the slider $u$ in Line 1 to help answer the following questions.

\begin{freeResponse}
(a) What can you say about the balloon's average rate of ascent over \emph{every} two minute period? Be specific and explain your reasoning. Include units.

(b) Activate the folder in Line 2 by clicking the circle icon at the left of the line. What can you say about the balloon's average rate of ascent over $1/2$-minute periods? Is it constant? If not, estimate the maximum and minimum average rates of ascent over a $1/2$-minute period. Explain your reasoning.
\end{freeResponse}

\pdfOnly{
Access Desmos interactives through the online version of this text at
 
\href{https://www.desmos.com/calculator/uiv2z0pgxx}.
}
 
\begin{onlineOnly}
    \begin{center}
\desmos{uiv2z0pgxx}{900}{600}
\end{center}
\end{onlineOnly}

\href{https://www.desmos.com/calculator/uiv2z0pgxx}{141: Not Constant Rate}

\end{question}


%It can be misleading to say that the height of the balloon increases 50 feet every minute as this could be interpreted as meaning \emph{only} that the height increases only over each one minute interval starting at 12:10pm (ie. from 12:10pm to 12:11pm, from 12:11pm to 12:12 pm, etc.), whereas in fact the height increases at the average rate of 50 ft/min over \emph{every} time interval between 12:10pm and 12:20pm. Furthermore, saying that the height increases 50 feet every minute obscures the units of the rate of change as being ft/min. So it's best to stick with the original wording, that the balloon ascends at a constant rate of 50 ft/min.

Now most functions are decidedly not linear, even over short periods of time. It is unlikely for a balloon to ascend at a contant rate for very long, nor is it really possible to drive at a constant speed for very long. Nevertheless, most functions are locally linear, or at least approximately so, and it is for primarily this reason that they are very useful. For example, over a short period of time we would expect a balloon to ascend or descend at a nearly constant rate and the speed of a car to be almost constant. 

Over long periods of time, some populations tend to grow exponentially. But over a short period of time, the growth is nearly linear. For a dramatic way to see this, take a look at the website below to see how the U.S. and world populations are growing:

\href{https://www.census.gov/popclock/}{Population Clock}

\begin{question} \label{Q1:LF}
Use the above population clock to estimate the rate at which the world (\emph{not} the U.S.) population is currently changing.
\end{question}

We can also see this same phenomenon graphically. Consider, for example, the graph of the function
\[
    T = f(m) , 0\leq m \leq 20 ,
\]
that expresses the temperature (in Celsius degrees) of a cup of coffee in terms of the number of minutes past noon is shown below. The function is not linear. But zoom in close to the point $P$ on the graph.

\begin{exploration}\label{Exp1:LF}
Zoom in close to point $P$ by scrolling with your mouse. 
\begin{question} \label{Q2:LF}
\begin{freeResponse}
(a) What do you notice about the graph of the function very close to $P$?
\end{freeResponse}
(b) Estimate the rate at which the temperature of the coffee is changing at 12:05pm. Explain your reasoning. Be sure to include units.
\end{question}

\pdfOnly{
Access Desmos interactives through the online version of this text at
 
\href{https://www.desmos.com/calculator/3bvwxb95es}.
}
 
\begin{onlineOnly}
    \begin{center}
\desmos{3bvwxb95es}{900}{600}
\end{center}
\end{onlineOnly}

\href{https://www.desmos.com/calculator/3bvwxb95es}{141: Cooling Coffee}

\end{exploration}


Back to our balloon that ascends at a constant rate of 50 ft/min between 12:10pm and 12:20pm. To determine its height at any time during this interval, we need to know the height at one specific time. 

\begin{example} \label{Ex2:LF}
Between 12:10pm and 12:20pm a balloon ascends at a constant rate of 50 ft/min. The balloon is 1428 feet above the ground at 12:13pm. Find a function 
\[
  h = f(t), 10\leq m \leq 20 ,
\]
that expresses the height of the balloon (in feet) in terms of the number of minutes past noon.


\begin{explanation}
Most of you are undoubtedly familiar with the slope-interecept form of an equation of a line, for this example written not as $y=b+mx$, but rather as 
\[
    h=f(t) =h_0 + m t , 10\leq t \leq 20.
\]
This formula relies on knowing the value of $h_0$; this is the balloon's height (in feet) at 12:00pm. But there is no need to know this height, nor do we have enough information to determine it (since we know nothing about the rate of ascent between noon and 12:10pm). Sure, we could assume that the balloon ascends at a constant rate from 12:00pm to 12:20pm and find the height at noon, but that would be an extra step. Furthermore, it would take our attention away from the motion of the balloon over the {\bf short} time interval between 12:10pm and 12;20pm. So {\bf forget about the slope-intercept formula.} For this class and for future math classes, the focus will be on the point-slope equation of a line.

Rather than just plugging numbers into a formula, let's think about how to find the balloon's height at any time between 12:10pm and 12:20pm. Suppose, for example we wish to determine first the balloon's height at 12:20pm relying on the facts that the balloon is 1428 feet hight at 12:13pm and that it ascends at a constant rate of 50 ft/min. 

The key is to compute the {\bf change} in the balloon's height from 12:13pm to 12:20pm. Because this time interval lasts 
\[
   (20 - 13) \text{ min } = 7 \text { min}
\]
and the balloon ascends at a constant rate of 50 ft/min, the change in height is from 12:13pm to 12:20pm is
\[
  \Delta h = f(20) - f(13) = (50 ft/min)(20 - 13 \text{ min})  = 350 ft .
\]
Then adding this change in height to the balloon's height at 12:13pm (1428 feet) gives the height at 12:20pm as
\[
   f(20) = 1428 \text{ ft } + 350 \text{ ft } = 1778 \text{ ft}.
\]
Putting all the computations in one line, the height at 12:20pm is
\begin{align*}
    f(\textcolor{red}{20}) &= 1428 \text{ ft } + (50 ft/min)( \textcolor{red}{20} - 13) \text{ min} . \\
\end{align*}
Or omitting the units, 
\[
     f(\textcolor{red}{20}) = 1428  + 50 ( \textcolor{red}{20} - 13 ) . 
\]

\begin{question}  \label{Qdftr343}
(a) Let's compute the balloon's height at 12:10pm in the same way. The change in height, now going backward in time from 12:13pm to 12:10pm is
\[
      \Delta h  = f(\answer{10}) - f(\answer{13}) = (\answer{50} \text{ ft/min})(\answer{10} - \answer{13})\text{ min} = \answer{-150} \text{ ft} .
\]
So the height of the balloon at 12:10pm is
\[
       f(\answer{10}) = \answer{1428}\text{ ft} + (\answer{50} \text{ ft/min})(\answer{10} - \answer{13})\text{ min} = \answer{1278} \text{ ft}.
\]

(b) Now to find the height at any time $t$ between $10$ and $20$ minutes past noon, we need only replace the \textcolor{red}{00} in the last equation with $\textcolor{red}{t}$. So omitting the units, the balloon's height (in feet) at $t$ minutes past noon is given by the function
\[
       f(\answer{t}) = \answer{1428}  + \answer{50}( \answer{t} - \answer{13}) , 10\leq t \leq 20.                   % f(\textcolor{red}{t}) = 1428  + 50( \textcolor{red}{t} - 13) , 10\leq t \leq 20.
\]

We should keep our focus on the time interval between 12:10pm and 12:20pm by {\bf stopping} here. No need to distribute and simplify. 

\begin{freeResponse}
(c) Explain the logic behind the above expression for $f(t)$. Include complete explanations (with units) of 

(i) the difference $\textcolor{red}{t} - 13$ ,

(ii) the product $50(\textcolor{red}{t}-13)$ , and

(iii) the sum 1428  + $50( \textcolor{red}{t} - 13)$  .

Identify each of these three expressions (a) - (c) on the graph below. Include a copy of the graph with your explanation. 
\end{freeResponse}

\pdfOnly{
Access Desmos interactives through the online version of this text at
 
\href{https://www.desmos.com/calculator/iyva54o8i5}.
}
 
\begin{onlineOnly}
    \begin{center}
\desmos{iyva54o8i5}{900}{600}
\end{center}
\end{onlineOnly}

\href{https://www.desmos.com/calculator/iyva54o8i5}{142: Balloon Constant Rate}

\end{question}

\end{explanation}
\end{example}



Now let's go back to the cooling coffee cup and see if we can approximate the temperature function near the point $P$.

\begin{question} \label{Ex3:LF} The graph of the function
\[
    T = f(m) , 0\leq m \leq 20 ,
\]
that expresses the temperature (in Celsius degrees) of a cup of coffee in terms of the number of minutes past noon is shown below. 

(a) Zoom in close enough to point $P$ to be able to get a good approximation to the coordinates of point $Q$. What do you notice about the graph in this small viewing window?

(b) Use the coordinates of points $P$ and $Q$ to find an expression for the linear function 
\[
  T = g(m) , 0\leq m \leq 20,
\]
that gives a good approximation to the temperature function $T=f(m)$ for times near 12:05pm. Use the point-slope equation of a line. Explain your reasoning by following Example \ref{Ex2:LF}.

(c) Graph the linear function $T=g(m)$ by entering the correct expression on Line 9 below.

(d) Zoom in toward $P$ and back out. What can you say about how well the linear function approximates the actual temperature function?

\pdfOnly{
Access Desmos interactives through the online version of this text at
 
\href{https://www.desmos.com/calculator/vce8fcabt7}.
}
 
\begin{onlineOnly}
    \begin{center}
\desmos{vce8fcabt7}{900}{600}
\end{center}
\end{onlineOnly}

\href{https://www.desmos.com/calculator/vce8fcabt7}{141: Cooling Coffee}

\end{question} 








\begin{example}  \label{Ex5:LF}
Suppose that between prices of $\$6.00/\text{burger}$ and $\$10.00/\text{burger}$ the average number of burgers Five Guys in Edmonds sells is a linear function of the price. They sell an average of $200$ burgers/day at a price of $\$6.00/\text{burger}$ and an average of $120$ burgers/day at a price of $\$10.00/\text{burger}$. Find a function that expresses the price in terms of the average number of burgers they sell.

\begin{explanation}
(a) {\bf Start by defining each variable in a complete sentence. Include units in the definitions (the units should be $\$$/burger and burgers/day). Use {\bf meaningful} variable names (a single letter for each). Do {\bf not} use $x$ and $y$.}

\pskip

We will let $p$ be the price (in $\$$/burger) and $q$ be the average number of burgers (measured in burgers/day) sold. %that Five Guys sells at a pirce of $p$  dollars/burger.

\pskip

(b) {\bf Relate the variables from part (a) with function notation.} % as in Example 2, where we let $T=f(m)$.}

\pskip

The problem asks us to express $p$ as a function of $q$. So we are looking for a function
\[
   p = f(q) , 120 \leq q \leq 200 ,
\]
that expresses the price (in dollars/burger) in terms of the average number of burgers (in burgers/day) they sell.

\pskip

%{\bf Note:} Although the price determines the average number of burgers they sell, the question asks us to express the price as a function of quantity. That isthat if we know the quantity we could determine the price. This is how economists do things.

%\pskip

(c) {\bf Sketch a graph of the function. Label axes with the appropriate variable names and units.}

\pskip 

I'll omit this part, but you should include a graph. In this case the graph of the function $p=f(q)$ would be a line  segment with endpoints $(120, 10)$ and $(200, 6)$.

\pskip

(d) {\bf Compute the appropriate rate of change of the linear function. Include units in your computation and use the delta notation for changes. Interpret the meaning of the slope.}

\pskip

The rate of change in the price with with respect to the average number of burgers sold is
\[
   \frac{\Delta p}{\Delta q} = \frac{(10-6) \text{dollars/burger}}{(120-200) \text{burgers/day}} = -\frac{1}{20} \frac{\text{dollars/burger}}{\text{burgers/day} } .
\]
So if Five Guys raise the price by 1 dollar/burger, they sell an average of 20 fewer burgers/day. Or if they raise the price by $\$0.05$/burger, they sell an average of 1 fewer burgers/day.
\pskip

{\bf Do not simplify the units, otherwise it will be impossible to interpret the rate of change.}

\pskip

(e) {\bf Use your slope and a known data point to write an expression for the function in point-slope form. Include a domain.}

Because $p=6$ when $q=200$, 
\[
  p = f(q) = 6 - \frac{1}{20} \left( q - 200 \right) , \text{ for } 120\leq q \leq 200 . 
\]
Or  because $p=10$ when $q=120$, 
\[
  p = f(q) = 10 - \frac{1}{20} \left( q - 120 \right) \text{ for } 120\leq q \leq 200.
\]
\end{explanation}
{\bf Leave the expressions like this. No need to simplify.}

\pskip 

(f) {\bf Explain the logic behind your function.}

The function
\[
   p = f(q) = 6 - \frac{1}{20} \left( q - 200 \right) , \text{ for } 120\leq q \leq 200
\]
tells us that to get the price $p$ when they sell $q$ burgers/day, we take the price ($\$6$/burger) when they sell 200 burgers/day and add to that the change in the price when the number of burgers they sell changes from 120 burgers/day to $q$  burgers/day.

\pskip

(g) {\bf Check that your function is correct.}

Using
\[
  p = f(q) = 6 - \frac{1}{20} \left( q - 200 \right) , \text{ for } 120\leq q \leq 200 , 
\]
we'll evaluate $f(200)$ to get
\[
    f(200) = 6 - \frac{1}{20} \left( 200 - 200 \right) =6 .
\]
This is correct since when they sell 200 burgers/day the price is $\$6$/burger.

And we'll evaluate $f(120)$ to get
\[
  f(120) = 6 - \frac{1}{20} \left( 120 - 200 \right)  = 6 - (-4) = 10 .
\]
That's also correct. When they sell 120 burgers/day the price is $\$10$/burger.


\end{example}



\begin{question}   \label{Qdfsdfbbhhh}
Suppose that between altitudes of $1,000$ and $12,000$ feet above sea level the air temperature is a linear function of altitude. Suppose also that the temperature is $57^\circ$F at at altitude of $2,000$ feet and $42^\circ$F at an altitude of $9,000$ feet.

(a) Find a function
\[
    T = f(h) ,
\]
that expresses the temperature (in Fahrenheit degrees) in terms of the altitude (measured in thousands of feet). First sketch a graph of the function $f$. Label the axes with the appropriate variable names and units. Be sure to consider the domain.

(b) Use set-builder notation to describe the domain of $f$.

(c) Find the rate of change of temperature with respect to altitude and explain what this means. Be sure to include units.

(d) Explain the \emph{logic} behind your function in this particular scenario.

(e) Use your function to determine the temperature at an altitude of $8,300$ feet.

(f) Use your function to determine the altitude where the temperature is $48^\circ$F.

(g) Write and solve an inequality to determine at what altitudes the temperature is at most $49.5^\circ$F.

(h) At what altitude is the temperature $3^\circ$F greater than it is at an altitude of $8732$ feet? Answer this without using a calculator.

(i) What happens to the air temperature if you increase your altitude by $4200$ feet?

(j) If you increase your altitude by $70\%$, the air temperature decreases by $5.1^\circ$F. What is your altitude?

\end{question}


\begin{question}  \label{Qdsfdsfgnn}
Suppose that between prices of $\$4$/burger and $\$7$/burger the average number of burgers sold per day is linear function of the price. Suppose also that an average of 102 burgers/day are sold at a price of $\$4$/burger and that an average of 67 burgers/day are sold at a price of $\$7$/burger.


(a) Find a function that expresses the price (measured in $\$$/burger) in terms of the average number of burgers sold (measured in burgers/day). Do \emph{not} simplfiy your expression by distributing. Be sure to start by defining your variables and the function’s name. Do not use x and y as the variables. Choose variables with more meaningful names. Include a graph to help with
your explanation. Include the domain of the function. See the class videos for help and use the point-slope form of an equation of a line. No credit given for any other methods.

(b) Use your function from part (a) to determine the price when an average of 82 burgers/day are sold.

(c) Use your function from part (a) to determine the average number of burgers sold per day at a price of $\$5.25$/burger.

(d) What is the rate of change of the price with respect to the average number of burgers sold per day? Include units. Then interpret the meaning of this rate of change in three equivalent but different ways.

(e) What happens to the average number of burgers sold per day if the price decreases by $\$0.30$/burger? Answer this question in complete generality as in the class videos.

\end{question}



\begin{example} \label{Ex6:LF}
Suppose that between speeds of $60$ miles/hour and $80$ miles/hour, the speed of a car is a linear function of its gas mileage. 

\pskip

Choose speeds for the endpoints of this interval and reasonable gas mileages for those speeds. Then find a function that expresses the {\bf speed in terms of the gas mileage} by following steps (a)-(g) in Example 8. Give complete explanations. Include a graph \emph{sketched by hand} to help with your explanations.
\end{example}


\begin{question}  \label{Qefltktl4gg}
Between speeds of $55$ miles/hour and $90$ miles/hour, the gas mileage of a Corvette is a linear function of its speed. The car gets $35$ miles/gallon at a speed of $55$ miles/hour and $20$ miles/gallon at a speed of $90$ miles/hour.

\begin{enumerate}

\item Find a function $v=f(G)$ that expresses the speed of the Corvette (in miles/hour) in terms of its gas mileage (in miles/gallon). Start by sketching a graph of the function. Label the axes with the appropriate variable names and units.

\item Use set notation to state the domain and range of $f$.

\item What are the units of the slope of your line in part (a)? Interpret its meaning.

\item At what speed does the car get $30$ miles/gal?

\item What is the gas mileage at a speed of $80$  miles/hour?

\item What happens to the gas mileage when the speed of the car decreases by $5$ miles/hour?

\item How can you increase the gas mileage by $4$ miles/gal?

\item Find a function $r=h(G)$ that expresses the rate at which the car burns gas (in gal/hour) in terms of its gas mileage. State the domain and range of $h$ in set notation.

\item Simplify the units of the slope in part (c). What does this suggest about its meaning? Would that be a correct interpretation?

\item Drag the slider $u$ in the worksheet below to approximate the rate (in miles/hr) at which the Corvette burns gas at a speed of $86$ miles/hour. Then use the appropriate function to compute the exact rate.

\item Drag the slider $u$ in the worksheet below to approximate the speed at which the Corvette burns gas at the rate of $2$ gal/hour. Then use the appropriate function to compute the exact rate.

\begin{onlineOnly}
    \begin{center}
\desmos{2fq4txcouq}{900}{600}
\end{center}
\end{onlineOnly}

\href{https://www.desmos.com/calculator/2fq4txcouq}{141: Corvette}

\item Find a function $G=w(r)$ that expresses the gas mileage (in miles/hour) of the Corvette in terms of the rate (in gal/hour) at which it burns gas. State the domain and range $w$ in set notation.

\end{enumerate}

\end{question}


\begin{question}  \label{Qdfg54t54t}
The graph of the linear function $f$ passes through the points $A(5,-7)$ and $B(-2,2)$.

(a) Sketch by hand a graph of the function $y=f(x)$. Label the axes with the appropriate variable names.

(b) Use point-slope to find an expression for $f(x)$. Do \emph{not} simplify your expression by distributing.

(c) Use algebra to solve the equation $f(x) = -106$.

(d) Use algebra to solve the inequality $f(x) < 86$.

(e) Use algebra to solve the inequality $f(x) \geq f(425)$.

(f) Use the graph of $f$ to solve the inequality $f(x)\leq f(-453)$.

(g) Use algebra to find all inputs to the function $f$ that are equal to their outputs.

(h) Use algebra to find the exact coordinates of the points where the graph of $f$ intersects the circle of radius $5$ centered at the point $B(-2,2)$.

\end{question}


\begin{question}  \label{Q324fggg}
Follow the methods of \emph{Example 8} for this problem.

Between speeds of 26 miles/hour and 68 miles/hour, the gas mileage of a car is a linear function of its speed. The car gets 41 miles/gallon at a speed of 33 miles/hour and 29 miles/gallon at a speed of 61 miles/hour.

(a) Sketch by hand a graph of the function 
\[
      s = f (G)
\]
that expresses the speed of the car (in miles/hr) in terms of its gas mileage (in miles/gallon). Label the axes with the appropriate variable names and units. Be sure to consider the appropriate domain.

(b) Compute the rate of change of the linear function $f$. Include units in your computation and use the delta notation for changes with the appropriate variable names. 

(c)  Interpret the meaning of the rate of change in part (b). 

(d) Find an expression for the function 
\[
   s = f (G). 
\]
Do not “simplify” your expression by distributing. Use the point-slope form of an equation of a line.

(e) Find the domain of the function f . Then modify your graph from part (a) if necessary. Label the coordinates of the endpoints of the graph.


(f) Use your function to determine the gas mileage at a speed of 50 miles/hour.

(g) Use  your function to determine the speed at which the car gets $38$ miles/gal.

(h ) What happens to the gas mileage if you decrease your speed by 5 miles/hour?

(i) At what speed is the gas mileage $5$ miles/gal greater than it is at $65$ miles/hr?


\end{question}


\end{document}
