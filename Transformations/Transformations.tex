\documentclass{ximera}
\title{Transformations}

\newcommand{\pskip}{\vskip 0.1 in}

\begin{document}
\begin{abstract}
Transforming the graphs of functions and relations.
\end{abstract}
\maketitle



\begin{question}  \label{Q:324bt44t}
(a) Sketch the curve
\[
    x^2 + y^2 = 25 .
\]

(b) Describe a composition of transformations that takes the curve
\[
        x^2 + y^2 = 25
\]
to each of the following curves. Sketch each curve along with the images of the points $(\pm 5,0)$, $(0,\pm 5)$, $(3,4)$ under the tranformations.

\pskip

(i) the curve
\[
          (x-3)^2 + (y+4)^2 = 25 .
\]

(ii)  the curve
\[
          x^2 +10x +y^2 - 8x = 0 .
\]

(iii) the curve
\[
       \left( \frac{x}{5} \right)^2 + \left( \frac{y}{2} \right)^2 = 25 .
\]

(iv) the curve
\[
       \left( \frac{x-3}{5} \right)^2 + \left( \frac{y-4}{2} \right)^2 = 25 .
\]

\end{question}


\begin{question}  \label{Q:3d3rfff}
The graph of the function
\[
   h = f(t) \,  ,  \, -2 \leq t \leq 4,
\]
expressing the height of a balloon, measured in thousands of feet, in terms of the number of hours past noon is shown below. Use the graph to answer the following questions.


\begin{onlineOnly}
    \begin{center}
\desmos{xupooue7ew}{900}{600}
\end{center}
\end{onlineOnly}

Desmos activity available at \href{https://www.desmos.com/calculator/xupooue7ew}{141: Transformations 2}


(a) Write the domain of $f$ in set-builder (not interval) notation.

(b) Write the range of $f$ in set-builder (not interval) notation.

(c) Let 
\[
    g_1(t) = f(t+3) - 2 .
\]
 
(i) Write the domain and range of $g$ in set builder notation.

(ii) Evaluate each of the following expressions if possible.

\begin{itemize}
\item{$g_1(1)$}

\item{$g_1(-3)$}

\item{$g_1(2)$}

\item{$g_1(-2)$}

\item{$g_1(4)$}
\end{itemize}

(ii) Solve each of the following equations.

\begin{itemize}
\item{$g_1(t) = 1$}

\item{$g_1(t) = 7$}

\item{$g_1(t) = 11$}

\item{$g_1(t) = 13$}

\item{$g_1(t) = 14$}
\end{itemize}

(iii) Use the results of parts (i) and (ii) to sketch a graph of the function 
\[
      h = g_1(t).
\]

(iv) Describe a composition of transformations that takes the graph of $h=f(t)$ to the graph of $h=g_1(t)$.

\end{question}


\begin{question}  \label{Q:vvvcccc}
This is a continuation of the previous problem with the same function $f$ and its graph.

\pskip

This time we'll define 
\[
     g_2(t) = \frac{1}{2}f(2t) .
\]

(a) Write the domain and range of $g_2$ in set-builder notation.

(b) Find the coordinates of the points on the graph of $h=g_2(t)$ corresponding to the points with $t$-coordinates
\[
 t = -2, -1, 0, 2, 3, 4
\]
on the graph of $h=f(t)$.

(c) Use the results of part (b) to sketch the graph of $h=g_2(t)$.

(d) Describe a composition of transformations that takes the graph of $h=f(t)$ to the graph of $h=g_2(t)$.

(e) Repeat parts (a) - (d) for the following functions.

\pskip


(i)  $h=g_3(t) = 2f(t/2)$

(ii) $h = g_4(t) = f(-t)$

(iii) $h = g_5(t) = 15 - f(t)$

(iv) $h = g_6(t) = f(6-t)$

(v) $h = g_7(t) = 15 - f(6-t)$

(vi) $h=g_8(t) = f(2t+8)$



\end{question}


\end{document}