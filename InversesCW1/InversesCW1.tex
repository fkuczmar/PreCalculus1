\documentclass{ximera}
\title{Inverse Functions CW1}

\newcommand{\pskip}{\vskip 0.1 in}

\begin{document}
\begin{abstract}
Pairs of inverse functions.
\end{abstract}
\maketitle


\begin{question} \label{QKfmd4903}
\begin{enumerate}
\item Describe the action of each of the following functions.

\item Then describe the action of the inverse function.

\item Use your description of the inverse to find an expression for the function $x=f^{-1}(y)$.

\item Use set-builder notation to state the domains of the functions $y=f(x)$ and $x=f^{-1}(y)$.

\item Use algebra to simplify the compositions
\[
    g(x) =  f^{-1}(f(x)) 
\]
and 
\[
     h(y) =  f(f^{-1}(y)) .
\]

\item  Graph the functions $X = g(x)$ and $Y = h(y)$.
\end{enumerate}

Here are the functions.

\begin{enumerate}
\item $f(x) = \frac{23}{17}x+5$

\item $f(x)=12-x$

\item $f(x) = 4 - \frac{6}{13}\left( x - 5 \right)$  \hskip 0.2 in \emph{Do not distribute}

\item $f(x) = \frac{24}{x}$

%\item $f(x) = \frac{20}{x-5}$

\item $f(x) = 4 - \frac{10}{x+3}$

\item $f(x) = 5 + \frac{10}{x-5}$

\item $f(x) = \frac{3x-5}{x+2}$  \hskip 0.2 in \emph{Hint:} Divide first.
\end{enumerate}

\end{question}


\begin{question} \label{Qpferpr39033}
The function
\[
       Q = f(P) \, , \, 5 \leq P \leq 15,
\]
expresses the average number of burgers a diner sells per day in terms of the price (in dollars/burger).

\begin{enumerate}

\item Would you expect $f$ to be an increasing or a decreasing function? Explain your reasoning.

\item Make the assumption in part (a) and interpret the meanings of each of the following expressions.

\begin{enumerate}
\item $f(10)$

\item $f^{-1}(30)$.

\item $f^{-1}(f(6)-10)$

\item $f(f^{-1}(30)-2)$

\item $f^{-1}(f(P)-10)$

\item $Q - f(f^{-1}(Q)-2)$

\end{enumerate}

\item Suppose the function $f$ is linear. Suppose also the diner sells an average of $53$ burgers/day when the price is $\$5$/burger and an average of $10$ burgers/day when the price is $\$15$/burger.

\begin{enumerate}
\item Find an expression for the function $Q=f(P)$.

\item Evaluate $f(f^{-1}(30)-2)$

\item Let 
\[
      p = g(P) =  f^{-1}(f(P)-10).   
\]

\begin{enumerate}
\item Find a simplified expression for the function $p=g(P)$.

\item Use set builder notation to write the domain and range of the function $p=g(P)$.

\item Graph the function $p=g(P)$. Label the coordinate axes with the appropriate variable names and units.
\end{enumerate}
\end{enumerate}
\end{enumerate}
 \end{question}

\begin{question} \label{Ex1:Qeuadratics}

\begin{enumerate}
\item The function $V=f(t)$, $0\leq t \leq 6$, expresses the volume of water in a tank, measured in gallons, in terms of the number of minutes past noon. 

\begin{enumerate}
\item Suppose the function $f$ is one-to-one. Explain what this would mean in everday English as it applies in this particular situation. Do not use any math terms (like input or output).

\item Suppose the function $f$ is \emph{not} one-to-one. Explain what this would mean in everday English as it applies in this particular situation. %Do not use any math terms (like input or output).

\end{enumerate}

\item Now suppose 
\[
     V = f(t) = \frac{1}{3}(t-6)^2 \, , 0\leq t \leq 6 .
\]

\begin{onlineOnly}
    \begin{center}
\desmos{ef7obzoxx3}{900}{600}
\end{center}
\end{onlineOnly}

\href{https://www.desmos.com/calculator/ef7obzoxx3}{141: Tank 5}

\begin{enumerate}
\item Use the graph of $V=f(t)$ above to determine if the inverse of the function $f$ also a function. Explain your reasoning. Explain also what this means in the context of this particular scenario.

\item Desribe the function $f^{-1}$ in this scenario. What does it take an input? What does it return as an output?

\item Find an expression for the function $t = f^{-1}(V)$. Write its domain using set notation.
\[
      \{ \answer{V} \, | \, \answer{0} \leq \answer{V} \leq \answer{12} \}
\]

\item Simplify the composition $f^{-1}\circ f$ algebraically. Use the approprite variable name for the input.

\item  Simplify the composition $f\circ f^{-1}$ algebraically. Use the approprite variable name for the input.

\item Drag point $P$ in the animation above to approximate a three-minute time inteval during which $6$ gallons of water drain form the tank. 

\item Use algebra to find a three-minute time inteval during which $6$ gallons of water drain form the tank. \emph{Start by writing a complete sentence that defines an unknown with units.} 

Check your answer by finding
\begin{enumerate}
\item the volume at the start of the three-minute interval and

\item the volume at the end of the three minute interval.
\end{enumerate}

\item Interpret the meaning of the function
\[
       W = g(V)= f(f^{-1}(V) + 3) 
\]

\item Solve the equation 
\[
   g(V) = V -6
\]
and interpret the meaning of the solution.
\end{enumerate}

\end{enumerate}

\end{question}


\begin{question}  \label{Ex:oodf35rfdsfo}
The function
\[
   G = f(v) = 62 - \frac{1000}{v+10}\,  , \, 20\leq v \leq 55 ,
\]
expresses the gas mileage of a car (in  miles/gal) in terms of its speed (in miles/hr). 

\begin{enumerate}

%\item Solve the appropriate equation to find the speed at which the car gets $37$ miles/gallon.

\item Describe the action of $f$ on its input.

\item Explain the meaning of the function $f^{-1}$. What does it take as an input? What does it return as an output?

\item Use  your description from part (a) to describe how $f^{-1}$ acts on its input. Then use this description to find an expression for the function 
\[
    v = f^{-1}(G).
\]

\item Express the domain of this function in set-builder (\emph{not} interval) notation.

\item At what speed does the car get $37$ miles/gal?

 
\item Suppose increasing your speed by 10 miles/hour increases  your gas mileage by $5$ miles/gal. Find your original speed. Then check your answer in the words of this question, not in your equation.

\end{enumerate}



\end{question}



\end{document}
