\documentclass{ximera}
\title{The Pythagorean Theorem}


\newcommand{\pskip}{\vskip 0.1 in}

\begin{document}
\begin{abstract}
Pythagorean Theorem.
\end{abstract}
\maketitle


\pskip


\begin{question}  \label{Q324dfg45rhp}
Imagine a one-mile long railroad track secured only at its two ends to the desert floor. Suppose on a hot day the track expands one foot and buckles into a curve. We would like to compute, or at least approximate, the height of the track above the ground at its midpoint. We really need calculus to determine the shape of the curve and the exact height, but we can get a pretty good approximation by supposing the track bends into two straight segments running from the high point to its endpoints.


\begin{freeResponse}
(a) Before doing any computation, guess the height of the track at its midpoint. What do you think? Note that there are 5280 feet in one mile.
\end{freeResponse}

(b) Now compute the height. How accurate was your estimate?
\end{question}


\begin{question}  \label{Qdfdst4nb554334}
(a) The bottom end of a $20$-foot ladder rests on level ground, $15$ feet away from a vertical wall. The top end rests against the wall. How high above the ground is the top end?

(b) One end of an $L$-foot long ladder rests on the ground, the other against a vertical wall. Find a function 
\[
  h = f(s) \, , \, 0\leq s \leq L,
\]
that expresses the height (in feet) of the top end above the ground in terms of the distance (in feet) from the bottom end to the wall.

(c)  Graph the function $h = f(s)$ with $L=20$.

(d) Find an equation of the ladder in the coordinate system below. Your equation should express $h$ in terms of $s$ and $L$ (do not assume $L=20$). It should also include the correct domain. Then move the slider $s$ to see if you are correct.

\begin{exploration}

\begin{onlineOnly}
    \begin{center}
\desmos{orpmncb1dx}{900}{600}
\end{center}
\end{onlineOnly}

\href{https://www.desmos.com/calculator/orpmncb1dx}{141: Ladder}

\end{exploration}
\end{question}


\begin{question}  \label{Qdfsf4tnbnt}
One end of a $10$-foot long rope is attached to the top of a $12$-foot tall pole. The other end of the rope is attached to a hat.

A gremlin, at most $12$ tall but perhaps very short, puts the hat on its head and walks away from the pole until the rope is taut.

(a) Find a function
\[
      h = f(s)
\]
that expresses the gremlin's height (in feet) in terms of its distance from the pole (in feet).

(b) Find the domain and range of the function $f$.

(c) Graph the function $f$. 

(d) Find a function
\[
     s = g(h)
\]
that expresses the gremlin's distance from the pole (in feet)  in terms of its height (in feet).

(e) Find the domain and range of $g$.

(f) Graph the function $g$.

(g) How would your functions in parts (a) and (d) change (if at all), if instead the gremlin is at least 12 feet tall? 

\end{question}


\begin{question}  \label{Qdfthhrhghgr}
On a clear day with an unobstructed view (like you might have at the beach or in a hot air balloon), the distance to the horizon is limited by the curvature of the earth as illustrated in the demonstration below.

\pdfOnly{
Access Desmos interactives through the online version of this text at
 
\href{https://www.desmos.com/calculator/ewowig5sgk}.
}
 
\begin{onlineOnly}
    \begin{center}
\desmos{ewowig5sgk}{900}{600}
\end{center}
\end{onlineOnly}

Desmos activity available at

\href{https://www.desmos.com/calculator/ewowig5sgk}{151:Distance to Horizon 1}

\pskip

The actual function 
\[
        s= f(h) \, , \, 0\leq h \leq 100,000 ,
\]
expresses the approximate distance to the horizon (the length of the red arc $AT$ above, measured in miles) in terms of your height above the ground (the distance $AP$ above, measured in feet). But for this we need trigonometry, so instead we will takdee the distance to the horizon $s$ to be the length of segment $\overline{AP}$.

We need one key fact - that the line of sight $\overline{PT}$ from the balloon to the horizon is tangent the earth at $P$. This makes $\angle PTO$ a right angle. That and the radius of the earth ($3960$ miles) is all you need to know to find an expression for the function $f$. Keep in mind that the input to $f$ is measured and feet and the output is measured in miles. Remember there are $5280$ feet in one mile.

The function
\[
        s= f(h)  = \answer{\sqrt{\left( 3960 + \frac{h}{5280}\right)^2 - 3960^2}}   \, , \, 0\leq h \leq 10,000 ,
\]
expresses the approximate distance to the horizon (the length of segment $\overline{PT}$ above, measured in miles) in terms of your height above the ground (the distance $AP$, measured in feet).


Explain why the function
\[
      s = g(h) =  1.22\sqrt{h}, 0\leq h \leq 100,000 ,
\]
gives a good approximation to $f$. Then use desmos to check this.


\pskip \pskip




\end{question}

\end{document}




