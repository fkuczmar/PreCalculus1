\documentclass{ximera}
\title{The Pythagorean Theorem}


\newcommand{\pskip}{\vskip 0.1 in}

\begin{document}
\begin{abstract}
Pythagorean Theorem.
\end{abstract}
\maketitle


\pskip


\begin{question}  \label{Q324dfg45rhp}
Imagine a 2000-foot long railroad track secured only at its two ends to the desert floor. Suppose on a hot day the track expands one foot and buckles into a curve. We would like to compute, or at least approximate, the height of the track above the ground at its midpoint. We really need calculus to determine the shape of the curve and the exact height, but we can get a pretty good approximation by supposing the track bends into two straight segments running from the high point to its endpoints.


\begin{freeResponse}
(a) Before doing any computation, guess the height of the track at its midpoint. What do you think?
\end{freeResponse}

(b) Now compute the height. How accurate was your estimate?
\end{question}


\begin{question}  \label{Qdfdst4nb554334}
(a) The bottom end of a $20$-foot ladder rests on level ground, $15$ feet away from a vertical wall. The top end rests against the wall. How high above the ground is the top end?

(b) One end of an $L$-foot long ladder rests on the ground, the other against a vertical wall. Find a function 
\[
  h = f(s) \, , \, 0\leq s \leq L,
\]
that expresses the height (in feet) of the top end above the ground in terms of the distance (in feet) from the bottom end to the wall.

(c)  Graph the function $h = f(s)$ with $L=20$.

(d) Find an equation of the ladder in the coordinate system below. Your equation should express $h$ in terms of $s$ and $L$ (do not assume $L=20$). It should also include the correct domain. Then move the slider $s$ to see if you are correct.

\begin{exploration}

\begin{onlineOnly}
    \begin{center}
\desmos{orpmncb1dx}{900}{600}
\end{center}
\end{onlineOnly}

\href{https://www.desmos.com/calculator/orpmncb1dx}{141: Ladder}

\end{exploration}
\end{question}


\begin{question}  \label{Qdfsf4tnbnt}
One end of a $10$-foot long rope is attached to the top of a $12$-foot tall pole. The other end of the rope is attached to a hat.

A gremlin, at most $12$ tall but perhaps very short, puts the hat on its head and walks away from the pole until the rope is taut.

(a) Find a function
\[
      h = f(s)
\]
that expresses the gremlin's height (in feet) in terms of its distance from the pole (in feet).

(b) Find the domain and range of the function $f$.

(c) Graph the function $f$. 

(d) Find a function
\[
     s = g(h)
\]
that expresses the gremlin's distance from the pole (in feet)  in terms of its height (in feet).

(e) Find the domain and range of $g$.

(f) Graph the function $g$.

\end{question}


\begin{question}  \label{Qdfthhrhghgr}
Imagine yourself at a Pacific coast beach on a clear day. While it might seem you can see forever, the distance to the horizon is limited by the curvature of the earth. Our aim in this problem is to find a function 
\[
        s= f(h) \, , \, 0\leq s \leq 10,000 ,
\]
that expresses the distance to the horizon (in miles) in terms of your height above the ground (in feet).
\end{question}

\end{document}




