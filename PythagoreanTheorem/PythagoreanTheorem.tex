\documentclass{ximera}
\title{The Pythagorean Theorem}


\newcommand{\pskip}{\vskip 0.1 in}

\begin{document}
\begin{abstract}
Pythagorean Theorem.
\end{abstract}
\maketitle


\pskip


\begin{question}  \label{Q324dfg45rhp}
Imagine a 2000-foot long railroad track secured only at its two ends to the desert floor. Suppose on a hot day the track expands one foot and buckles into a curve. We would like to compute, or at least approximate, the height of the track above the ground at its midpoint. We really need calculus to determine the shape of the curve and the exact height, but we can get a pretty good approximation by supposing the track bends into two straight segments running from the high point to its endpoints.


\begin{freeResponse}
(a) Before doing any computation, guess the height of the track at its midpoint. What do you think?
\end{freeResponse}

(b) Now compute the height. How accurate was your estimate?
\end{question}

\begin{question}  \label{Qdfdst4nb554334}
(a) The bottom end of a $20$-foot rests on level ground, $15$ feet away from a vertical wall. The top end rests against the wall. How high above the ground is the top end?

(b) One end of an $L$-foot long ladder rests on the ground, the other against a vertical wall. Find a function 
\[
  h = f(s) \, , \, 0\leq s \leq L,
\]
that expresses the height (in feet) of the top end above the ground in terms of the distance (in feet) from the bottom end to the wall.

(c)  Graph the function $h = f(s)$ with $L=20$.

(d) Find an equation of the ladder in the coordinate system below. Your equation should express $h$ in terms of $s$ and $L$ (do not assume $L=20$). It should also include the correct domain. Then move the slider $s$ to see if you are correct.

\begin{exploration}

\begin{onlineOnly}
    \begin{center}
\desmos{9dmlpjfvfl}{900}{600}
\end{center}
\end{onlineOnly}

\href{https://www.desmos.com/calculator/9dmlpjfvfl}{141: Ladder}

\end{exploration}

\end{question}



\end{document}




