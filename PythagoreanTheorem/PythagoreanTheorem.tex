\documentclass{ximera}
\title{The Pythagorean Theorem}


\newcommand{\pskip}{\vskip 0.1 in}

\begin{document}
\begin{abstract}
Pythagorean Theorem.
\end{abstract}
\maketitle


\pskip

\section{The Pythagorean Theorem}

A reflexive response when asked about the Pythagorean theorem is to say $a^ + b^2 = c^2$. But this is incomplete without giving the meanings of $a$, $b$, and $c$. They are the lengths of the sides of a \emph{right} triangle. Here $a$ and $b$ are the lengths of the two \emph{legs}. These are the sides that form the right angle. They are the two shorter sides in the triangle. And $c$ is the length of the \emph{hypotenuse}, the side opposite the right angle. This is the longest side.

The Pythagorean theorem in this form is really a statement about areas. 

\begin{onlineOnly}
    \begin{center}
\desmos{y2o9zqibsv}{900}{600}
\end{center}
\end{onlineOnly}

\href{https://www.desmos.com/calculator/y2o9zqibsv}{141: Pythagorean Thm Statement}

\begin{question} \label{QLdfer3}
What does the Pythagorean Theorem say about the areas of the three squares above?
\begin{freeResponse}
\end{freeResponse}
\end{question}


But there's nothing special about squares. Any three curves with the same shape will do. For example, we could construct semicrircles on the outside of each side like this.

\begin{onlineOnly}
    \begin{center}
\desmos{yuminrnigw}{900}{600}
\end{center}
\end{onlineOnly}

\href{https://www.desmos.com/calculator/yuminrnigw}{141: Pythagorean Thm Statement 2}

\begin{question} \label{QLdfer3r}
What does the Pythagorean Theorem say about the areas of the three seimicircles above? Explain why.
\begin{freeResponse}
\end{freeResponse}
\end{question}

Or equilateral triangles.

\begin{onlineOnly}
    \begin{center}
\desmos{aqs8soxmvq}{900}{600}
\end{center}
\end{onlineOnly}

\href{https://www.desmos.com/calculator/yuminrnigw}{141: Pythagorean Thm Statement 3}

\begin{question} \label{QLdfe44r3r}
What does the Pythagorean Theorem say about the areas of the three triangles above? Explain why.
\begin{freeResponse}
\end{freeResponse}
\end{question}

But probably the best choice, the one that let's us see immediately \emph{why} the Pythagorean theorem is true, would be to construct three triangles with the same shape as right triangle $\Delta ABC$ and just the right size.


\begin{onlineOnly}
    \begin{center}
\desmos{oaxj7u8xly}{900}{600}
\end{center}
\end{onlineOnly}

\href{https://www.desmos.com/calculator/oaxj7u8xly}{141: Pythagorean Thm Statement 4}

\begin{question} \label{Qer3Ldfe44r3r}
Explain how this picture \emph{proves} the Pythagorean theorem. 
\begin{freeResponse}
\end{freeResponse}
\end{question}




\section{A Buckled Railroad Track}



\begin{question}  \label{Q324dfg45rhp}
Imagine a $5$-km long railroad track secured only at its two ends to the desert floor. Suppose on a hot day the track expands one meter and buckles into a curve. We would like to compute, or at least approximate, the height of the track above the ground at its midpoint. We really need calculus to determine the shape of the curve and the exact height, but we can get a pretty good approximation by supposing the track bends into two straight segments running from the high point to its endpoints.

\begin{enumerate}
\item Without doing any computation, guess the height of the track at its midpoint. What do you think? 
\begin{freeResponse}
\end{freeResponse}

\item Now use the Pythagorean theorem to approximte the height. Doe not use a calcluator. How accurate was your guess?

\end{enumerate}
\end{question}




\begin{question}  \label{QLkfde9fdfs}
Let's suppose we didn't know about the Pythagorean theorem and try to answer the previous question emperically.

Suppose, for example, we have a string $11$ centimeters long and pin its ends $10$ cm apart. Then we lift the midpoint of the string upward to make the string taut. We'll see something like the figure below, where the ends of the string $AM^\prime B$ are pinned at points $A$ and $B$.

\begin{onlineOnly}
    \begin{center}
\desmos{bse86hs9ng}{900}{600}
\end{center}
\end{onlineOnly}

\href{https://www.desmos.com/calculator/bse86hs9ng}{141: Railroad Track}

The idea now is to use the data we can gather here to answer the previous question.  

\begin{enumerate}
\item First approximate the height of the midpoint $M^\prime$ of the string above the ground (ie. approximate the length of segment $MM^\prime$. Do this visually, without using the power of the computer, as if you had a physical piece of string and were measuring its height with a ruler.

\item Now let's thing about relative changes, in both the length of our railroad track (ie. the string) and its height.

\begin{enumerate}
\item If the relaxed track had length $10$cm and its length increased to $11$cm, the relative change in its length is $\answer{10}\%$.

\item And in this case, the height of the track, about $2.3$ cm at its midpoint, as a percentage of the original track length ($10$cm) is approximately
\[
  \frac{\answer{2.2}\text{ cm}}{10\text{ cm}} = \answer{22}\%.    
\]
Input this fraction (as a decimal) in the second column ($y_1$) of the table in Line 1 above. 

\item Now let's gather more data by changing the relative change
\[
        k = \frac{\text{change in the length of the stretched string}}{\text{distance between its pinned ends}},
\] 
in the length of the string. Do this by dragging point $P$ above. Choose $k=0.2, 0.3, 0.4, 0.5$ to complete the table in Line 1.

\item Use this data to approximate the height of the track in Question 1, where the track had a relaxed length of $5$ km and the stretched track a length one meter greater. For this, you'll need to look at the plot of the data points in Table 1 and make a reasonble guess about the kind of curve on which they lie.
\end{enumerate}
\end{enumerate}
\end{question}


\begin{question} \label{QPlFErerr33e}
This is a generalization of Question 5.

A railroad track $L$-km long railroad track secured only at its two ends to the desert floor. Suppose on a hot day the track expands $\Delta L$ km and buckles into a curve.To approximate the track's height at its midpoint, we assume the track bends into two straight segments running from the high point to its endpoints.

\begin{enumerate}
\item Find a expression for the exact height.

\item Assume $\Delta L$ is small relative to $L$ and find an expression for the approximate height $h$.

\item Find an expression that approximates the relative height $h/L$ in terms of the relative change $\Delta L/L$.

\item When a railroad track expands by $50$cm, its buckles to a height of $20$ meters. How high would the buckle have been if

\begin{enumerate}
\item the same track expanded by $100$cm instead?

\item a track of the same material twice as long expanded by $50$ cm? By $100$ cm?
\end{enumerate}
\end{enumerate}

\end{question}



\section{A Leaning Ladder}

\begin{question}  \label{Qdfdst4nb554334}
(a) The bottom end of a $20$-foot ladder rests on level ground, $15$ feet away from a vertical wall. The top end rests against the wall. How high above the ground is the top end?

(b) One end of an $L$-foot long ladder rests on the ground, the other against a vertical wall. Find a function 
\[
  h = f(s) \, , \, 0\leq s \leq L,
\]
that expresses the height (in feet) of the top end above the ground in terms of the distance (in feet) from the bottom end to the wall.

(c)  Graph the function $h = f(s)$ with $L=20$.

(d) Find an equation of the ladder in the coordinate system below. Your equation should express $h$ in terms of $s$ and $L$ (do not assume $L=20$). It should also include the correct domain. Then move the slider $s$ to see if you are correct.

\begin{exploration}

\begin{onlineOnly}
    \begin{center}
\desmos{orpmncb1dx}{900}{600}
\end{center}
\end{onlineOnly}

\href{https://www.desmos.com/calculator/orpmncb1dx}{141: Ladder}

\end{exploration}
\end{question}


\begin{question}  \label{Qdfsf4tnbnt}
One end of a $10$-foot long rope is attached to the top of a $12$-foot tall pole. The other end of the rope is attached to a hat.

A gremlin, at most $12$ tall but perhaps very short, puts the hat on its head and walks away from the pole until the rope is taut.

(a) Find a function
\[
      h = f(s)
\]
that expresses the gremlin's height (in feet) in terms of its distance from the pole (in feet).

(b) Find the domain and range of the function $f$.

(c) Graph the function $f$. 

(d) Find a function
\[
     s = g(h)
\]
that expresses the gremlin's distance from the pole (in feet)  in terms of its height (in feet).

(e) Find the domain and range of $g$.

(f) Graph the function $g$.

(g) How would your functions in parts (a) and (d) change (if at all), if instead the gremlin is at least 12 feet tall? 

\end{question}


\begin{question}  \label{Qdfthhrhghgr}
On a clear day with an unobstructed view (like you might have at the beach or in a hot air balloon), the distance to the horizon is limited by the curvature of the earth as illustrated in the demonstration below.

\pdfOnly{
Access Desmos interactives through the online version of this text at
 
\href{https://www.desmos.com/calculator/ewowig5sgk}.
}
 
\begin{onlineOnly}
    \begin{center}
\desmos{ewowig5sgk}{900}{600}
\end{center}
\end{onlineOnly}

Desmos activity available at

\href{https://www.desmos.com/calculator/ewowig5sgk}{151:Distance to Horizon 1}

\pskip

The actual function 
\[
        s= f(h) \, , \, 0\leq h \leq 100,000 ,
\]
expresses the approximate distance to the horizon (the length of the red arc $AT$ above, measured in miles) in terms of your height above the ground (the distance $AP$ above, measured in feet). But for this we need trigonometry, so instead we will takdee the distance to the horizon $s$ to be the length of segment $\overline{AP}$.

We need one key fact - that the line of sight $\overline{PT}$ from the balloon to the horizon is tangent the earth at $P$. This makes $\angle PTO$ a right angle. That and the radius of the earth ($3960$ miles) is all you need to know to find an expression for the function $f$. Keep in mind that the input to $f$ is measured and feet and the output is measured in miles. Remember there are $5280$ feet in one mile.

The function
\[
        s= f(h)  = \answer{\sqrt{\left( 3960 + \frac{h}{5280}\right)^2 - 3960^2}}   \, , \, 0\leq h \leq 10,000 ,
\]
expresses the approximate distance to the horizon (the length of segment $\overline{PT}$ above, measured in miles) in terms of your height above the ground (the distance $AP$, measured in feet).


Explain why the function
\[
      s = g(h) =  1.22\sqrt{h}, 0\leq h \leq 100,000 ,
\]
gives a good approximation to $f$. Then use desmos to check this.


\pskip \pskip




\end{question}

\end{document}




