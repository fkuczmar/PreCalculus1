\documentclass{ximera}
\title{Exponential Growth}

\newcommand{\pskip}{\vskip 0.1 in}

\begin{document}
\begin{abstract}
Exponential functions and their inverses.
\end{abstract}
\maketitle

A population that grows \emph{linearly} increases at a constant absolute rate, like at the rate of $20,000$ people/yr. 

A population that grows \emph{exponentially} increases at a constant \emph{relative} rate, like at the rate of $20\%$/yr. To understand what this means, suppose the population was $P$ at the start of the year 2020. Then one year later the population would be $20\%$ greater, or 
\[
   P + 20\% P  = P + 0.2 P = 1.2P .
\]
We call 1.2 the \emph{annual growth factor}. To get the population one year from, we just multiply the current population by this factor of 1.2

A population might decrease exponentially, say by $20\%$/yr. 

\begin{question}  \label{Q1:ExponentialG}
(a) Suppose a population decreases by $20\%$ each year and that the current population is $P$. What would the population by one year from now?  
\[
\answer{0.8P}
\]

(b) What is the annual growth factor?
\[
   \answer{0.8}
\]

\end{question}

\begin{question}  \label{Q2:ExponentialG}
(a) Suppose a population increases by $20\%$ over a one-year period and then decreases by $20\%$ over the next year. What is the relative change in the population over the two-year period?
\[
   \text{The relative change in the population over the two-year period is }   \answer{-4\%} .
\]

\end{question}

\begin{question}  \label{Q3:ExponentialG}
A population increases by $40\%$ each year. Find the relative change in the population over a two-year period. 
\begin{multipleChoice}  
\choice{$80\%$}  
\choice[correct]{$96\%$}  
\choice{$20\%$}  
\end{multipleChoice}  
\end{question}



\begin{example}
not yet
\end{example}


\end{document}