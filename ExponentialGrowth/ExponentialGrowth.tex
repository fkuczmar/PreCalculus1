\documentclass{ximera}
\title{Exponential Growth}

\newcommand{\pskip}{\vskip 0.1 in}

\begin{document}
\begin{abstract}
Exponential functions and their inverses.
\end{abstract}
\maketitle

A population that grows \emph{linearly} increases at a constant absolute rate, perhaps at the rate of $20,000$ people/yr. 

A population that grows \emph{exponentially} increases at a constant \emph{relative} rate, perhaps at the rate of $20\%$/yr. To understand what this means, suppose the population was $P$ at the start of the year 2020. Then one year later the population would be $20\%$ greater, or 
\[
   P + 20\% P  = P + 0.2 P = 1.2P .
\]
We call 1.2 the \emph{annual growth factor}. To get the population one year from, we just multiply the current population by this factor of 1.2

\begin{question}  \label{Q00:ExponentialG}
A population grows exponentially, increasing by $25\%$ each year. Suppose the current population is $P$.

(a) What will the population be one year from now?
\[
\answer{1.25P}
\]
 
(b) What was the population one year ago?
\[
\answer{0.8P}
\]

(c) What will the population be three years from now?
\[
\answer{1.953125P}
\]

(d) What is the three-year growth factor?
\[
\answer{1.953125}
\]

(e) What is the relative change in the population over a three-year period?
\[
\text{The population increases by } \answer{95.3125}\% \text{ every three years.}
\]

\end{question}


\begin{question}  \label{Q1:ExponentialG}
Suppose a population decreases exponentially, decreasing by $20\%$ each year and that the current population is $P$. 

(a) What will the population be one year from now?  
\[
\answer{0.8P}
\]

(b) What is the annual growth factor?
\[
   \answer{0.8}
\]

(c) What was the population one year ago?
\[
\answer{1.25P}
\]

(d) What will the population be two years from now?
\[
\answer{0.64P}
\]

(e) What is the two-year growth factor?
\[
\answer{0.64}
\]

(f) What is the relative change in the population over a two-year period?
\[
\text{The population decreases by } \answer{36}\% \text{ every two years.}
\]

\end{question}

\begin{question}  \label{Q2:ExponentialG}
(a) Suppose a population increases by $20\%$ over a one-year period and then decreases by $20\%$ over the next year. What is the relative change in the population over the two-year period?
\[
   \text{The relative change in the population over the two-year period is }   \answer{-4\%} .
\]

\end{question}

\begin{question}  \label{Q3:ExponentialG}
A population increases by $40\%$ each year. Find the relative change in the population over a two-year period. 
\begin{multipleChoice}  
\choice{$80\%$}  
\choice[correct]{$96\%$}  
\choice{$20\%$}  
\end{multipleChoice}  
\end{question}


\begin{question}   \label{Q4:ExponentialG}
The population of a colony of bacteria grows exponentially between 12pm and 6pm. The population is $400,000$ at 12:00pm and $440,000$ at 1:00pm. 

(a) What is the one-hour growth factor? Explain what this means.

(b) What happens to the population every hour?

(c) What is the three-hour growth factor? What happens to the population every three hours?

(d) Find a function 
\[
     P = f(t) \, , 0\leq t \leq 6 
\]
that expresses the population in terms of the number of hours past noon.

\end{question}


\begin{question}   \label{Q5:ExponentialG}
The population of a colony of bacteria grows exponentially between 12pm and 6pm. The population is $400,000$ at 12:00pm and $520,000$ at 3:00pm. 

(a) What is the three-hour growth factor?

(b) What happens to the population every three hours?

(c) What is the one-hour growth factor?

(d) Find a function 
\[
     P = f(t) \, , 0\leq t \leq 6 
\]
that expresses the population in terms of the number of hours past noon.

\end{question}

\begin{question}   \label{Q6:ExponentialG}
The world population was approximately 1.5 billion in 1900 and 2.5 billion in 1950. Assume that between 1900 and 2023 the population grew exponentially.

(a) What is the one-year growth factor? Give an exact value and then an approximate value to 4 decimal places.
\[
   \text{Exact growth factor }  \answer{(5/3)^{1/50}} \hskip 0.3 in  \text{Approximate growth factor } \answer{1.0103}
\]

(b) Use the exact growth factor to find a function
\[
     P = f(t) \, , 0\leq t \leq 123 ,
\]
that expresses the population (measured in billions of people) in terms of the number of years past 1900.

\[
   f(t) = \answer{1.5 \left(\frac{5}{3}\right)^{t/50}}
\]

(c) Use your function to predict the world population in 2023. How does this compare with the actual population?

\end{question}


\begin{question}   \label{Q88:ExponentialG}
Suppose that between sea level and an altitude of 10 km, the air pressure is an exponential function of altitude. Suppose further that the pressure is 14.7 pounds/(square inch) at sea level and 10.17 pounds/(square inch) at an altitude of 3 km.

(a) What happens to the air pressure when the altitude increases by 3 km?

(b) What happens to the air pressure when the altitude increases by 1 km?

(c) Find a function
\[
    P = f(h) \, , 0\leq h \leq 10,
\]
that expresses the pressure (measured in lbs/(square inch) in terms of the altitude (measured in kilometers). Enter the function below.
\[
      f(h) = \answer{14.7(10.17/14.7)^{h/3}} \, , 0\leq h \leq 10 .
\]

(d) Enter your function in Line 1 of the exploration below (ie. replace the incorrect function $f(h)=14.7-h$ with the correct one) and use the graph to estimate the altitude at which the pressure is 7 lbs/(square inch).


\pdfOnly{
Access Desmos interactives through the online version of this text at
 
\href{https://www.desmos.com/calculator/a8d1scpez0}.
}
 
\begin{onlineOnly}
    \begin{center}
\desmos{a8d1scpez0}{900}{600}
\end{center}
\end{onlineOnly}

\end{question}


\begin{question}   \label{Q9:ExponentialG}
Suppose both the population and national debt of a small country grow exponentially. The population doubles every 20 years and the national debt doubles every 12 years. How long does it take the per-capita (ie. per-person) share of the national debt to double?
\[
    \answer{30} \text{ years}
\]
\end{question}


\begin{question}   \label{Q10:ExponentialG}
Returning to the scenario of Question 7, 

(a) find a function
\[
    P = g(T) \, , 0\leq T \leq 373 ,
\]
that expresses the population in terms of the number of years since 1650. 

{\it Hint:} First express $t$ (the number of years since 1900) in terms of $T$.

(b) Use your function from part (a) to find a function
\[
  A = g_1(T) \, , 0\leq T \leq 373 ,
\]
that expresses the land area (measured in billions of hectares) needed to feed the world population as a function of the number of years since 1650 (see the caption to Figure 10 below and assume the present (1970) productivity level).

(c) Use your function from part (a) to find functions
\[
  A = g_2(T)  \text{ and } A = g_3(T)\, , 0\leq T \leq 373 ,
\]
that express the land area (measured in billions of hectares) needed to feed the world population as a function of the number of years since 1650 at levels of productivity double and quadruple the 1970 level).

(d) Find a function
\[
    A = h(T) \, , 250\leq T \leq 373 ,
\]
that expresses the area (in billions of hectares) available for agriculture as a function of the number of years since 1650. Assume there were 3.2 billion acres available in 1900 and that each additional person added to the population after 1900 required 0.08 hectares of arable land for infrastructure that rendered that land unavailable for agriculture.

\pdfOnly{
Access Desmos interactives through the online version of this text at
 
\href{https://www.desmos.com/calculator/ovjygr01yy}.
}
 
\begin{onlineOnly}
    \begin{center}
\desmos{ovjygr01yy}{900}{600}
\end{center}
\end{onlineOnly}


\end{question}









\begin{example} \label{E1:ExpGrowth}
Suppose the population of a colony of bacteria changes exponentially between noon and 6pm. For each of the following scenarios, do the following:

i) Describe the growth factor over the appropriate period.

ii) Describe the relative change in the population over that period.

iii) Find a function
\[
       P = f(t) \, , 0\leq t \leq 6 ,
\]
that expresses the population (measured in hundreds of thousands of bacteria) in terms of the number of hours past noon.

iv) Find the population at 6pm.

\pskip

(a) The population is $400,000$ at noon and $450,000$ at 1pm.

(b) The population is $400,000$ at noon and $300,000$ at 1pm.

(c) The population is $400,000$ at noon and $500,000$ at 2pm.

(d) The population is $400,000$ at noon and $500,000$ at 2:30pm.

(e)  The population is $400,000$ at noon and $300,000$ at 2:20pm.

(f)  The population is $400,000$ at 1:00pm and $500,000$ at 3:00pm.

\end{example}


\end{document}