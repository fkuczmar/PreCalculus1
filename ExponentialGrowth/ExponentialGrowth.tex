\documentclass{ximera}
\title{Exponential Growth}

\newcommand{\pskip}{\vskip 0.1 in}

\begin{document}
\begin{abstract}
Exponential functions and their inverses.
\end{abstract}
\maketitle

A population that grows \emph{linearly} increases at a constant absolute rate, perhaps at the rate of $20,000$ people/yr. 

A population that grows \emph{exponentially} increases at a constant \emph{relative} rate, perhaps at the rate of $20\%$/yr. To understand what this means, suppose the population was $P$ at the start of the year 2020. Then one year later the population would be $20\%$ greater, or 
\[
   P + 20\% P  = P + 0.2 P = 1.2P .
\]
We call 1.2 the \emph{annual growth factor}. To get the population one year from, we just multiply the current population by this factor of 1.2

\begin{question}  \label{Q00:ExponentialG}
A population grows exponentially, increasing by $25\%$ each year. Suppose the current population is $P$.

(a) What will the population be one year from now?
\[
\answer{1.25P}
\]
 
(b) What was the population one year ago?
\[
\answer{0.8P}
\]

(c) What will the population be three years from now?
\[
\answer{1.953125P}
\]

(d) What is the three-year growth factor?
\[
\answer{1.953125}
\]

(e) What is the relative change in the population over a three-year period?
\[
\text{The population increases by } \answer{95.3125\%} \text{ every three years.}
\]

\end{question}


\begin{question}  \label{Q1:ExponentialG}
Suppose a population decreases exponentially, decreasing by $20\%$ each year and that the current population is $P$. 

(a) What will the population be one year from now?  
\[
\answer{0.8P}
\]

(b) What is the annual growth factor?
\[
   \answer{0.8}
\]

(c) What was the population one year ago?
\[
\answer{1.25P}
\]

(d) What will the population be two years from now?
\[
\answer{0.64P}
\]

(e) What is the two-year growth factor?
\[
\answer{0.64}
\]

(f) What is the relative change in the population over a two-year period?
\[
\text{The population decreases by } \answer{36\%} \text{ every two years.}
\]

\end{question}

\begin{question}  \label{Q2:ExponentialG}
(a) Suppose a population increases by $20\%$ over a one-year period and then decreases by $20\%$ over the next year. What is the relative change in the population over the two-year period?
\[
   \text{The relative change in the population over the two-year period is }   \answer{-4\%} .
\]

\end{question}

\begin{question}  \label{Q3:ExponentialG}
A population increases by $40\%$ each year. Find the relative change in the population over a two-year period. 
\begin{multipleChoice}  
\choice{$80\%$}  
\choice[correct]{$96\%$}  
\choice{$20\%$}  
\end{multipleChoice}  
\end{question}



\begin{example}
Suppose the population of a colony of bacteria changes exponentially between noon and 6pm. For each of the following scenarios, do the following:

i) Describe the growth factor over the appropriate period.

ii) Describe the relative change in the population over that period.

iii) Find a function
\[
       P = f(t) \, , 0\leq t \leq 6 ,
\]
that expresses the population (measured in hundreds of thousands of bacteria) in terms of the number of hours past noon.

iv) Find the population at 6pm.

\pskip

(a) The population is $400,000$ at noon and $450,000$ at 1pm.

(b) The population is $400,000$ at noon and $300,000$ at 1pm.

(c) The population is $400,000$ at noon and $500,000$ at 2pm.

(d) The population is $400,000$ at noon and $500,000$ at 2:30pm.

(e)  The population is $400,000$ at noon and $300,000$ at 2:20pm.

(f)  The population is $400,000$ at 1:00pm and $500,000$ at 3:00pm.

\end{example}


\end{document}