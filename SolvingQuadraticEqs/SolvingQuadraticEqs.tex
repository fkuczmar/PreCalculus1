\documentclass{ximera}
\title{Solving Quadratic Equations}


\newcommand{\pskip}{\vskip 0.1 in}

\begin{document}
\begin{abstract}
Solving quadratic equations and completing the square.
\end{abstract}
\maketitle




\section{A Common Error}

\begin{enumerate}
\item A typical quadratic equation has either two real solutions or no real solutions. It very rarely has exactly one real solution.

Some examples:

\begin{enumerate}

\item A common error is to think
\[
      (x + 5)^2 = x^2 + 25.
\]
This is incorrect. For example,
\[
    (2 + 5)^2 = 7^2 = 49
\]
but 
\[
 2^2 + 5^2 = 4 + 25 = 29.
\]

So
\[
   (2+5)^2 \neq 2^2 + 5^2 .
\]

To multlply out $(x+5)^2$ we need to use the distributive property multiple times.
\begin{align*}
(x+5)^2 &= (x+5)(x+5) \\
             &= x(x+5) + 5(x+5) \\
             &= x^2 + 5x + 5x + 25 \\
             &= x^2 +10x + 25.
\end{align*}

There is a picture that goes with the identity
\begin{align*}
(a+b)^2 &= (a+b)(a+b) \\
             &= a(a+b) + b(a+b) \\
             &= a^2 + ab + ab + b^2 \\
             &= a^2 +2ab + b^2.
\end{align*}

The key idea is that we can interpret $a^2$ as the area of a square with side length $a$, at least when $a>0$. And also $ab$ as the area of an $a\times b$ rectangle, at least when $a,b>0$.

Mistakenly supposing $(a+b)^2 = a^2 + b^2$ amounts to forgetting the two rectangles in the picture below.

\begin{onlineOnly}
    \begin{center}
\desmos{j1ghh5lcse}{900}{600}
\end{center}
\end{onlineOnly}

\href{https://www.desmos.com/calculator/j1ghh5lcse}{141: Completing the Square}

The idea is that the area of the big square (ie. the whole picture) is $(a+b)^2$ and that this area is the sum of four smaller areas. 



\section{Another Common Error}

\item Another common error is to think the equation 
\[
    x^2 = 9
\]
has only one solution, $x=3$. But in fact it has two real solutions because there are two real numbers you can square to get $9$, namely $3$ \emph{OR} $-3$. 

So the solutions to the equation
\[
    x^2=9
\]
are
\[
   x = 3 \, \text{ or } \, x=-3.
\]
The solution set to the equation
\[
      x^2 = 9
\]
is 
\[
  \{ x | x = 3 \text{ or } x=-3 \} .
\]

Read this as the set of all values of $x$ such that $x=3$ or $x=-3$.

\item The equation 
\[
        x^2 = -9
\]
has no \emph{real} solutions since squaring a real number never gives a negative number.

\item Find an example of a quadratic equation with exactly one real solution.
\begin{freeResponse}
\end{freeResponse}

\end{enumerate}
\end{enumerate}


\section{Be Lazy Whenever Possible}

Do not work harder than necessary. 

To solve the equation
\[
      13 - (x-5)^2 = -10,
\]
for example, it is \emph{not} necessary to distribute $(x-5)^2$. Instead, subtract $13$ from both sides to get
\[
    - (x-5)^2 = -23.
\] 
Then multiply both sides by $-1$ giving
\[
   (x-5)^2 = 23.
\]
Now there are two numbers you can square to get $23$, either $\sqrt{23}$ or $-\sqrt{23}$. So
\[
   x-5 = \pm \sqrt{23}
\]
and 
\[
   x = 5 \pm \sqrt{23}.
\]

Our conclusion is the solution set to the equation
\[
      13 - (x-5)^2 = -10,
\]
is
\[
   \{ x | x = \pm \sqrt{23} \} .
\]


\section{Completing the Square}

An important part of this class is developing you algebra skills and so we will almost always avoid relying on formulas. For solving equations this generally means not using the quadratic formula. Instead, we'll complete the square.

Here's an example.

\begin{example}  \label{Ex:er433r3rfsdstr}
Solve the equation
\[
      4x^2 - (x-5)^2 = 1 .
\]

\begin{explanation}
Unlike the last example, we can't be lazy here (why not?) and we need to distribute. Then
\[
       4x^2 - (x^2 - 10x +25) = 1 . 
\]
So
\[
    4x^2 - x^2 + 10x - 25 = 1 
\]
and
\[
      3x^2 + 10x - 24 = 0.
\]

To solve this equation by completing the square, we add $24$ to both sides giving
\[
    3x^2 + 10x = 24.
\]
Now we would like to interpret the left side as an area. For this, it would be much easier if the coefficient of $x^2$ were a perfect square. We could divide both sides by $3$, but to avoid fractions we'll \emph{multiply} both sides by $3$. Then
\[
  3 (3x^2 +10x) = 3(24)
\]
and
\[
       9x^2 + 30x = 72.
\]
And we'll write this last equation as
\[
     9x^2 + 15x + 15x = 72.
\]

The reason for doing this is so we can intepret the left side
\[
  9x^2 + 15x + 15x
\]
as the sum of three areas, a $3x \times 3x$ square with area $9x^2$ and two rectangles, each with area $15x$. 

\begin{onlineOnly}
    \begin{center}
\desmos{7tfjwvehju}{900}{600}
\end{center}
\end{onlineOnly}

\href{https://www.desmos.com/calculator/7tfjwvehju}{141: Completing the Square 2}

Our problem now is to complete the square by finding the area of the missing square and adding that area to both sides of our equation
\[
   9x^2 + 15x + 15x = 72 .
\]

The key idea is to look at the two rectangles, each with area $15x$. We know the length of each rectangle is equal to the side length $3x$ of the big red square. So the length of width of each rectangle must be
\[
    \frac{\answer{15x}}{\answer{3x}} = 5 .
\]

So we now have the picture below.

\begin{onlineOnly}
    \begin{center}
\desmos{oglxjqw5iw}{900}{600}
\end{center}
\end{onlineOnly}

\href{https://www.desmos.com/calculator/oglxjqw5iw}{141: Completing the Square 2B}

This tells us that we should add the additional area $5^2$ to both sides of our last equation
\[
      9x^2 + 15x + 15x = 72
\]
to get
\[
   9x^2 + 15x + 15x + 5^2 = 72 + 5^2 .
\]
Now the left side is a perfect square. From the last picture we can see the big square (ie. the whole picture) has side length $3x+5$. So our equation becomes
\[
   (3x + 5)^2 = 97 .
\]
Then 
\[
   3x + 5 = \pm \sqrt{97}
\]
and
\[
     x = \frac{1}{3}\left( -5 \pm \sqrt{97}  \right) .
\]

Our conclusion:

The solution set to the equation
\[
      4x^2 - (x-5)^2 = 1 
\]
is
\[
 \left\{ x | x =  \frac{1}{3}\left( -5 \pm \sqrt{97}  \right) \right\} .
\]
\end{explanation}

\end{example}


\end{document}