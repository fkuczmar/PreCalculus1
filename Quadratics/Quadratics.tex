\documentclass{ximera}
\title{Quadratic Functions, More Inverses}

\newcommand{\pskip}{\vskip 0.1 in}

\begin{document}
\begin{abstract}
Quadratic functions and more about inverse functions.
\end{abstract}
\maketitle

Remember the key idea about inverse functions.

\pskip

\emph{Inverse functions undo each other.}

\pskip

To show two functions are inverses, it is \emph{not} enough to show that \emph{one} undoes the other. We need to show that \emph{each} undoes the other. 

So to show two functions are $y=f(x)$ and $x=g(y)$ are inverses of each other, we need to show:

\begin{enumerate}
\item The function $g$ undoes the function $f$. This amounts to proving that
\[
        g(f(x)) = x 
\]
for \emph{all} $x$ in the domain of $f$.

\item And we also need to show that $f$ undoes $g$. This amounts to proving that
\[
        f(g(y)) = y
\]
for \emph{all} $y$ in the domain of $g$.
\end{enumerate} 

Simplifying the two compositions above requires doing some algebra. But on the other hand, showing that $f$ and $g$ are \emph{not} inverses of each other is \emph{not} an algebra problem. It requires only that we find a specific input to the function $f$, call it $x_0$, such that 
\[
    g(f(x_0)) \neq x_0 , 
\]
or that we find a specific input to $g$ (say $y_0$) such that
\[
    f(g(y_0)) \neq y_0.
\]

Here are some examples.

\begin{question}  \label{Q345rgt5t45hh}
Let 
\[
    f(x) = \sqrt{x}, x \geq 0
\]
and 
\[
  g(x) =  x^2 , x\in\mathbb{R} .
\]
\begin{enumerate}
\item Does $g$ undo $f$? If so, prove this algebraically. If not, prove this by finding a specific input $x_0$ to $f$ such that $g(f(x_0))\neq x_0$.

\item Does $f$ undo $g$? If so, prove this algebraically. If not, prove this by finding a specific input $y_0$ to $g$ such that $f(g(y_0))\neq y_0$.

\item Are $f$ and $g$ inverses of each other?
\end{enumerate}
\end{question}



\begin{question}    \label{Q1:Quadratics}
Let 
\[
    f(x) = x^2 \, , x \in \mathbb{R} ,
\]
and 
\[
     g(x) = \sqrt{x} \, , x\geq 0 .
\]

(a) Explain why the inverse of the function $f$ is not a function.

(b) Let 
\[
    h(x) = g(f(x)) .
\]

(i) Find the domain of $h$.

(ii) Simplify $h(x)$.

(iii) Graph the function $y=h(x)$.

\pskip

(c) Answer parts (i)-(iii) above for the function
\[
   j(x) = f(g(x)) .
\]

(d) Let
\[
   k(x) = x^2 \, , x\leq 0 .
\]

(i) Is the inverse of the function $k$ a function? Explain.

(ii) Find an expression for $k^{-1}(y)$ by expressing the input to the function $y=k(x)$ in terms of its output.

(iii) Find the domain of $k^{-1}(y)$.

(iv) Simplify the composition $k^{-1}(k(x))$. Graph the composition.

\end{question}


\begin{example}     \label{Ex99:Quadratics}
Let 
\[
   f(x) = \sqrt{25-x^2} \, , -5 \leq x \leq 5 .
\]

(a) Use your knowledge of circles to sketch a graph of the function $y=f(x)$ by hand.

(b) Explain what it means for a funtion to be one-to-one.

(c) Is the function $f$ one-to-one? Explain your reasoning.

(d) Explain what it means for a relation to be a function.

(e) Is the inverse of $f$ a function? Explain your reasoning.

\begin{explanation}

(a) Since 
\[
    y = \sqrt{25-x^2},
\]
we know that $y \geq 0$. Then squaring both sides and doing some algebra (you should fill in the details) gives
\[
    x^2 + y^2 =25 \text{ and } y\geq 0.
\]
The curve 
\[
   x^2 + y^2 = 25
\]
is a circle of radius 5 centered at the origin. But since $y\geq 0$, the graph (not shown) of the function $y=f(x)$ is a semicircle in the first and second quadrants with endpoints $(5,0)$ and $(-5,0)$.

(b) A function is one-to-one if any two distinct inputs have distinct outputs.

(c) The function $f$ is {\bf not} one-to-one. We can see this because the two inputs $x=-5$ and $x=5$ give the same output
\[
     f(5) = \sqrt{25-5^2} = 0 
\]
and
\[
     f(-5) = \sqrt{25-(-5)^2} = 0 = f(5). 
\]

We can also see that $f$ is not one-to-one from the graph. There are many horizontal lines that intersect the graph in two points. This means that there are many pairs of inputs that give the same output.

(d) A function is a relation when each input has exactly one output.

(e) Because $f$ is not one-to-one its inverse is {\bf not} a function. For example, the input $y=0$ to the inverse relation would have two outputs, $x=-5$ or $x=5$, making the inverse relation not a function.

\end{explanation}

\end{example}


\begin{example} \label{Ex98:Quadratics}
Let 
\[
    y=   f(x) = \sqrt{25-x^2} \, , 0\leq x \leq 5 .
\]
\begin{enumerate}
\item Use your knowledge of circles to graph the function $y=f(x)$ \emph{by hand}. Use the graph to thoroughly explain why $f$ is one-to-one.

\item Find an expression for the inverse function
\[
    x = f^{-1}(y)
\]
that expresses the input of $f$ in terms of its output. State the domain of $f^{-1}$ using set notation.

\item Simplify the composition $f^{-1}(f(x))$ algebraically.

\item Graph the function $y=f^{-1}(x)$. 

\item Explain how you could have realized that $f^{-1}(x) = f(x)$ by looking at the graph of $y=f(x)$.
\end{enumerate}


\begin{explanation}
\begin{enumerate}
\item The graph of $y=f(x)$ is a quarter-circle in the first quadrant centered at the origin with endpoints $(5,0)$ and $(0,5)$. Since each horizontal line intersects the graph in at most one point, any two different inputs to $f$ have different ouputs. This means that $f$ is one-to-one.

\item We know from the graph in part (a) that the range of $f$ is the set
\[
   \{ y \, | \,  0\leq y \leq 5 \} .
\]
This is the domain of the function $x=f^{-1}(y)$.

Now to find an expression for $f^{-1}(y)$, we solve the equation
\[
   y=   \sqrt{25-x^2} \, , 0\leq x \leq 5 
\]
for $x$. Squaring both sides gives
\[
   y^2 = 25 - x^2 .
\]
So 
\[
     x^2 = 25-y^2
\]
and 
\[
   x  = \pm \sqrt{25-y^2} .
\]

To decide which sign to choose, we go back to the domain of $f$, where we are given that 
\[
     0 \leq x \leq 5 .
\]
Since $x\geq 0$, this means we must choose the positive sign and so
\[
    x = f^{-1}(y) = \sqrt{25-y^2}.
\]
We should also include the domain $0\leq y \leq 5$ (as stated above). So the inverse of the function
\[
    y=   f(x) = \sqrt{25-x^2} \, , 0\leq x \leq 5 
\]
is the function
\[
    x = f^{-1}(y) = \sqrt{25-y^2} \, , 0\leq y \leq 5 .
\]

\item We simplify the composition $f^{-1}(f(x))$ as follows:
\begin{align*}
           f^{-1}(f(x)) &= f^{-1}(\sqrt{25-x^2})   \\
                             &= \sqrt{25 - \left( \sqrt{25-x^2}\right)^2}  \\
                             &= \sqrt{25 - (25-x^2)}  \\
                             &= \sqrt{x^2} \\
                             &= |x|  \\
                             &= x \text{ since } 0\leq x \leq 5.
\end{align*}

\item Since the variables $x$ and $y$ have no practical meaning, we can switch them in our expression for $x = f^{-1}(y)$. So the inverse of the function
\[
    y=   f(x) = \sqrt{25-x^2} \, , 0\leq x \leq 5 
\]
is the function
\[
    y = f^{-1}(x) = \sqrt{25-x^2} \, , 0\leq x \leq 5 .
\]
This is identical to the function $f$. That is, $f$ is its own inverse and the graphs of $f$ and $f^{-1}$ are identical.

\item Since the graph of $y=f(x)$ is symmetric about the line $y=x$, we could have realized from the start that $f$ is its own inverse.

\end{enumerate}

\end{explanation}

\end{example}

\begin{example}  \label{Ex97:Quadratics}
Answer parts (a)-(d) of Example \ref{Ex98:Quadratics} for the function
\[
    y=   f(x) = \sqrt{25-x^2} \, , -5 \leq x \leq 0 .
\]

\end{example}

\begin{example}  \label{Ex96:Quadratics}
Answer parts (a)-(e) of Example \ref{Ex98:Quadratics} for the function
\[
    y=   f(x) = - \sqrt{25-x^2} \, , -5 \leq x \leq 0 .
\]

\end{example}



\begin{exploration} \label{E1:Quadratics}
The animation below shows water draining from a tank. 

\begin{onlineOnly}
    \begin{center}
\desmos{pdghky6tie}{900}{600}
\end{center}
\end{onlineOnly}

\href{https://www.desmos.com/calculator/pdghky6tie}{141: Draining Tank 1}

\begin{enumerate}
\item Use the animation to sketch by hand a graph of the function $V=f(t)$ that expresses the depth of the water as a function of time. Explain your reasoning.

\item Is the function $f$ one-to-one? Explain your reasoning.

\item Is the inverse of $f$ a function? Explain your reasoning.

\end{enumerate}
\end{exploration}


\begin{example} \label{Ex1:Quadratics}

\begin{enumerate}
\item The function $V=f(t)$, $0\leq t \leq 6$, expresses the volume of water in a tank, measured in gallons, in terms of the number of minutes past noon. 

\begin{enumerate}
\item Suppose the function $f$ is one-to-one. Explain what this would mean in everday English as it applies in this particular situation. Do not use any math terms (like input or output).

\item Suppose the function $f$ is \emph{not} one-to-one. Explain what this would mean in everday English as it applies in this particular situation. %Do not use any math terms (like input or output).

\end{enumerate}

\item Now suppose 
\[
     V = f(t) = \frac{1}{3}(t-6)^2 \, , 0\leq t \leq 6 .
\]

\begin{onlineOnly}
    \begin{center}
\desmos{ef7obzoxx3}{900}{600}
\end{center}
\end{onlineOnly}

\href{https://www.desmos.com/calculator/ef7obzoxx3}{141: Tank 5}

\begin{enumerate}
\item Use the graph of $V=f(t)$ above to determine if the inverse of the function $f$ also a function. Explain your reasoning. Explain also what this means in the context of this particular scenario.

\item Desribe the function $f^{-1}$ in this scenario. What does it take an input? What does it return as an output?

\item Find an expression for the function $t = f^{-1}(V)$. Write its domain using set notation.
\[
      \{ \answer{V} \, | \, \answer{0} \leq \answer{V} \leq \answer{12} \}
\]

\item Simplify the composition $f^{-1}\circ f$ algebraically. Use the approprite variable name for the input.

\item  Simplify the composition $f\circ f^{-1}$ algebraically. Use the approprite variable name for the input.,

\end{enumerate}

\end{enumerate}

\end{example}


\begin{exploration} \label{E2:Quadratics}
The animation below shows the motion of a balloon. 

 
\begin{onlineOnly}
    \begin{center}
\desmos{95v58n7mti}{900}{600}
\end{center}
\end{onlineOnly}

\href{https://www.desmos.com/calculator/95v58n7mti}{141: Balloon 1}

\begin{enumerate}

\item Use the animation to sketch by hand a graph of the function $h=f(t)$ that expresses the height of the balloon as a function of time. Explain your reasoning.

\item Is $f$ one-to-one? Explain your reasoning.

\item Is the inverse of $f$ a function? Explain your reasoning.

\end{enumerate}

\end{exploration}



\begin{exploration} \label{E4:Quadratics}
The function 
\[
\   h=f(t) \, , 0\leq t \leq 90, 
\]
expresses the height of the balloon (measured in feet) in terms of the number of minutes past noon. Its graph is shown below. 

\pdfOnly{
Access Desmos interactives through the online version of this text at
 
\href{https://www.desmos.com/calculator/wxeu32b4dp}.
}
 
\begin{onlineOnly}
    \begin{center}
\desmos{wxeu32b4dp}{900}{600}
\end{center}
\end{onlineOnly}


Let 
\[
    T = u(h)
\]
be the function that gives the time (measured in minutes past noon) when the balloon is $h$ feet above the ground and on its way up.

Let 
\[
    T = d(h)
\]
be a function that gives the time (measured in minutes past noon) when the balloon is $h$ feet above the ground and on its way down.

\begin{enumerate}

\item Find the domain and range of the function $u$. Use set notation with the appropriate variable names.

The domain of the functin $T=u(h)$ is
\[
    \{      \answer{h} \, | , \answer{5} \leq \answer{h} \leq \answer{45} \}.
\]
The range is the set
\[
   \{      \answer{T} \, | , \answer{0} \leq \answer{T} \leq \answer{36} \} .
\]

\item Find the domain and range of the function $d$. Use set notation with the appropriate variable names.

\item Explain the meaning of each of the following compostions. Then evaluate each composition if possible.

\begin{enumerate}
\item $u(f(20))$    (Click the Hint tab above for the solution.)

\begin{hint}
The expression $u(f(20))$ is the time when the balloon is on its way up and has the same height as it does at 12:20pm. Because the balloon is on its way up at 12:20pm
\[
   u(f(20)) = 20.
\]
\end{hint}

\item $d(f(20))$

\item $u(f(5))$

\item $d(f(5))$

\item $f(u(80))$

\item $f(u(30))$

\item $f(d(30))$

\end{enumerate}

\item Explain the meaning of the composition $u\circ f$. Find the domain and range of the composition. Use set notation with the appropriate variable names.

\item Explain the meaning of the composition $d\circ f$. Find the domain and range of the composition. Use set notation with the appropriate variable names.

\item Explain the meaning of the composition $f\circ u$. Find the domain and range of the composition. Use set notation with the appropriate variable names.

\item Explain the meaning of the composition $f\circ d$. Find the domain and range of the composition. Use set notation with the appropriate variable names.

\item Use the graph of the function $h=f(t)$ above to sketch graphs of the following functions. Explain your reasoning.

\begin{enumerate}
\item $u\circ f$

\item $d\circ f$

\item $f\circ u$

\item $f\circ d$

\end{enumerate}

\end{enumerate}

\end{exploration}




\begin{example} \label{Ex7:Quadratics}
The function $y=g(x)$ has domain
\[
   \{  x \, | \, 4 \leq x \leq 6   \}.
\]
The graph of the function $f$ is shown below.

(a) Find the domain of the compoisition $g\circ f$. Write it as a set.

(b) What can you say about the domain of the compostion $f\circ g$?


\pdfOnly{
Access Desmos interactives through the online version of this text at
 
\href{https://www.desmos.com/calculator/fjbqii5wbo}.
}
 
\begin{onlineOnly}
    \begin{center}
\desmos{fjbqii5wbo}{900}{600}
\end{center}
\end{onlineOnly}


\end{example}



\begin{example} \label{Ex8:Quadratics}
The function 
\[
   h = f(t) = 320 - 5 (t-6)^2 \, , 0\leq t \leq 14 ,
\]
expresses the height of a rock (in meters) in terms of the number of seconds past noon.

\pdfOnly{
Access Desmos interactives through the online version of this text at
 
\href{https://www.desmos.com/calculator/ew9jsipj5z}.
}
 
\begin{onlineOnly}
    \begin{center}
\desmos{nho1omvdtd}{900}{600}
\end{center}
\end{onlineOnly}

(a) Find a function $T=u(h)$ that takes as an input a height (in meters) and returns as an output the time (in seconds past noon) when the rock was at that height and on its way up. Include the domain.

(b) Explain the meaning of the composition $u\circ f$.

(c) Find the domain of the composition $u\circ f$. Write it as a set with the appropriate variable.

(d) Use the graph of the function $f$ above to sketch a graph of the function $T= u(f(t))$. Explain your reasoning.

(e) Simplify the composition $T=u(f(t))$ alegebraically. Your final expression should involve the absolute value function. Then graph the composition and see how it compares with your sketch from part (d).  %Then find a piecewise-defined equivalent form of the composition.

(f) Find the inverse of the function $u$. Stop and think. You need not do any algebra for this.

(g) Repeat parts (a)-(f) for the function $T=d(h)$ that takes as an input a height (in meters) and returns as an output the time (in seconds past noon) when the rock was at that height and on its way down.

\end{example}


\section*{Exercises}

\begin{exercise}   \label{Ex:LKdjfgd}
For each of the following scenarios, do the following:

\begin{itemize}
\item{Describe what it means in that particular scenario for the given function to be one-to-one.}

\item{Decide whether the given function is likely to be one-to-one. Explain your reasoning.}
\end{itemize}

(a) The function 
\[
     T = f(h)  \, , 0\leq h \leq 24
\]
that expresses the temperature (in Fahrenheit degrees) on a summer day in Shoreline in terms of the number of hours past midnight.

(b) The function
\[
     Q = f(p) \, , 4\leq p \leq 12 ,
\]
that expresses the average number of burgers/day Five Guys in Edmonds sells as a function of the price (in $\$$/burger).

(c) The function
\[
     T = g(h) \, , 3000\leq h \leq 12000 ,
\]
that expresses the temperature (in Fahrenheit degrees) in terms of the altitude (in feet) on the west side of Mount Rainier.

(d) The function 
\[
   h = f(t) \, 0\leq t \leq 365
\]
that expresses the depth of water (measured in feet) in the Lincoln Park reservoir in terms of the number of days since the start of the year 2023. 

\end{exercise}

\begin{exercise}  \label{E:dsf4ttmb}
The function
\[
    h = f(t) \, , 1\leq t \leq 4
\]
expresses the altitude of a balllon (measured in thousands of feet) in terms of the number of hours past noon.

(a) What would it mean in this particular scenario for the function $f$ to be one-to-one?

\pskip

Suppose now that 
\[
    h = f(t) = 2 + \sqrt{25-(t-5)^2} \, , 1\leq t \leq 5.
\]

(b) Sketch a graph of the function $f$. Label the axes with the appropriate variable names and units.

(c) Use your graph to determine if the function $f$ is one-to-one. Explain.

(d) What is the meaning of the function $f^{-1}$? What does it take as an input? What does it return as an output?

(e) Find an expression for the function $f^{-1}$. Include the appropriate domain.

(f) Graph the function $f^{-1}$. Label the axes with the appropriate variable names and units.

(g) Use algebra to simplify the compositions $f^{-1}(f(t))$ and $f(f^{-1}(h))$. Show all steps.

(h) Graph the compositions $f^{-1}(f(t))$ and $f(f^{-1}(h))$. Label the axes with the appropriate variable names and units.

(i) Explain the meaning of the function
\[
      g(t)=f^{-1}(f(t)-1) .
\]
What does it take as an input? What does it return as an output?

(j) Find the domain of the function $g$. 

\end{exercise}


\begin{exercise} \label{Eds5567y7y6644}
\begin{enumerate}
\item The function $h=f(t)$, $0\leq t \leq 8$, expresses the height (in meters) of a rock launched straight up on the planet Krypton terms of the number of seconds past noon. 

\begin{enumerate}
\item Suppose the function $f$ is one-to-one. Explain what this would mean in everday English as it applies in this particular situation. Do not use any math terms (like input or output).

\item Suppose the function $f$ is \emph{not} one-to-one. Explain what this would mean in everday English as it applies in this particular situation. %Do not use any math terms (like input or output).

\end{enumerate}

\item Now suppose 
\[
     h = f(t) = 40 - \frac{3}{5}(t-8)^2 \, , 0\leq t \leq 8 .
\]

\begin{onlineOnly}
    \begin{center}
\desmos{medroshqo6}{900}{600}
\end{center}
\end{onlineOnly}

\href{https://www.desmos.com/calculator/medroshqo6}{141: Rock 5}

\begin{enumerate}
\item Use the graph of $h=f(t)$ above to determine if the inverse of the function $f$ also a function. Explain your reasoning. Explain also what this means in the context of this particular scenario.

\item Desribe the function $f^{-1}$ in this scenario. What does it take an input? What does it return as an output?

\item Find an expression for the function $t = f^{-1}(h)$. Write its domain using set notation.
\[
      \{ \answer{g} \, | \, \answer{1.6} \leq \answer{h} \leq \answer{40} \}
\]

\item Simplify the composition $f^{-1}\circ f$ algebraically. Use the approprite variable name for the input.

\item  Simplify the composition $f\circ f^{-1}$ algebraically. Use the approprite variable name for the input.,

\end{enumerate}

\end{enumerate}

\end{exercise}

\end{document}