\documentclass{ximera}
\title{Quadratic Functions, More Inverses}

\newcommand{\pskip}{\vskip 0.1 in}

\begin{document}
\begin{abstract}
Quadratic functions and more about inverse functions.
\end{abstract}
\maketitle

\begin{question}    \label{Q1:Quadratics}
Let 
\[
    f(x) = x^2 \, , x \in \mathbb{R} ,
\]
and 
\[
     g(x) = \sqrt{x} \, , x\geq 0 .
\]

(a) Explain why the inverse of the function $f$ is not a function.

(b) Let 
\[
    h(x) = g(f(x)) .
\]

(i) Find the domain of $h$.

(ii) Simplify $h(x)$.

(iii) Graph the function $y=h(x)$.

\pskip

(c) Answer parts (i)-(iii) above for the function
\[
   j(x) = f(g(x)) .
\]

(d) Let
\[
   k(x) = x^2 \, , x\leq 0 .
\]

(i) Is the inverse of the function $k$ a function? Explain.

(ii) Find an expression for $k^{-1}(y)$ by expressing the input to the function $y=k(x)$ in terms of its output.

(iii) Find the domain of $k^{-1}(x)$.

(iv) Simplify the composition $k^{-1}(k(x))$. Graph the composition.

\end{question}



\begin{exploration} \label{E1:Quadratics}
The animation below shows water draining from a tank. Play the animation and sketch by hand a graph of the function $V=f(t)$ that expresses the depth of the water as a function of time. Explain your reasoning.


\pdfOnly{
Access Desmos interactives through the online version of this text at
 
\href{https://www.desmos.com/calculator/pdghky6tie}.
}
 
\begin{onlineOnly}
    \begin{center}
\desmos{pdghky6tie}{900}{600}
\end{center}
\end{onlineOnly}
\end{exploration}


\begin{example} \label{Ex1:Quadratics}
The function 
\[
     V = f(t) = \frac{1}{3}(t-6)^2 \, , 0\leq 6 \leq t ,
\]
expresses the depth of water in a tank, measured in feet, in terms of the number of minutes past noon. 

(a) Sketch a graph of the function $V=f(t)$.

(b) Is the inverse of the function $f$ also a function? Explain your reasoning.

(c) Desribe the function $f^{-1}$ in this scenario. What does it take an input? What does it return as an output?

(d) Find an expression for the function $t = f^{-1}(V)$. Find its domain.

(e) For the compositions
\[
    f^{-1}\circ f \text{  and  } f\circ f^{-1} ,
\]

(i) find their domains. State each as a set. Use the appropriate variable name for each.

(ii) Use the appropriate variable name for the inputs and simplify each composition algebraically.

(iii) Graph the compostions.


\end{example}


\begin{exploration} \label{E2:Quadratics}
The animation below shows the motion of a balloon. Play the animation and sketch by hand a graph of the function $h=f(t)$ that expresses the height of the balloon as a function of time. Explain your reasoning.


\pdfOnly{
Access Desmos interactives through the online version of this text at
 
\href{https://www.desmos.com/calculator/95v58n7mti}.
}
 
\begin{onlineOnly}
    \begin{center}
\desmos{95v58n7mti}{900}{600}
\end{center}
\end{onlineOnly}
\end{exploration}



\begin{exploration} \label{E4:Quadratics}
The graph of the function 
\[
\   h=f(t) \, , 0\leq t \leq 90, 
\]
that expresses the height of the balloon (measured in feet) in terms of the number of minutes past noon is shown below. 

\pdfOnly{
Access Desmos interactives through the online version of this text at
 
\href{https://www.desmos.com/calculator/wxeu32b4dp}.
}
 
\begin{onlineOnly}
    \begin{center}
\desmos{wxeu32b4dp}{900}{600}
\end{center}
\end{onlineOnly}


Let 
\[
    T = u(h)
\]
be a function that expresses the time (measured in minutes past noon) when the balloon is $h$ feet above the ground and on its way up.

Let 
\[
    T = d(h)
\]
be a function that expresses the time (measured in minutes past noon) when the balloon is $h$ feet above the ground and on its way down.

\pskip

(a) Find the domains and ranges of the functions $u$ and $d$. Use set notation.

(b) Explain the meaning of the composition $u\circ f$. Find the domain and range of the composition. Use the graph of $f$ above to sketch a graph of the composition.

(c) Repeat part (b) for the composition $d\circ f$.

(d) Repeat part (b) for the compositions $f\circ u$ and $f\circ d$.


\end{exploration}



\end{document}