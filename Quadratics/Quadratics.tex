\documentclass{ximera}
\title{Quadratic Functions, More Inverses}

\newcommand{\pskip}{\vskip 0.1 in}

\begin{document}
\begin{abstract}
Quadratic functions and more about inverse functions.
\end{abstract}
\maketitle

\begin{question}    \label{Q1:Quadratics}
Let 
\[
    f(x) = x^2 \, , x \in \mathbb{R} ,
\]
and 
\[
     g(x) = \sqrt{x} \, , x\geq 0 .
\]

(a) Explain why the inverse of the function $f$ is not a function.

(b) Let 
\[
    h(x) = g(f(x)) .
\]

(i) Find the domain of $h$.

(ii) Simplify $h(x)$.

(iii) Graph the function $y=h(x)$.

\pskip

(c) Answer parts (i)-(iii) above for the function
\[
   j(x) = f(g(x)) .
\]

(d) Let
\[
   k(x) = x^2 \, , x\leq 0 .
\]

(i) Is the inverse of the function $k$ a function? Explain.

(ii) Find an expression for $k^{-1}(y)$ by expressing the input to the function $y=k(x)$ in terms of its output.

(iii) Find the domain of $k^{-1}(y)$.

(iv) Simplify the composition $k^{-1}(k(x))$. Graph the composition.

\end{question}


\begin{example}     \label{Ex99:Quadratics}
Let 
\[
   f(x) = \sqrt{25-x^2} \, , -5 \leq x \leq 5 .
\]

(a) Use your knowledge of circles to sketch a graph of the function $y=f(x)$ by hand.

(b) Explain what it means for a funtion to be one-to-one.

(c) Is the function $f$ one-to-one? Explain your reasoning.

(d) Explain what it means for a relation to be a function.

(e) Is the inverse of $f$ a function? Explain your reasoning.

\begin{explanation}

(a) Since 
\[
    y = \sqrt{25-x^2},
\]
we know that $y \geq 0$. Then squaring both sides and doing some algebra (you should fill in the details) gives
\[
    x^2 + y^2 =25 \text{ and } y\geq 0.
\]
The curve 
\[
   x^2 + y^2 = 25
\]
is a circle of radius 5 centered at the origin. But since $y\geq 0$, the graph (not shown) of the function $y=f(x)$ is a semicircle in the first and second quadrants with endpoints $(5,0)$ and $(-5,0)$.

(b) A function is one-to-one if any two distinct inputs have distinct outputs.

(c) The function $f$ is {\bf not} one-to-one. We can see this because the two inputs $x=-5$ and $x=5$ give the same output
\[
     f(\pm 5) = \sqrt{25-(\pm 5)^2} = 0 .
\]

We can also see that $f$ is not one-to-one from the graph. There are many horizontal lines that intersect the graph in two points. This means that there are many pairs of inputs that give the same output.

(d) A relation is a function when each input has exactly one output.

(e) Because $f$ is not one-to-one its inverse is {\bf not} a function. For example, the input $y=0$ to the inverse relation would have two outputs, $x=-5$ or $x=5$, making the inverse relation not a function.

\end{explanation}

\end{example}


\begin{example} \label{Ex98:Quadratics}
Let 
\[
    y=   f(x) = \sqrt{25-x^2} \, , 0\leq x \leq 5 .
\]

(a) Explain why $f$ is one-to-one.

(b) Find an expression for the inverse function
\[
    x = f^{-1}(y)
\]
that expresses the input $f$ in terms of its output. State its domain.

(c) Graph the function $y=f^{-1}(x)$.


\begin{explanation}
(a) The graph of $y=f(x)$ is a quarter-circle in the first quadrant centered at the origin with endpoints $(5,0)$ and $(0,5)$. Since each horizontal line intersects the graph in at most one point, any two different inputs to $f$ have different ouputs. This means that $f$ is one-to-one.

(b) We know from the graph in part (a) that the range of $f$ is the set
\[
   \{ y \, | \,  0\leq y \leq 5 \} .
\]
This is the domain of the function $x=f^{-1}(y)$.

Now to find an expression for $f^{-1}(y)$, we solve the equation
\[
   y=   \sqrt{25-x^2} \, , 0\leq x \leq 5 
\]
for $x$. Squaring both sides gives
\[
   y^2 = 25 - x^2 .
\]
So 
\[
     x^2 = 25-y^2
\]
and 
\[
   x  = \pm \sqrt{25-y^2} .
\]

To decide which sign to choose, we go back to the domain of $f$, where we are given that 
\[
     0 \leq x \leq 5 .
\]
Since $x\geq 0$, this means we must choose the positive sign and so
\[
    x = f^{-1}(y) = \sqrt{25-y^2}.
\]
We should also include the domain $0\leq y \leq 5$ (as stated above). So the inverse of the function
\[
    y=   f(x) = \sqrt{25-x^2} \, , 0\leq x \leq 5 
\]
is the function
\[
    x = f^{-1}(y) = \sqrt{25-y^2} \, , 0\leq y \leq 5 .
\]

(c) Since the variables $x$ and $y$ have no practical meaning, we can switch them in our expression for $x = f^{-1}(y)$. So the inverse of the function
\[
    y=   f(x) = \sqrt{25-x^2} \, , 0\leq x \leq 5 
\]
is the function
\[
    y = f^{-1}(x) = \sqrt{25-x^2} \, , 0\leq x \leq 5 .
\]
This is identical to the function $f$. That is, $f$ is its own inverse and the graphs of $f$ and $f^{-1}$ are identical.

\end{explanation}

\end{example}

\begin{example}  \label{Ex97:Quadratics}
Answer parts (a)-(c) of Example \ref{Ex98:Quadratics} for the function
\[
    y=   f(x) = \sqrt{25-x^2} \, , -5 \leq x \leq 0 .
\]

\end{example}

\begin{example}  \label{Ex96:Quadratics}
Answer parts (a)-(c) of Example \ref{Ex98:Quadratics} for the function
\[
    y=   f(x) = - \sqrt{25-x^2} \, , -5 \leq x \leq 0 .
\]

\end{example}



\begin{exploration} \label{E1:Quadratics}
The animation below shows water draining from a tank. Play the animation and sketch by hand a graph of the function $V=f(t)$ that expresses the depth of the water as a function of time. Explain your reasoning.


\pdfOnly{
Access Desmos interactives through the online version of this text at
 
\href{https://www.desmos.com/calculator/pdghky6tie}.
}
 
\begin{onlineOnly}
    \begin{center}
\desmos{pdghky6tie}{900}{600}
\end{center}
\end{onlineOnly}
\end{exploration}


\begin{example} \label{Ex1:Quadratics}
The function 
\[
     V = f(t) = \frac{1}{3}(t-6)^2 \, , 0\leq t \leq 6 ,
\]
expresses the depth of water in a tank, measured in feet, in terms of the number of minutes past noon. 

(a) Sketch a graph of the function $V=f(t)$.

(b) Is the inverse of the function $f$ also a function? Explain your reasoning.

(c) Desribe the function $f^{-1}$ in this scenario. What does it take an input? What does it return as an output?

(d) Find an expression for the function $t = f^{-1}(V)$. Find its domain.

(e) For the compositions
\[
    f^{-1}\circ f \text{  and  } f\circ f^{-1} ,
\]

(i) find their domains. State each as a set. Use the appropriate variable name for each.

(ii) Use the appropriate variable name for the inputs and simplify each composition algebraically.

(iii) Graph the compostions.


\end{example}


\begin{exploration} \label{E2:Quadratics}
The animation below shows the motion of a balloon. Play the animation and sketch by hand a graph of the function $h=f(t)$ that expresses the height of the balloon as a function of time. Explain your reasoning.


\pdfOnly{
Access Desmos interactives through the online version of this text at
 
\href{https://www.desmos.com/calculator/95v58n7mti}.
}
 
\begin{onlineOnly}
    \begin{center}
\desmos{95v58n7mti}{900}{600}
\end{center}
\end{onlineOnly}
\end{exploration}



\begin{exploration} \label{E4:Quadratics}
The graph of the function 
\[
\   h=f(t) \, , 0\leq t \leq 90, 
\]
that expresses the height of the balloon (measured in feet) in terms of the number of minutes past noon is shown below. 

\pdfOnly{
Access Desmos interactives through the online version of this text at
 
\href{https://www.desmos.com/calculator/wxeu32b4dp}.
}
 
\begin{onlineOnly}
    \begin{center}
\desmos{wxeu32b4dp}{900}{600}
\end{center}
\end{onlineOnly}


Let 
\[
    T = u(h)
\]
be a function that expresses the time (measured in minutes past noon) when the balloon is $h$ feet above the ground and on its way up.

Let 
\[
    T = d(h)
\]
be a function that expresses the time (measured in minutes past noon) when the balloon is $h$ feet above the ground and on its way down.

\pskip

(a) Find the domains and ranges of the functions $u$ and $d$. Use set notation.

(b) Explain the meaning of the composition $u\circ f$. Find the domain and range of the composition. Use the graph of $f$ above to sketch a graph of the composition.

(c) Repeat part (b) for the composition $d\circ f$.

(d) Repeat part (b) for the compositions $f\circ u$ and $f\circ d$.

\end{exploration}




\begin{example} \label{Ex7:Quadratics}
The function $y=g(x)$ has domain
\[
   \{  x \, | \, 4 \leq x \leq 6   \}.
\]
The graph of the function $f$ is shown below.

(a) Find the domain of the compoisition $g\circ f$. Write it as a set.

(b) What can you say about the domain of the compostion $f\circ g$?


\pdfOnly{
Access Desmos interactives through the online version of this text at
 
\href{https://www.desmos.com/calculator/fjbqii5wbo}.
}
 
\begin{onlineOnly}
    \begin{center}
\desmos{fjbqii5wbo}{900}{600}
\end{center}
\end{onlineOnly}


\end{example}



\begin{example} \label{Ex8:Quadratics}
The function 
\[
   h = f(t) = 320 - 5 (t-6)^2 \, , 0\leq t \leq 14 ,
\]
expresses the height of a rock (in meters) in terms of the number of seconds past noon.

\pdfOnly{
Access Desmos interactives through the online version of this text at
 
\href{https://www.desmos.com/calculator/ew9jsipj5z}.
}
 
\begin{onlineOnly}
    \begin{center}
\desmos{ew9jsipj5z}{900}{600}
\end{center}
\end{onlineOnly}

(a) Find a function $T=u(h)$ that takes as an input a height (in meters) and returns as an output the time (in seconds past noon) when the rock was at that height and on its way up. Include the domain.

(b) Explain the meaning of the composition $u\circ f$.

(c) Find the domain of the composition $u\circ f$. Write it as a set with the appropriate variable.

(d) Use the graph of the function $f$ above to sketch a graph of the function $T= u(f(t))$. Explain your reasoning.

(e) Simplify the composition $T=u(f(t))$ alegebraically. Your final expression should involve the absolute value function. Then graph the composition and see how it compares with your sketch from part (d).  %Then find a piecewise-defined equivalent form of the composition.

(f) Find the inverse of the function $u$. Stop and think. You need not do any algebra for this.

(g) Repeat parts (a)-(f) for the function $T=d(h)$ that takes as an input a height (in meters) and returns as an output the time (in seconds past noon) when the rock was at that height and on its way down.

\end{example}


\end{document}