\documentclass{ximera}
\title{Composition of Functions}

\newcommand{\pskip}{\vskip 0.1 in}

\begin{document}
\begin{abstract}
Composing functions.
\end{abstract}
\maketitle


Composing functions is a lot like composing actions. For example, consider the actions of putting on your socks and putting on your shoes. Two key points. First is that order matters. We get a completely different result if were to first put on our shoes and then our socks. And depending on how flexible your socks are, it might not even be possible to put your socks over your shoes.

The second point is that to undo the compostion of first putting on your socks and then your shoes, we undo the two actions in the \emph{reverse} order - we first take off our shoes and then our socks. And whether we're talking about socks and shoes or functions, it pays to remember this idea.

\begin{example}  \label{Erer54r56t364}
Use the graphs of the functions $y=f(x)$ and $y=g(x)$ below to find approximate answers to the following questions. Drag points $A$ and $B$ as needed to help with your approximations. Use set notation to write solution sets when solving an equation.

\begin{onlineOnly}
    \begin{center}
\desmos{zxb4adsbjb}{900}{600}
\end{center}
\end{onlineOnly}

\href{https://www.desmos.com/calculator/zxb4adsbjb}{141: Composition 1}

\begin{enumerate}

\item Evaluate $f(g(-2))$.

\item Evalute $g(f(0))$.

\item Solve the equation $g(f(x))=10$.

\item Solve the equation $f(g(x)) = 10$.

\item Solve the equation $g(f(x)) = -10$.

\item Solve the equation $g(f(x)) = 0$.

\item Solve the equation $f(g(x)) = 0$.

\item Solve the equation $f(f(x)) = -10$.

\item Solve the equation $g(g(x)) = 2$.

\end{enumerate}

\end{example}


\begin{example}  \label{Ex0:Comp}
Let
\[
    f(x) = \frac{24}{x^2-x-6}
\]
and 
\[
    g(x) = \frac{3x-5}{x-4} .
\]


Answer each of the following questions \emph{without} using the quadratic formula. Use set notation to write solution sets when solving an equation.
\begin{enumerate}
\item Evaluate $f(g(3))$.

\item Evalute $g(f(0))$.

\item Solve the equation $g(f(x))=23/10$.

\item Solve the equation $g(f(x)) = 17/8$.

\item Solve the equation $f(g(x)) = -4$.

\item Solve the equation $f(g(x)) = -6$.

\item Solve the equation $g(g(x)) = 2$.

\item Solve the equation $f(f(x)) = -4$.
\end{enumerate}


\begin{explanation}
(a) To evaluate a composition, we work our way from the inside out. So to evaluate $f(g(3))$, we first evaluate $g(3)$ to get
\[
    g(3) = \frac{(3)(3)-5}{3-4} = -4 .
\]
Then we input $g(3) = -4$ into the function $f$ to get
\[
   f(g(3)) = f(-4) = \frac{24}{(-4)^2 - (-4) -2} = \frac{4}{3} .
\]


\end{explanation}

\end{example}



\begin{question} \label{Ex:435rtggytt}
Use the graphs of the function $y=f(x)$ and $y=g(x)$ to approximate the answers the following questions. Use the method of $u$-substitution. Drag the sliders to the appropriate positions and show screenshots to help with your explanations. Do \emph{not} make any computations.

Write solutions to equations and inequalities using set-builder notation, \emph{not} interval notation.

\begin{onlineOnly}
    \begin{center}
\desmos{xkgmezvis2}{900}{600}
\end{center}
\end{onlineOnly}


\pskip \pskip

(a) Evaluate $f(g(3))$.

(b) Solve the equation $f(g(x)) = 3$.

(c) Solve the equation $g(f(x))=1$.

(d) Solve the equation $f(f(x)) = 5$.

(e) Solve the equation $g(g(x)) = 0$.

(f) Solve the inequality $f(x)>2$.

(g) Solve the inequality $f(x)<6$.

(h) Solve the inequality $g(x)<0$.

(i) Solve the inequality $f(g(x))\geq 3$.

(j) Solve the equation $g(x)=x$.

(k) Solve the equation $f(g(x))=x$.

(l) Solve the equation $f(x)=2 - 3(x-3)$.

(m) Solve the equation $g(x) = 4+(x-3)$.


\end{question}



\begin{example} \label{Ex2:Comp}
Use the function
\[
   g(x) = \frac{x}{x-5} .
\]
and the graph of the function $y=f(x)$ below to find approximate answers to the following questions. Include a screenshot for each question to help with your explanations, being sure to drag the vertical and horizontal lines to their appropriate positons. 

(a) Evaluate $g(f(7))$.

(b) Evaluate $g(f(3))$

(c) Evaluate $f(g(10))$.

(d) Evaluate $f(g(0))$.

(e) Evaluate $f(f(5))$.

(f) Solve the equation $f(g(x))=5$.

(g) Solve the equation $f(g(x))=-5$.

(h) Solve the equation $g(f(x))=-5$.

(i) Solve the equation $g(f(x))=1$.

(j) Solve the equation $f(f(x))=0$.

\pdfOnly{
Access Desmos interactives through the online version of this text at
 
\href{https://www.desmos.com/calculator/eefrziwbhk}.
}
 
\begin{onlineOnly}
    \begin{center}
\desmos{eefrziwbhk}{900}{600}
\end{center}
\end{onlineOnly}

\begin{explanation}
(a) To evaluate $g(f(7))$, we first use the graph to approximate $f(7)$. To do this, drag point $A$ to have coordinates $(7,0)$ and read off the $y$-coordinate of the point where the vertical line through $A$ intersects the graph. This tells us that $f(7)= 3$. Now use the expression for $g$ to get
\[
     g(f(7)) = g(3) = \frac{3}{3-5} = -3/2.
\]

(b) To evaluate $g(f(3))$, we do the same as in part (a). From the graph we know that $f(3)=5$. Then
\[
  g(f(3)) = g(5) = \frac{5}{5-5} = \frac{5}{0}
\]
and so $g(f(3))$ is undefined.

(c) To evaluate $f(g(10))$, we first evaluate $g(10)$ to get
\[
     g(10) = \frac{10}{10-5} = \frac{10}{5} =2.
\]
Then use the graph of $f$ to see $f(2) \sim 3.2$. Our conclusion is that 
\[
    f(g(10)) =f(2) \sim 3.2 .
\]

\end{explanation}


\end{example}




\begin{example}  \label{Ex3:Comp}
The function
\[
     h = f(s) , 0\leq s \leq 25 ,
\]
expresses the altitude (in thousands of feet) in terms of your trip odometer reading (in miles) as you drive along a road in Colorado. 

The function 
\[
  T = g(h) , 0\leq h \leq 11 ,
\]
expresses the temperature (in $^\circ$C) in terms of the altitude (in thousands of feet) at points along the road.

Translate each of the following if possible, either from math to English or vice-versa. Then use the graphs of the functions $f$ and $g$ below to find approximate answers to both the mathematical questions and (in complete sentences) the English questions.

(a) Evaluate $g(f(8))$.

(b) Evaluate $f(g(8))$.

(c) Find the temperature at an altitude of 10,000 feet.

(d) Find the temperature when the odometer reads 10 miles.

(e) At what altitude(s) is the temperature $10^\circ$?

(f) At what altitude(s) is the temperature $3^\circ$?

(g) Solve the equation $g(f(s)) = 5$.

(h) Solve the equation $g(f(s)) = 9$.

(i) Solve the equation $g(f(s)) = g(f(8))$.

(j) At what odometer readings is the temperature at most $5^\circ$C?

(k) Find the coordinates of the turning points of the function $f$. Describe each as a local minimum or local maximum and explain their significance in the context of this scenario.

(l)  Find the coordinates of the turning points of the function $g$. Describe each as a local minimum or local maximum and explain their significance in the context of this scenario.

(m) Find the coordinates of the turning points of the function $g\circ f$. Describe each as a local minimum or local maximum and explain their significance in the context of this scenario.

(n) Compute the average rate of change in altitude with respect to the odometer reading between odometer readings of 10 miles and 18 miles. Include units and explain the meaning in the context of this scenario.

(o) Compute the average rate of change in temperature with respect to altitude between odometer readings of 10 miles and 18 miles. Include units and explain the meaning in the context of this scenario.

(p) Compute the average rate of change in temperature with respect to the odometer reading between odometer readings of 10 miles and 18 miles. Include units and explain the meaning in the context of this scenario.

(q) Find a relationship among the three rates of change in parts (n)-(p).


\begin{exploration}\label{Exp3:Comp}

\pdfOnly{
Access Desmos interactives through the online version of this text at
 
\href{https://www.desmos.com/calculator/lvcgcrrsqu}.
}
 
\begin{onlineOnly}
    \begin{center}
\desmos{lvcgcrrsqu}{900}{600}
\end{center}
\end{onlineOnly}
\end{exploration}

\end{example}



\begin{example} \label{Ex4:Comp}
In the context of the previous example, explain each of the following in plain English (without using any mathematical notation or vocabulary and without referring to the graphs):

\pskip

(a) What we are trying to find when we solve the equation $g(f(s)) = 10$.

(b) How we go about answering the English version of the above question.

\end{example}


\begin{question}   \label{Q1:Comp}
There are two conditions for $x$ to be in the domain of the composition $f(g(x))$. What are they?
\end{question}

\begin{example} \label{Ex1:Comp}
Let 
\[
     f(x) = \frac{x+3}{7x+23}
\]
and
\[
    g(x) = \frac{x}{x-5} .
\]

Use the answer to Question 1 to help you with some of the following questions. Do {\bf not} compute any compositions. Express each domain in set notation.
 
(a) Find the domain of $f$.

(b) Find the domain of $g$.

(c) Find the domain of $g\circ f$.

(d) Find the domain of $f\circ g$.

(e) Find the domain of $g\circ g$.

(f) Find the domain of $g\circ g\circ g$.

\end{example}


\end{document}