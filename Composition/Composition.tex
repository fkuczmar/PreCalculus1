\documentclass{ximera}
\title{Composition of Functions}

\newcommand{\pskip}{\vskip 0.1 in}

\begin{document}
\begin{abstract}
Composing functions.
\end{abstract}
\maketitle

\begin{question}   \label{Q1:Comp}
There are two conditions for $x$ to be in the domain of the composition $f(g(x))$. What are they?
\end{question}

\begin{example} \label{Ex1:Comp}
Let 
\[
     f(x) = \frac{x+3}{7x+23}
\]
and
\[
    g(x) = \frac{x}{x-5} .
\]

Use the answer to Question 1 to help you with some of the following questions. Do {\bf not} compute any compositions. Express each domain in set notation.
 
(a) Find the domain of $f$.

(b) Find the domain of $g$.

(c) Find the domain of $g\circ f$.

(d) Find the domain of $f\circ g$.

(e) Find the domain of $g\circ g$.

(f) Find the domain of $g\circ g\circ g$.

\end{example}


\begin{example} \label{Ex2:Comp}
Use the function
\[
   g(x) = \frac{x}{x-5} .
\]
and the graph of the function $y=f(x)$ below to find approximate answers to the following questions.

(a) Evaluate $g(f(7))$.

(b) Evaluate $g(f(3))$

(c) Evaluate $f(g(10))$.

(d) Evaluate $f(g(0))$.

(e) Evaluate $f(f(5))$.

(f) Solve the equation $f(g(x))=5$.

(g) Solve the equation $f(g(x))=-5$.

(h) Solve the equation $g(f(x))=-5$.

(i) Solve the equation $g(f(x))=1$.

(j) Solve the equation $f(f(x))=0$.

\pdfOnly{
Access Desmos interactives through the online version of this text at
 
\href{https://www.desmos.com/calculator/pzcx5cosis}.
}
 
\begin{onlineOnly}
    \begin{center}
\desmos{pzcx5cosis}{900}{600}
\end{center}
\end{onlineOnly}

\end{example}




\begin{example}  \label{Ex3:Comp}
The function
\begin{exploration}\label{Exp3:Comp}

\pdfOnly{
Access Desmos interactives through the online version of this text at
 
\href{https://www.desmos.com/calculator/7inaza5mft}.
}
 
\begin{onlineOnly}
    \begin{center}
\desmos{7inaza5mft}{900}{600}
\end{center}
\end{onlineOnly}
\end{exploration}

\end{example}

\end{document}