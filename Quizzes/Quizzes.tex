\documentclass{ximera}
\title{Quiz Solutions}


\newcommand{\pskip}{\vskip 0.1 in}

\begin{document}
\begin{abstract}
Chapter quizzes
\end{abstract}
\maketitle


\pskip

\section*{Circle Fundamentals}


\begin{question}  \label{Q:9444433331350}
Imagine a railroad track $L$ feet long secured only at its two ends to the desert floor. Suppose on a hot day the track expands one foot and buckles into a curve. We would like to compute, or at least approximate, the height of the track above the ground at its midpoint. We really need calculus to determine the shape of the curve and the exact height, but we can get a pretty good approximation by supposing the track bends into two straight segments running from the high point to its endpoints.

Make the assumption above and find an expression for the height of the track above its midpoint. Simplify your expression as much as possible. 

Explain your reasoning behind each solution. Define, with units, any variables you introduce. Include a picture to help with your explanation.

\begin{explanation}
In the figure below, the track is pinned at points $A$ and $B$. Point $M$ is the midpoint of the track and $C$ the point on the ground directly below $M$.

\begin{onlineOnly}
    \begin{center}
\desmos{uio1ixkbxe}{900}{600}
\end{center}
\end{onlineOnly}

\href{https://www.desmos.com/calculator/uio1ixkbxe}{141: Railroad Track}


Assuming the expanded track runs along segments $\overline{AM}$ and $\overline{MB}$,  segment $\overline{AM}$ has length
\[
   AM = \frac{L+1}{2}
\]
equal to half the length of the expanded track, measured in feet.

Segment $\overline{AC}$ has length
\[
  AC = L/2
\]
equal to half the length of the original track (also in feet). 

Let $h$ be the height of the midpoint of the expanded track above the ground, measured in feet. Then by the Pythagorean theorem in right triangle $\Delta ACM$,
\[
    \left(  \frac{L}{2} \right)^2  + h^2 =    \left( \frac{L+1}{2} \right)^2 .
\]
and (algebra left for you to fill in)
\[
      h^2 = \frac{2L+1}{4} .
\]
Since $h>0$,
\[
    h = \frac{\sqrt{2L+1}}{2}
\]

So the midpoint of the expanded track is approximately $\frac{1}{2}\sqrt{2L+1}$ feet above the ground.

\end{explanation}


\end{question}


\begin{question}  \label{Q8dsf8r3tg;lyhg}
(a) Find an equation of the circle centered at the point $A(-3,2)$ that passes through the point $B(-7,-3)$. 

(b) Explain the \emph{logic} behind your equation. 

\begin{explanation}

(a)-(b) By the Pythagorean theorem, the circle has radius 
\[
       \text{dist}(A,B) = \sqrt{(3-(-7))^2 + (2-(-3))^2} = \sqrt{41} .
\]

The circle is the set of points exactly $\sqrt{41}$ units from its center $A(-3,2)$. So a point $P$ with coordinates $(x,y)$ lies on the circle if and only if 
\[
  \text{dist} (A,P) = \sqrt{41} .
\]
Using the Pythagorean theorem again, the distance $AP$ is
\[
        \text{dist} (A,P) = \sqrt{(x+3)^2 + (y-2)^2} .
\]
So the point $P(x,y)$ lies on the circle if and only if 
\[
   \sqrt{(x+3)^2 + (y-2)^2} = \sqrt{41}.
\]

That's an equation of the circle centered at $A(-3,2)$ passing through the point $B(-7,-3)$.


\end{explanation}



\end{question}


%\begin{question}  \label{Q:009357865}
%Find equations of two circles with radius $4$ that are tangent to both coordinate axes. Draw a picture to help with your explanation.
%\end{question}

\begin{question}  \label{Q:df454tt4443}
Find the coordinates of the point on the circle centered at the origin with radius $6$ that is farthest from the point $A(3,-5)$. Draw a picture to help with your explanation.

\begin{hint}
The point on the circle farthest from the point $A(3,-5)$ lies on the line through the origin and $A$. Start by finding an equation of this line in the form $y=f(x)$. Then substitute $f(x)$ for $y$ in the equation of the circle and solve for $x$. Keep going...
\end{hint}

\begin{explanation}

Imagine the largest circle centered at the point $A(3,-5)$ that intersects the given circle ${\cal C}$. This largest circle is tangent to ${\cal C}$ at the point we are looking for (ie. the point on ${\cal C}$ farthest from $A$.) Try to convince yourself of this by dragging the slider $r$ in the worksheet below.  
\begin{onlineOnly}
    \begin{center}
\desmos{wwqosobejf}{900}{600}
\end{center}
\end{onlineOnly}

\href{https://www.desmos.com/calculator/wwqosobejf}{141: Quiz 1 Q3}

Since the line through the centers of two circles tangent to each other passes through the point of tangency, the point we're looking for, call it $P$, is the point in the second quadrant where the line through the origin and $A(3,-5)$ intersects the given circle. To find the coordinates of $P$, we'll use algebra.

First, an equation of the circle with radius $6$ centered at the origin is
\[
     \sqrt{x^2 + y^2} = 6.
\] 
Second, an equation of the line through the origin and $A(3,-5)$ is
\[
     y = -\frac{5}{3}x .
\]
Substituting this expression for $y$ in the equation of the circle gives
\[
  \sqrt{x^2 + \frac{25}{9}x^2} = 6 
\]
or equivalently
\[
      \sqrt{\frac{34}{9}x^2} = 6 .
\]
So
\[
       \frac{\sqrt{34}}{3}|x| = 6
\]
and
\[
     | x | = \frac{18}{\sqrt{34}} .
\]
This tells us that 
\[
     x = \pm \frac{18}{\sqrt{34}} .
\]
So the line $OA$ through the origin and $A$ intersects the circle in points with $x$-coordinates $x=\pm 18/\sqrt{34}$.  The point with $x$-coordinate $x=18/\sqrt{34}$ is in the fourth quadrant and is the point of ${\cal C}$ \emph{closest} to $A$. But we want the point $P$ \emph{farthest} from $A$. This is in the second quadrant. Its $x$-coordinate is $x = -18/\sqrt{34}$. And its $y$-coordinate is
\[
    y = \left( -\frac{5}{3} \right) \left(-  \frac{18}{\sqrt{34}}  \right) = \frac{30}{\sqrt{34}}.
\]

So the point on the circle with radius $6$ centered at the origin closest to $A(3,-5)$ has coordinates
\[
   (x,y) = \left(  - \frac{18}{\sqrt{34}} ,  \frac{30}{\sqrt{34}}         \right).
\]

\end{explanation}

\end{question}

\begin{question} \label{Qsdfl4345r3}
Find an equation of the circle through the points $O(0,0)$, $A(6,4)$, and $B(-2,-4)$. Explain your reasoning. Draw a picture to help with your explanation.

\begin{explanation}
The key idea is that the center of a circle through the points $O(0,0)$, $A(6,4)$, and $B(-2,-4)$ is equidistant from all three points. 

\begin{onlineOnly}
    \begin{center}
\desmos{uqzhkgp2ju}{900}{600}
\end{center}
\end{onlineOnly}

\href{https://www.desmos.com/calculator/uqzhkgp2ju}{141: Quiz 1 Q4}

We'll start by ignoring point $B$ and finding an equation of the set of points equidistant from $O$ and $A$. A point $P(x,y)$ is equidistant from the points $O(0,0)$ and $A(6,4)$ if and only if
\[
       \text{dist}(P,O) = \text{dist}(P,A) 
\]
or equivalently if and only if 
\[
   \sqrt{x^2+y^2} = \sqrt{(x-6)^2 + (x-4)^2}  .
\]
Some algebra (left for you) leads to the equation
\[
       3x + 2y = 13 .
\]
for the set of points equidistant from $O$ and $A$. The center of our circle must lie on this line.

The center of our circle must also be equidistant from $O$ and $B$. The set of all such points is described by the condition that
\[
   \text{dist}(P,O) = \text{dist}(P,V) 
\]
and by the equation
\[
   \sqrt{x^2+y^2} = \sqrt{(x+2)^2 + (y+4)^2}  .
\]
Some algebra (left for you) leads to the equation
\[
       x+2y = -5 .
\]
for the set of points equidistant from $O$ and $B$. The center of our circle must also lie on this line.

So it is necessary for the center of our circle to lie on the lines with equations
\[
 \begin{cases}
             3x  + 2y = 13 \\
             x+2y = -5   .
\end{cases}
\]
This is also sufficient because a point that is both equidistant from $O$ and $A$ as well as from $O$ and $B$, is necessarily equidistant from $O$, $A$, and $B$. 

So to find the center of the circle through $O$, $A$, and $B$, we just need to solve the system of equations above. Solving the second equation for $x$ give
\[
     x = -2y - 5.
\]
Subsituting this expression for $x$ in the first equation gives
\[
   3 (-2y-5) + 2y = 13 
\]
and 
\[
          y= -7 .
\]
Then 
\[
   x = -2(-7) - 5  = 9 .
\]

So the center $P$ of the unique circle through $O(0,0)$, $A(6,4)$, and $B(-2,-4)$ has coordinates
\[
    (x,y) = (9, -7).
\]
And since the origin $O$ lies on this circle, the circle has radius 
\[
   \text{dist}(O,P) = \sqrt{9^2 + (-7)^2 = \sqrt{130}}.
\]
 
That's all we need and an equation of the circle through $O(0,0)$, $A(6,4)$, and $B(-2,-4)$ is
\[
  \sqrt{(x-9)^2 + (y+7)^2} = \sqrt{130} .
\]

\end{explanation}

\end{question}



%\begin{question}  \label{Q:df3r44t54}
%Find equations of all circles centered at the point $(-5,3)$ that are tangent to the circle with equation
%\[
%         (x+3)^2 + (y-2)^2 = 20
%\]
%Draw a picture to help with your explanation.
%\end{question}


%\begin{question}  \label{Q:9834342}
%Find equations of all circles through the point $(-2,3)$ that are tangent to both coordinate axes. Draw a picture to help with your explanation.
%\end{question}




\end{document}
