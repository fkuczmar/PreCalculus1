\documentclass{ximera}
\title{Quiz Solutions}


\newcommand{\pskip}{\vskip 0.1 in}

\begin{document}
\begin{abstract}
Chapter quizzes
\end{abstract}
\maketitle


\pskip

\section*{Circle Fundamentals}


\begin{question}  \label{Q:9444433331350}
Imagine a railroad track $L$ feet long secured only at its two ends to the desert floor. Suppose on a hot day the track expands one foot and buckles into a curve. We would like to compute, or at least approximate, the height of the track above the ground at its midpoint. We really need calculus to determine the shape of the curve and the exact height, but we can get a pretty good approximation by supposing the track bends into two straight segments running from the high point to its endpoints.

Make the assumption above and find an expression for the height of the track above its midpoint. Simplify your expression as much as possible. 

Explain your reasoning behind each solution. Define, with units, any variables you introduce. Include a picture to help with your explanation.

\begin{explanation}
In the figure below, the track is pinned at points $A$ and $B$. Point $M$ is the midpoint of the track and $C$ the point on the ground directly below $M$.

\begin{onlineOnly}
    \begin{center}
\desmos{vhuz17uwsa}{900}{600}
\end{center}
\end{onlineOnly}

\href{https://www.desmos.com/calculator/vhuz17uwsa}{141: Railroad Track}


Assuming the expanded track runs along segments $\overline{AM}$ and $\overline{MB}$,  segment $\overline{AM}$ has length
\[
   AM = \frac{L+1}{2}
\]
equal to half the length of the expanded track, measured in feet.

Segment $\overline{AC}$ has length
\[
  AC = L/2
\]
equal to half the length of the original track (also in feet). 

Let $h$ be the height of the midpoint of the expanded track above the ground, measured in feet. Then by the Pythagorean theorem in right triangle $\Delta ACM$,
\[
    \left(  \frac{L}{2} \right)^2  + h^2 =    \left( \frac{L+1}{2} \right)^2 .
\]
and (algebra left for you to fill in)
\[
      h^2 = \frac{2L+1}{4} .
\]
And since $h>0$,
\[
    h = \frac{\sqrt{2L+1}}{2}
\]

So the midpoint of the expanded track is approximately $\frac{1}{2}\sqrt{2L+1}$ feet above the ground.

\end{explanation}


\end{question}


\begin{question}  \label{Q8dsf8r3tg;lyhg}
(a) Find an equation of the circle centered at the point $(-3,2)$ that passes through the point $(-7,-3)$. 

(b) Explain the \emph{logic} behind your equation. 
\end{question}


%\begin{question}  \label{Q:009357865}
%Find equations of two circles with radius $4$ that are tangent to both coordinate axes. Draw a picture to help with your explanation.
%\end{question}

\begin{question}  \label{Q:df454tt4443}
Find the coordinates of the point on the circle centered at the origin with radius $6$ that is farthest from the point $(3,-5)$. Draw a picture to help with your explanation.

\begin{hint}
The point on the circle farthest from the point $A(3,-5)$ lies on the line through the origin and $A$. Start by finding an equation of this line in the form $y=f(x)$. Then substitute $f(x)$ for $y$ in the equation of the circle and solve for $x$. Keep going...
\end{hint}

\end{question}

\begin{question} \label{Qsdfl4345r3}
Find an equation of the circle through the points $(0,0)$, $(6,4)$, and $(-2,-4)$. Explain your reasoning. Draw a picture to help with your explanation.

\end{question}



%\begin{question}  \label{Q:df3r44t54}
%Find equations of all circles centered at the point $(-5,3)$ that are tangent to the circle with equation
%\[
%         (x+3)^2 + (y-2)^2 = 20
%\]
%Draw a picture to help with your explanation.
%\end{question}


%\begin{question}  \label{Q:9834342}
%Find equations of all circles through the point $(-2,3)$ that are tangent to both coordinate axes. Draw a picture to help with your explanation.
%\end{question}




\end{document}
