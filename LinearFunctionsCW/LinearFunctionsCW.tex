\documentclass{ximera}
\title{Linear Functions CW}

\newcommand{\pskip}{\vskip 0.1 in}

\begin{document}
\begin{abstract}
The point-slope equation of a line.
\end{abstract}
\maketitle


\begin{question}  \label{QLer3r3r3}
Between 12:15pm and 1:00pm a balloon descends at a constant rate of $37$ ft/min.

The balloon is at an altitude of $3235$ feet at 12:34 pm.

\begin{enumerate}
\item Use common sense to find the altitude of the balloon at 12:55. Explain your reasoning. Include units for each number in your computation.

\item Using the same logic as in part (a), find an expression for the function 
\[
  h = f(t) \, , \, 15\leq t \leq 60 .
\]
that give the balloon's height (in feet) at time $t$ minutes past noon.

\item Use your function from part (b) to determine when the balloon is at an altitude of $2900$ feet.
\end{enumerate}

\end{question}


\begin{question} \label{QKdfder33}
The graph of a linear function $y=f(x)$ passes through the points $(-4,5)$ and $(7,-3)$.

\begin{enumerate}
\item Sketch a reasonably accurate graph of the function $y=f(x)$.

\item Use point-slope to find an expression for the function $y=f(x)$. Explain your logic. Part of your explanation should include a sentence that begins with the phrase

 \emph{A point $P$ (different from $(4,5)$) with coordinates $(x,y)$ lies on the line through $(-4,5)$ with slope ?? if and only if ...}

\item Use your function from part (b) to solve the inequality
\[
     f(x) < 85 .
\]
Write your answer as a solution set.

\item Solve the inequality
\[
      f(x) < f(85) .
\]
Write your answer as a solution set.
\end{enumerate}
\end{question}

\begin{question} \label{QLfefrre3}
The equation 
\[
  25p + 7q = 300 \, ,\, 5 \leq p \leq 12 ,
\]
relates the average number of burgers $q$ a food truck sells per day in terms of the price $p$ (measured in dollars/burger).

What happens to the average number of burgers/day sold if they raise the price by $\$0.75$/burger? Be specific. Try to solve this problem in general.
 
\end{question}

\begin{question}  \label{Qeeetl4gg}
Between speeds of $55$ miles/hour and $90$ miles/hour, the gas mileage of a Corvette is a linear function of its speed. The car gets $35$ miles/gallon at a speed of $55$ miles/hour and $20$ miles/gallon at a speed of $90$ miles/hour.

\begin{enumerate}

\item Find a function $v=f(G)$ that expresses the speed of the Corvette (in miles/hour) in terms of its gas mileage (in miles/gallon). Start by sketching a graph of the function. Label the axes with the appropriate variable names and units.

\item Use set notation to state the domain and range of $f$.

\item What are the units of the slope of your line in part (a)? Interpret its meaning.

\item At what speed does the car get $30$ miles/gal?

\item What is the gas mileage at a speed of $80$  miles/hour?

\item What happens to the gas mileage when the speed of the car decreases by $5$ miles/hour?

\item How can you increase the gas mileage by $4$ miles/gal?

\item Find a function $r=h(G)$ that expresses the rate at which the car burns gas (in gal/hour) in terms of its gas mileage. State the domain and range of $h$ in set notation.

\item Simplify the units of the slope in part (c). What does this suggest about its meaning? Would that be a correct interpretation?

\item Drag the slider $u$ in the worksheet below to approximate the rate (in miles/hr) at which the Corvette burns gas at a speed of $86$ miles/hour. Then use the appropriate function to compute the exact rate.

\item Drag the slider $u$ in the worksheet below to approximate the speed at which the Corvette burns gas at the rate of $2$ gal/hour. Then use the appropriate function to compute the exact rate.

\begin{onlineOnly}
    \begin{center}
\desmos{2fq4txcouq}{900}{600}
\end{center}
\end{onlineOnly}

\href{https://www.desmos.com/calculator/2fq4txcouq}{141: Corvette}

\item Find a function $G=w(r)$ that expresses the gas mileage (in miles/hour) of the Corvette in terms of the rate (in gal/hour) at which it burns gas. State the domain and range $w$ in set notation.

\end{enumerate}
\end{question}


\end{document}