\documentclass{ximera}
\title{Quadratic Functions, More Inverses CW}

\newcommand{\pskip}{\vskip 0.1 in}

\begin{document}
\begin{abstract}
Quadratic functions and more about inverse functions.
\end{abstract}
\maketitle

\begin{question} \label{Qdrer3r3}
The function
\[
   V =f(t) = 2(t-10)^2 \, , \, 0\leq t \leq 10 ,
\]
expresses the volume of water (measured in gallons) in a tank in terms of the number of minutes past noon.

\begin{enumerate}
\item Is the function $f$ one-to-one? Explain your reasoning as it applies in this particular scenario. Start by sketching a graph of the function by hand.

\item Find an expression for the function $f^{-1}$. Use the appropriate variable names for the input and output.

\item State the domain of $f^{-1}$ in set-builder notation.

\item Use algebra to show that $f^{-1}$ undoes $f$. Use the appropriate variable name as an input.
\end{enumerate}
\end{question}


\begin{question} \label{QPfr3frE}
The function
\[
     G = f(v) = -80 + 5v - \frac{1}{20}v^2 \, , \, 30\leq v \leq 75 ,
\]
expresses the gas mileage of a car (in miles/gal) in terms of its speed (in miles/hour).

\begin{enumerate}
\item Complete the square to write function $f$ in vertex form.

\item Use part (a) to sketch by hand a graph of the function $G=f(v)$.

\item Use your graph to determine if the function $f$ is one-to-one. Explain your reasoning in the context of this \emph{particular} situation.

\item Find an expression for the function
\[
       v = h(G) 
\]
that expresses the speed of the car (in miles/hr) in terms of its gas mileage for speeds between $30$ miles/hour and $50$ miles/hour.

\item Write the domain and range of the function $v=h(G)$ in set-builder notation.

\item Evaluate the following expressions and explain their meanings.
\begin{enumerate}
\item $h(f(40))$

\item $h(f(55))$

\item $h(f(75))$

\end{enumerate}

\item Explain the meaning of the composition 
\[
     G= h(f(v)) .
\]
What does it take as an input? What does it return as an output?

\item Find the domain and range of the composition $h\circ f$.

\item Simplify the composition
\[
         G = h(f(v)) .
\]

\item Graph the composition
\[
     G = h(f(v))
\]
by hand.
\end{enumerate}
\end{question}

\end{document}