\documentclass{ximera}
\title{Logarithmic Functions}

\newcommand{\pskip}{\vskip 0.1 in}

\begin{document}
\begin{abstract}
Logarithmic functions.
\end{abstract}
\maketitle


The most important thing to know about logarithmic functions is that they are the inverses of the exponential functions. So, for example, the inverse of the exponential funciton
\[
    f(x) = 2^x
\]
is the logarithmic function 
\[
   f^{-1}(x) = \log_2 x.
\]
Then 
\[
    f^{-1}(8) = \log_2 (8)    
\]
is the power of $2$ that gives $8$. And since
\[
   2^3 = 8 ,
\]
we know that
\[
   \log_2(8) = 3 .
\]

We can express the description of $\log_28$ as being the power of $2$ that gives $8$ like this:
\[
   2^{\log_2 8} = 8.
\]

\begin{question}    \label{Ex0:LogF}
(a) What is the meaning of $\log_2(10)$? What is its approximate value? (Do not use a calculator.)

(b) What is the {\bf meaning} of $2^{\log_2 10}$ ? What is its value?

(c) What is the {\bf meaning} of $\log_2 2^{10}$ ? What is its value?

\end{question}


\begin{question}  \label{Q1:LogF}
Let 
\[
    f(x) = 3^x .
\]

(a) Find an expression for $f^{-1}(x)$.
\[
   f^{-1}(x) = \answer{\log_3 x}.
\]

(b) What is the domain of $f$?

(c) What is the domain of $f^{-1}$?

(d) Simplify each of the following algebraically. Show all the steps and explain the logic.

(i) $f^{-1}(f(8))$

(ii)  $f^{-1}(f(x))$

(iii) $f^{-1}(81)$

(iv) $f^{-1}(1/81)$

(v) $f^{-1} (\sqrt{3})$

(vi) $f(f^{-1}(6))$

(vii) $f(f^{-1}(x))$


\end{question}


\begin{question}  \label{Ex1:LogF}

{\bf Part 1:} Use the graph of the function $f(x)=2^x$ below to approximate each of the following. Use the sliders to approximate coordinates instead of using Desmos to read off coordinates.

(a) $\log_2 (10)$

(b) $\log_2 3$

(c) $\log_2 0$

\pskip

{\bf Part 2:} Use your approximations from parts (a) and (b) as necessary to approximate each of the following. Do not use the graph. Explain your reasoning using properties of the exponential function, not the log function.

(d) $\log_2 \sqrt{10}$

(e) $\log_2(1/3)$ 

(e)  $\log_2 160$ 

(f) $\log_{10} 2$ 

(g) $\log_2 (10,000)$

(h) $\log_2(30)$

(i) $\log_2(3^{20})$. 

(j) $\log_{3}(30)$.

\begin{exploration}
\pdfOnly{
Access Desmos interactives through the online version of this text at
 
\href{https://www.desmos.com/calculator/mir81iod8b}.
}
 
\begin{onlineOnly}
    \begin{center}
\desmos{mir81iod8b}{900}{600}
\end{center}
\end{onlineOnly}
\end{exploration}

\end{question}


\begin{question} \label{Q4:LogF}
Would you expect $\log_2(24)$ to be greater than or less than $4.5$? Explain your reasoning. Use the graph of the function $y=2^x$ to help. Do not use technology.
\end{question}



One of the main uses of the logarithmic functions is to solve exponential equations.

\begin{question} \label{Q5:LogF}
Let
\[
   P =  f(t) = 5000 (2^{t/8}) , \, 0\leq t \leq 40.
\]

(a) Describe what $f$ does to its input.

(b) Use your response to part (a) to describe what $f^{-1}$ does to its input.

(c) Use your description from part (b) to find an expression for $f^{-1}(P)$. 

(d) What is the domain of $f^{-1}$?

(e) Explain why the following algebraic computation of $f^{-1}(P)$ does {\bf not} match your description from part (b).

\pskip

Since
\[
   P = 5000 (2^{t/8}) ,
\]
\[
   P/5000 = 2^{t/8} ,
\]
and 
\[
   \log_{10}\left( \frac{P}{5000} \right) = \log_{10} (2^{t/8}) .
\]
Then
\[
   \log_{10}\left( \frac{P}{5000} \right) =  \left( \frac{t}{8}  \right)  \log_{10} 2 
\]
and so 
\[
    t = \frac{8 \log_{10}(P/5000)}{\log_{10} 2} .
\]

(f) Change the algebra in part (e) to match your verbal description from part (b).
\end{question}


\end{document}