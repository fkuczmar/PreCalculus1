\documentclass{ximera}
\title{Logarithmic Functions}

\newcommand{\pskip}{\vskip 0.1 in}

\begin{document}
\begin{abstract}
Logarithmic functions.
\end{abstract}
\maketitle


The most important thing to know about logarithmic functions is that they are the inverses of the exponential functions. So, for example, the inverse of the exponential funciton
\[
    f(x) = 2^x
\]
is the logarithmic function 
\[
   f^{-1}(x) = \log_2 x.
\]
Then 
\[
    f^{-1}(8) = \log_2 (8)    
\]
is the power of $2$ that gives $8$. And since
\[
   2^3 = 8 ,
\]
we know that
\[
   \log_2(8) = 3 .
\]

We can express the description of $\log_28$ as being the power of $2$ that gives $8$ like this:
\[
   2^{\log_2 8} = 8.
\]

\begin{question}    \label{Ex0:LogF}
(a) What is the \emph{meaning} of $\log_2(100)$? Approximate its value \emph{without} using a calculator. Explain your reasoning.

(b) What is the {\bf meaning} of $2^{\log_2 100}$ ? Determine its value \emph{without} a calculator. Explain your reasoning.

(c) What is the {\bf meaning} of $\log_2 (2^{100})$ ? Determine its value \emph{without} a calculator. Explain your reasoning.

\end{question}


\begin{question}  \label{Qdr435rtgfg}
Evaluate each of the following expressions \emph{without} using a calculator. Explain your logic.
\begin{enumerate}
\item $\log_3 81$

\item $\log_3 (1/81)$

\item $\log_{81}3$

\item $\log_{81} (1/3)$

\item $\log_5 5$

\item $\log_5 (1/5)$

\item $\log_5 (1/25)$

\item $\log_5 (5^8)$

\item $\log_5(5^8 \cdot 5^4)$

\item $\log_5\left(  \frac{5^{13}}{5^{4}} \right)$

\item $\log_5(25^7)$

\item $\log (1000)$

\item $\log (0.001)$

\item $3^{\log_3 9}$

\item $3^{\log_3 15}$

\item $9^{\log_3 8}$

\item $3^{\log_9 8}$

\item $\log_8 16$

\item $\log_{16}8$

\end{enumerate}
\end{question}

\begin{question} \label{Q435tt5444}
The function
\[
     P = f(t) = 10000 \cdot 2^{\frac{1}{4}(t-5)} \, , \, -10\leq t \leq 14,
\]
expresses the population of a small colony of bacteria in terms of the number of hours past noon.
\begin{enumerate}
\item Does the population increase or decrease? Describe how. Be precise.

\item Find the popuation at 1:00 am without using a calculator.

\item Write and solve an equation to determine when the population is $2500$. Explain your reasoning.

\item Use the graph of the function 
\[
     P = g(u) = 2^u
\]
below to find approximate answers to the following questions by either evaluating the function $f$ or solving an equation with the function $f$. In the latter case use $u$-substitution.

\begin{onlineOnly}
    \begin{center}
\desmos{fslkdtjqjl}{900}{600}
\end{center}
\end{onlineOnly}

\href{https://www.desmos.com/calculator/fslkdtjqjl}{141: Exp Growth 9}

\begin{enumerate}
\item Find the population at 11:00pm.

\item Determine when the population is $25,000$.
\end{enumerate}
\end{enumerate}
\end{question}




\begin{question} \label{Qdferwer}
The function
\[
     P = f(t) = 10000 \cdot 3^{\frac{1}{4}(t-5)} \, , \, -10\leq t \leq 14,
\]
expresses the population of a small colony of bacteria in terms of the number of hours past noon.

Use the graph of the function 
\[
     P = g(u) = 2^u
\]
below to find approximate answers to the following questions by either evaluating the function $f$ or solving an equation with the function $f$. In the latter case use $u$-substitution.

\begin{onlineOnly}
    \begin{center}
\desmos{fslkdtjqjl}{900}{600}
\end{center}
\end{onlineOnly}

\href{https://www.desmos.com/calculator/fslkdtjqjl}{141: Exp Growth 9}

\begin{enumerate}
\item Find the population at 11:00pm.

\item Determine when the population is $25,000$.
\end{enumerate}
\end{question}





\begin{question}  \label{Q1:LogF}
Let 
\[
   y =  f(x) = 3^x .
\]

(a) Find an expression for $f^{-1}(y)$.
\[
      f^{-1}(y) = \answer{\log_3 y}.
\]

(b) Find an expression for $f^{-1}(x)$.
\[
   f^{-1}(x) = \answer{\log_3 x}.
\]

(c) Use set-builder notation, \emph{not} interval notation, to write the domain and range of the function $y=f(x)$. Explain how  you know.

The domain of the function $f(x) = 3^x$ is the set
\[
   \{   x \, | , x\in \mathbb{R}  \}
\]
and the range is the set
\[
    \{   \answer{y} \, | , \answer{y} > \answer{0}  \} .
\]

(d) Use set-builder notation, \emph{not} interval notation, to write the domain and range of the function $y=f^{-1}(x)$. Explain how  you know.

(e) Simplify each of the following algebraically. Show all the steps and explain the logic.

(i) $f^{-1}(f(8))$

(ii)  $f^{-1}(f(x))$

(iii) $f^{-1}(81)$

(iv) $f^{-1}(1/81)$

(v) $f^{-1} (\sqrt{3})$

(vi) $f(f^{-1}(6))$

(vii) $f(f^{-1}(x))$


\end{question}


\begin{question}  \label{Ex1:LogF}

{\bf Part 1:} Use the graph of the function $f(x)=2^x$ below to approximate each of the following. Use the sliders to approximate coordinates instead of using Desmos to read off coordinates. Show a screenshot for each to help with your explanations.

(a) $\log_2 (10)$

(b) $\log_2 3$

(c) $\log_2 (5)$

%(d) $\log_2(1/3)$

(d) $\log_2 0$

\pskip

{\bf Part 2:} Use your approximations from parts (a) and (b) as necessary to approximate each of the following. Do not use the graph. Explain your reasoning using properties of the exponential function, not the log function.

(d) $\log_2 \sqrt{10}$

(e) $\log_2(1/3)$ 

(e)  $\log_2 160$ 

(f) $\log_{10} 2$ 

(g) $\log_2 (10,000)$

(h) $\log_2(30)$

(i) $\log_2(3^{20})$. 

(j) $\log_{3}(30)$.

\begin{exploration}
\pdfOnly{
Access Desmos interactives through the online version of this text at
 
\href{https://www.desmos.com/calculator/mir81iod8b}.
}
 
\begin{onlineOnly}
    \begin{center}
\desmos{mir81iod8b}{900}{600}
\end{center}
\end{onlineOnly}
\end{exploration}

\end{question}


\begin{question}  \label{Q111:LogF}
Make up and answer your own questions like those of parts (a), (b), (d)-(i) of Question 3 using the graph of the function $f(x)=3^x$ shown below. Follow the same directions as above. Give complete explanations.

\begin{exploration}
\pdfOnly{
Access Desmos interactives through the online version of this text at
 
\href{https://www.desmos.com/calculator/ubdjkkq4tu}.
}
 
\begin{onlineOnly}
    \begin{center}
\desmos{ubdjkkq4tu}{900}{600}
\end{center}
\end{onlineOnly}
\end{exploration}





\end{question}


\begin{question} \label{Q4:LogF}
Would you expect $\log_2(24)$ to be greater than or less than $4.5$? Explain your reasoning. Use the graph of the function $y=2^x$ to help. Do not use technology.
\end{question}



One of the main uses of the logarithmic functions is to solve exponential equations.


\begin{question}  \label{Q:dfehyyh5h}
The balance in an account grows exponentially. The account takes $5$ years to grow from $\$10,000$ to $\$15,000$. 

(a) Describe how the balance grows.

(b) Find a function $B = f(t)$ that expresses the balance in the account (in dollars) in terms of the number of years since the balance was $\$10,000$.

(c) Use desmos to graph the function in part (b). 

(d) Use your graph from part (c) to approximate how long it takes the balance to increase from $\$15,000$ to $\$20,000$. Then compute the exact time \emph{without} a calculator. Then give a decimal approximation to the nearest hundredth of a year.

(e) Use your graph from part (c) to approximate how long it takes the balance to increase from $\$40,000$ to $\$45,000$. Then compute the exact time \emph{without} a calculator. Then give a decimal approximation to the nearest hundredth of a year.

\end{question}

\begin{question}  \label{Q:DLDFLLLLL}
How long does it take an investment to double at a nominal rate of $12\%$/yr if interest is compounded

(a) annually?

(b) quarterly?

(c) monthly?

(d) continuously?

Answer each question by first finding a function that expresses the balance in terms of time (be sure to define the input and output to each function and include units in the definitions).

\end{question}


\begin{question}  \label{Q:dfbhhhhh}
A radioactive substance decays exponentially, taking 100 years to decrease from a mass of $200$ grams to a mass of $160$ grams.

(a) Describe how the mass decreases.

(b)  Find a function 
\[
 M= f(t) \, , \, t\geq 0 ,
\]
that expresses the mass (in grams) in terms of the number of years since the mass was $200$ grams.

(c) Use desmos to graph the function in part (b). 

(d) Use your graph from part (c) to approximate how long it takes the mass to decrease from $160$ grams to $120$ grams. Then compute the exact time \emph{without} a calculator. Then give a decimal approximation to the nearest hundredth of a year.

(e) Use your graph from part (c) to approximate how long it takes the mass to decrease by $80\%$. Then compute the exact time \emph{without} a calculator. Then give a decimal approximation to the nearest hundredth of a year.

(f) Find an expressoin for the function 
\[ 
      t  = f^{-1}(M) .
\]
State its domain and range using set-buider notation. Describe also the meaning of this function. What does it take as an input? What does it return as an output.

(g) Use the graph of the function $f$ from part (c) to sketch \emph{by hand} a graph of the function $f^{-1}$. Label the axes with the appropriate variable names and units.

(h) Use your function from part (f) to determine how long it takes the mass to decrease from $200$ grams to $180$ grams.

\end{question}



\begin{question} \label{Q5:LogF}
Let
\[
   P =  f(t) = 5000 (2^{t/8}) , \, 0\leq t \leq 40.
\]

(a) Describe what $f$ does to its input.

(b) Use your response to part (a) to describe what $f^{-1}$ does to its input.

(c) Use your description from part (b) to find an expression for $f^{-1}(P)$. 

(d) What is the domain of $f^{-1}$?

(e) Explain why the following algebraic computation of $f^{-1}(P)$ does {\bf not} match your description from part (b).

\pskip

Since
\[
   P = 5000 (2^{t/8}) ,
\]
\[
   P/5000 = 2^{t/8} ,
\]
and 
\[
   \log_{10}\left( \frac{P}{5000} \right) = \log_{10} (2^{t/8}) .
\]
Then
\[
   \log_{10}\left( \frac{P}{5000} \right) =  \left( \frac{t}{8}  \right)  \log_{10} 2 
\]
and so 
\[
    t = \frac{8 \log_{10}(P/5000)}{\log_{10} 2} .
\]

(f) Change the algebra in part (e) to match your verbal description from part (b).
\end{question}


\begin{question} \label{Q7:LogF}
Between noon and 6pm a population of bacteria grows exponentially. The population is $200,000$ at 2pm and $260,000$ at 3pm.

(a) Find a function 
\[
     P = f(t) \, , 0\leq t \leq 6 ,
\]
that expresses the population in terms of the number of hours past noon. Explain your reasoning thoroughly.

(b) Find an expression for the function $f^{-1}$. Use the appropriate variable names for the input and output. Use the logarithm with the correct base, namely the one that undoes the exponential function.

(c) Express the domain of $f^{-1}$ as a set. Use the appropriate variable name. Explain your reasoning.

(d) Find the {\bf exact} time when the population is $230,000$. Do {\bf not} use a calculator.

(e) Use a calculator to approximate the time to the nearest minute when the population is $230,000$.

\end{question}

\begin{question}  \label{Q:LKKDBBDEvc}
At noon a cup of coffee at a temperature of $90^\circ$C is placed in a room held at a constant temperature of $20^\circ$C. Five minutes later the temperature of the coffee is $80^\circ$C.

Suppose that the positive difference in the temperatures of the room and coffee deceases exponentially.

(a) Describe how the temperature difference changes.

(b) Use  your description from part (a) to find a function
\[
   D = f(t) \, , \, t \geq 0,
\]
that expresses the postive temperature difference (in Celsius degrees) in terms of the number of minutes past noon.

(c) Use part (b) to find a function
\[
         C = g(t) \, , \, t \geq 0,
\]
that expresses the temperature of the coffee in terms of the number of minutes past noon.

(d) Graph the function $g$ \emph{by hand}. Label any asymptotes with their equations. Label the axes with the appropriate variable names and units.

(e) Find an expression for the function $g^{-1}$. Use the appropriate variables for its input and output. Write the domain and range of this function in set-builder notation.

(f) Graph the function $g^{-1}$ \emph{by hand}. Label any asymptotes with their equations. Label the axes with the appropriate variable names and units.

(g) What is the temperature of the coffee at 12:15pm?

(h) When is the temperature of the coffee $50^\circ$?
\end{question}


\end{document}