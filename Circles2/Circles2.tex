\documentclass{ximera}
\title{Circles, Part 2}


\newcommand{\pskip}{\vskip 0.1 in}

\begin{document}
\begin{abstract}
Circles in the Euclidean plane.
\end{abstract}
\maketitle


\pskip




\section*{Circles Through Three Points}

\begin{question}  \label{Q983g4332}
Find an equation of the circle through the points $A(0,0)$, $B(6,0)$ and $C(5,3)$.  

\begin{hint}
Do this as follows:

(a) First find equations of 

(i) the set of points equidistant from $A$ and $B$ and

(ii) the set of points equidistant from $A$ and $C$.

(b) Then use algebra to find the coordinates of the point equidistant from $A$, $B$, and $C$.

(c) Then write an equation of the circle through $A$, $B$, and $C$.

\end{hint}

Then use the desmos worksheet below to check your work.

\begin{onlineOnly}
    \begin{center}
\desmos{6uhkzyev0c}{900}{600}
\end{center}
\end{onlineOnly}

\end{question}




\begin{question}  \label{Ex7sdfsdf}
(a) Find an equation of the circle through the points $O(0,0)$, $A(a,0)$, and $B(b,c)$. 

(b) Input the coordinates of the center and the equation of the circle in the worksheet below. Then drag points $A$ and $B$ to see if you are correct.

\begin{onlineOnly}
    \begin{center}
\desmos{op1ebkzx9n}{900}{600}
\end{center}
\end{onlineOnly}

\href{https://www.desmos.com/calculator/op1ebkzx9n}{141: Circle Through 3 General Points}


\begin{hint}
Since there are three conditions, that the circle pass through the given points, we should expect there to be at most finitely many such circles. 

The key to solving this problem is to find the possible coordinates of the center $P(x,y)$. Since the circle passes through the points $O(0,0)$ and $A(a,0)$, its center is equidistant from these points and its coordinates $(x,y)$ must satisfy the equation
\[
    \sqrt{x^2 + y^2} = \answer{\sqrt{(x-a)^2+y^2}} .
\]
After simplifying this equation, we find that the center $P$ lies on the line  
\[
   x = \answer{a/2}.
\]
Similarly, since $P$ is equidistant from $O(0,0)$ and $B(b,c)$,
\[
  \sqrt{x^2 + y^2} = \answer{\sqrt{(x-b)^2+(y-c)^2}} ,
\]
and so $P$ also lies on the line 
\[
  \answer{2b}x + \answer{2c}y = b^2 + \answer{c^2} .
\]
The center therefore is the point where these two lines intersect and we can find its coordinates by substituting $x=a/2$ into the second equation. Carry out this algebra and input the coordinates $(x_1,y_1)$ of the center of the circle through the points $O(0,0)$, $A(a,0)$, and $B(b,c)$ below.

\[
  (x_1, y_1) = \answer{ \left( \frac{a}{2}  ,  \frac{b^2+c^2-ab}{2c}   \right)} 
\]

\end{hint}

\end{question}



%This gives
%\[
%        2b (a/2) + 2cy = b^2 + c^2
%\]
%and
%\[
%      y = \frac{b^2+c^2-ab}{2c} .
%\]



%So the center of the circle through the points $O(0,0)$, $A(a,0)$, and $B(b,c)$ has coordinates
%\[
%  (x_1, y_1) = \left(   \frac{a}{2}  ,  \frac{b^2+c^2-ab}{2c}    \right) .
%\]

\begin{exploration}\label{exp:circle7}
You may have noticed that we did not think about whether there is always a {\bf unique} circle through the points $O(0,0)$, $A(a,0)$, and $B(b,c)$. Drag points $A$ and $B$ in the Desmos activity below to explore this question. Try to reconcile what you see here with what the above algebra suggests.
 
\pdfOnly{
Access Desmos interactives through the online version of this text at
 
\href{https://www.desmos.com/calculator/cc7xl7pjad}.
}
 
\begin{onlineOnly}
    \begin{center}
\desmos{cc7xl7pjad}{900}{600}
\end{center}
\end{onlineOnly}
\end{exploration}



\begin{example}  \label{Ex8}
Find an equation of the circle through the points $O(0,0)$, $A(a,b)$, and $C(c,d)$.

\begin{exploration}\label{exp:circle8}
(a) Correct the expressions for the coordinates $(x_1,y_1)$ of the circle's center in Lines 13 and 14 of the Desmos Activity below.

(b) Correct the equation for the circle in Line 15.

(c) Input equations for the 3 sets of points equidistant from the three pairs of points $O$, $A$, and $B$ in Lines 16-18.

(d)  Drag points $A$ and $B$ in the Desmos activity below to explore when there is a unique circle through $O$, $A$, and $B$. Try to reconcile what you see here with what the above algebra suggests.

 
\pdfOnly{
Access Desmos interactives through the online version of this text at
 
\href{https://www.desmos.com/calculator/cpwwnom2sw}.
}
 
\begin{onlineOnly}
    \begin{center}
\desmos{cpwwnom2sw}{900}{600}
\end{center}
\end{onlineOnly}
\end{exploration}

\end{example}


\begin{question}  \label{Q:Erer3r3r3r}
Find an equation of the circle through the points $A(1,1)$, $B(4,3)$, and $C(8,0)$. Do this as follows.

(a) First forget about point $C$ and think about the family of circles through $A$ and $B$. Complete the following sentence in everday English:

\pskip

\emph{A point $P$ is the center of a circle through $A$ and $B$ if and only if ...}

\pskip

(b) Suppose now $P$ has coordinates $(x,y)$ and translate your sentence from part (b) into an equation. Then simplify this equation.

(c) Graph the center set from part (b) in the desmos worksheet below. Does your center set look correct?

\begin{onlineOnly}
    \begin{center}
\desmos{oatvcruval}{900}{600}
\end{center}
\end{onlineOnly}


\href{https://www.desmos.com/calculator/oatvcruval}{141: Circle Thru Three Points}


(d) Repeat parts (a)-(c) for another pair of points.

(e) Then use algebra to determine the center of the circle through points $A$, $B$, and $C$. Plot the center in the desmos worksheet.

(f) Finally find an equation of the circle and enter it in the worksheet above to see if you are correct.
\end{question}


\section*{Misplaced Questions}  
\begin{question}  \label{Q:POer3r3r3r3r}
(a) Find the coordinates of all  points on the $x$-axis twice as far from the point $A(6,0)$ as from the origin $O(0,0)$.

(b) Find the coordinates of all  points on the $y$-axis twice as far from the point $A(6,0)$ as from the origin.

(c) Find an equation of the set of points twice as far from the point $A(6,0)$ as from the origin.

(d) Describe the set of points in part (c) geometrically.
\end{question}



\begin{question}   \label{Q:3435r55533}
This question belongs in the section \emph{Tangent Circles}.

Use algebra to find the coordinates of the point on the circle
\[
     x^2 + 8x + y^2 - 6y = 0
\]
closest to the point $A$ with coordinates $(4,1)$. Find also the coordinates of the point on the circle farthest from $A$.
\end{question}



\section*{Circles in a Corner}
\begin{question}  \label{Q3442dsfdsf55t}
(a) How many circles of radius $4$ are tangent to the coordinate axes? 

(b) Find equations of two such circles.

(c) Find an equation of the circle of radius $r > 0$ in the first quadrant tangent to the coordinate axes. Enter this equation in Line 3 of the worksheet below. Check that you are correct by dragging the slider $r$.

\begin{exploration}
\begin{onlineOnly}
    \begin{center}
\desmos{3prz8y0fja}{900}{600}
\end{center}
\end{onlineOnly}

(d) Drag the slider $r$ above to determine how many circles tangent to the coordinate axes pass through the point $(5,3)$. 

(e) Use algebra to find equations of all the circle(s) in part (d). Enter these equations on Lines 4 and 5 of the desmos worksheet.



\end{exploration}

\end{question}




\section*{Surprising Circles}

\begin{exercise}  
  Is the point $(12,0)$ twice as far from $(8,0)$ as it is from $(20,0)$? 
  \begin{multipleChoice}  
    \choice{Yes}  
    \choice[correct]{No}  
        \end{multipleChoice}  
\end{exercise}  



\begin{example}
(a) Is the point $Q(5,0)$ twice as far from $E(0,0)$ as it is from $F(15,0)$.

(b) Find the coordinates of all points on the $y$-axis that are three times as far from the point $A(3,-2)$ as from the point $B(1,6)$. %Check your answer(s) by computing the appropriate distances.
\end{example}

\begin{explanation}
(a) This is not really a math question, but an English question. Since
\[
    \text{dist}(Q,E) = 5
\]
and 
\[
    \text{dist}(Q,F) = 10  %= 2 \, \text{dist}(Q,E),
\]
$Q(5,0)$ is twice as far from $F(15,0)$ as it is from $E(0,0)$.

\pskip

{\bf Conclusion:} No, $Q(5,0)$ is {\bf not} twice as far from $E(0,0)$ as it is from $F(15,0)$. It's the other way around; $Q(5,0)$ is twice as far from $F(15,0)$ as it is from $E(0,0)$. 

\pskip

(b) Let $P$ be a point on the $y$-axis with coordinates $(0,y)$ that is three times as  from the point $A(3,-2)$ as it is from the point $B(1,6)$. Then
\[
    \text{dist}(A,P) = 3 \, \text{dist}(B,P) .
\]
Using the Pythagorean Theorem (or distance formula)
\[
    \text{dist}(A,P) = \sqrt{(0-3)^2 + (y+2)^2} = \sqrt{y^2 + 4y + 13}
\]
and
\[
    \text{dist}(B,P) = \sqrt{(0-1)^2 + (y-6)^2} = \sqrt{y^2-12y+37}.
\]
So $P(0,y)$ is  three times as  from $A(3,-2)$ as it is from $B(1,6)$ when
\[
   \sqrt{y^2 + 4y + 13} = 3 \sqrt{y^2-12y+37} .
\]
Squaring both sides gives
\[
  y^2 + 4y + 13 = 9(y^2-12y+37)
\]
and 
\[
        8y^2 - 112y +320 = 0 .
\]
So
\[
     y^2 - 14y + 40 = 0
\]
and
\[
     (y -4 )(y - 10) = 0.
\]
So
\[
     y-4 = 0    \;\;\;  \text{or} \;\;\; y-10=0
\]
and
\[
    y=4  \;\; \; \text{or} \;\;\ y=10 .
\]

\pskip

{\bf Conclusion:} There are two points on the $y$-axis twice as far from $A(3,-2)$ as from $B(1,6)$. They have coordinates $(0,4)$ or $(0,10)$.

\end{explanation}


\begin{exercise}  
  Which of the following points are twice as far from $(0,0)$ as from $(6,0)$? Check all that apply.  
  \begin{selectAll}  
    \choice{$(2,0)$}  
    \choice[correct]{$(4,0)$}  
    \choice[correct]{$(12,0)$} 
    \choice{$(-6,0)$} 
     \end{selectAll}  
\end{exercise}  



\begin{question} 
Find an equation for the set of points twice as far from $(0,0)$ as from $(6,0)$. 
\[
  \answer{\sqrt{x^2+y^2} = 2 \sqrt{(x-6)^2+y^2}} 
\]
    \end{question}




\begin{example}   \label{E:348543534}
Show that the curve with equation
\[
   x^2  - 10x + y^2 + 7y = 5
\]
is a circle. Then determine the coordinates of its center and its radius.
\end{example}


\begin{explanation}
We need to complete the square twice. The expression
\[
     x^2 - 10x
\]
is not a perfect square. But by adding the constant $(-10/2)^2 = (-5)^2 = 25$, we get a new expression that is a perfect square, namely,
\[
     x^2 -10x + (-5)^2 = (x-5)^2 .
\]
We can check this by distributing:
\[
    (x-5)^2 = (x-5)(x-5) = x^2 -5x -5x + 25 = x^2 - 10x + 25 .
\]

Similarly, the expression
\[
   y^2 + 7y
\]
is also not a perfect square. But by adding the constant $(7/2)^2 = 49/4$, we get a new expression that is a perfect square, namely
\[
     y^2 + 7y + (7/2)^2 = (y+7/2)^2 .
\]

Now let's write the equation of the curve as
\[
     (x^2  - 10x + \;\;\;\; ) + (y^2 + 7y + \;\;\;\; ) = 5 .
\]
We'll add $(-5)^2$ and $(7/2)^2$ to both sides, getting
\[
     (x^2  - 10x + (-5)^2 ) + (y^2 + 7y + (7/2)^2 ) = 5 + (-5)^2 + (7/2)^2 ,
\]
or
\[
    (x-5)^2 + (y+7/2)^2 = 5 + 25 + \frac{49}{4} = \frac{169}{4} .
\]
Taking the square root of both sides gives
\[
    \sqrt{(x-5)^2 + (y+7/2)^2} = 13/2 .
\]
The expression $\sqrt{(x-5)^2 + (y+7/2)^2}$ measures the distance from the point $P(x,y)$ to the point $A(5,-7/2)$. Since this distance is equal to the constant $13/2$, the graph of all points $P(x,y)$ whose coordinates satisfy the above equation is a circle with center $A(5,-7/2)$ and radius $13/2$.

\pskip 

{\bf Important Point:} Stick with fractions when completing the square. Do not use decimals.

\end{explanation}


\section*{Chapter Quiz}

\begin{itemize}

\item{Do this quiz without using any sources. Nothing. Not to check grammar, not to look up information. Nothing. Just do the best you can with what you know.}

\item{It's ok to use a calculator for arithmetic.}

\item{Use only the material from the \emph{Introduction} and the chapter \emph{Circles} from our class notes.}

\item{Include brief explanations for your solution to each problem.}

\end{itemize}


\pskip

\begin{question}  \label{Q:9444433331350}
Imagine a railroad track $L$ feet long secured only at its two ends to the desert floor. Suppose on a hot day the track expands one foot and buckles into a curve. We would like to compute, or at least approximate, the height of the track above the ground at its midpoint. We really need calculus to determine the shape of the curve and the exact height, but we can get a pretty good approximation by supposing the track bends into two straight segments running from the high point to its endpoints.

Make the assumption above and find an expression for the height of the track above its midpoint. Simplify your expression as much as possible. 

Explain your reasoning behind each solution. Define, with units, any variables you introduce. Include a picture to help with your explanation.

\end{question}


\begin{question}  \label{Q8dsf8r3tg;lyhg}
(a) Find an equation of the circle centered at the point $(-3,2)$ that passes through the point $(-7,-3)$. 

(b) Explain the \emph{logic} behind your equation. 
\end{question}


%\begin{question}  \label{Q:009357865}
%Find equations of two circles with radius $4$ that are tangent to both coordinate axes. Draw a picture to help with your explanation.
%\end{question}

\begin{question}  \label{Q:df454tt4443}
Find the coordinates of the point on the circle centered at the origin with radius $6$ that is farthest from the point $(3,-5)$. Draw a picture to help with your explanation.

\begin{hint}
The point on the circle farthest from the point $A(3,-5)$ lies on the line through the origin and $A$. Start by finding an equation of this line in the form $y=f(x)$. Then substitute $f(x)$ for $y$ in the equation of the circle and solve for $x$. Keep going...
\end{hint}

\end{question}

\begin{question} \label{Qsdfl4345r3}
Find an equation of the circle through the points $(0,0)$, $(6,4)$, and $(-2,-4)$. Explain your reasoning. Draw a picture to help with your explanation.

\end{question}





\end{document}














