\documentclass{ximera}
\title{Introduction}

\newcommand{\pskip}{\vskip 0.1 in}

\begin{document}
\begin{abstract}
Placement check. 
\end{abstract}
\maketitle


Answer the following questions to the best of your ability {\bf without using any sources}. {\bf No calculator, no websites, no AI, no friend, nothing. Just your own ideas}. Show all work. %Omit those questions you cannot solve. 

The purpose of this assignment is to check if you have placed into the correct course. 

\pskip

\begin{question}  \label{Q9df934gmmn}
Solve each equation or fully simplify each expression. Show all work. Do {\bf not} use the quadratic formula. You can check your answers below.

\pskip

(a) $ 5 - 3(7)$.  

\[
       5 - 3(7) =  \answer{-16}
\]

(b) $6 - \frac{13}{37} \left(  w - 6887 \right) = 45$

The solution set to the equation is
\[
    \{w | w= \answer{6776} \} .
\]

(c) $\frac{4}{11} - \frac{8}{9}$

\[
    \frac{4}{11} - \frac{8}{9} = -\frac{52}{99} 
\]

(d) $341 - 3 (x+542)^2 =121$

The solution set to the equation is
\[
       \{x | x= \answer{-542 \pm \sqrt{40}} \}.
\]

(e) $7s^2 = 45s $

The solution set to the equation is
\[
       \{s | s= \answer{0, 45/7} \}.
\]

(f) $2^{-3} + 36^{1/2}$

\[
     2^{-3} + 36^{-1/2} = \answer{\frac{7}{24}}
\]

(g) $\log_4 64 - \log_3 (1/9)$

\[
   \log_4 64 - \log_3 (1/9) = \answer{5}
\]


\end{question}

\begin{question}  \label{Q:9sdf85r3}
Let 
\[
   f(x) = 6 - 3(x-153) .
\]

(a) Evaluate $f(150)$.

(b) Solve the equation $f(x)=150$.

(c) Solve the equation
\[
      f(x) = f(150)-10 .
\]
\end{question}  

\begin{question} \label{Qpdf0gbvgbrtg}
Let
\[
      f(x) = 4 - x - x^2 .
\]
(a) Evaluate $f(-4)$.

(b) Solve the equation 
\[
      f(x) = -1 .
\]

(c) Solve the equation
\[
       f(x) = f(17/23) .
\]
\end{question}


\begin{question}  \label{Qdfdcg4tythyh5t}
Let 
\[
       f(x) = 42 - \frac{9}{13}(x - 140) .
\]
Solve the inequality
\[
     f(x) > f(150) .
\]
\end{question}


\begin{question}  \label{Eer5htrree}
Explain what it means for a population to grow exponentially.
\end{question}

\begin{question}
The function
\[
  P = f(t) = 20 (3)^{t/5} \, , \, -1 \leq t \leq 12 ,
\]
expresses the population (in millions) of a colony of bacteria in terms of the number of hours past noon. 

When are there are 80 million bacteria?
\end{question}


\begin{question}  \label{Q99834322}
Between speeds of $20$ miles/hour and $58$ miles/hour the gas mileage of a car is a linear function of its speed. Suppose that the car gets $18$ miles/gallon at a speed of $20$ miles/hour and $35$ miles/gallon at a speed of $58$ miles/hour.

(a) Find a function 
\[
       S = f(G)
\]
that expresses the speed of the car (in miles/hr) in terms of its gas mileage (in miles/gal).

(b) Find a function 
\[
   R = g(S)
\]
that expresses the rate (in gal/hour) at which the car burns gas in terms of its speed (in miles/hour).

\end{question}


\begin{question}  \label{QDFdf4444}
The function
\[
      T = f(m) = 20 + 70 (2^{-m/10}) \, , \, 0 \leq t \leq 30 ,
\]
expresses the temperature (in Celsius degrees) of a cup of coffee in terms of the number of minutes past noon.

(a) Find the domain and range of $f$.

(b) Find an expression for the function $g$ that takes as an input the time (in terms of the number of minutes past noon) and returns as an output the temperature (in Celsius degrees) of the coffee at that time.

(c) Find the domain and range of $g$.

\end{question}

\begin{question}  \label{Q87d6fgte}
Between noon and 11pm, the population of a colony of bacteria decreases exponentially. The population is 17 million at noon and 9 million at 4pm.

Find a function
\[
        P = f(t) \, , \, 0\leq t \leq 11 ,
\]
that expresses the population (in millions) in terms of the number of hours past noon.
\end{question}


\begin{question}  \label{Q98d33dg5r5r3}
Between 1pm and 3pm a balloon descends at a constant rate. The balloon is $5341$ feet high at 1:30pm and $2001$ feet high at 2:15 pm.

Find a function
\[
      h = g(t) \, , \, 1\leq t \leq 3
\]
that expresses the altitude of the balloon (in feet) in terms of the number of hours past noon.
\end{question}

\end{document}