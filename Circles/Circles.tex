\documentclass{ximera}
\title{Circles}


\newcommand{\pskip}{\vskip 0.1 in}

\begin{document}
\begin{abstract}
Circles in the Euclidean plane.
\end{abstract}
\maketitle


\pskip

A circle is the set of points in a plane that are a given distance from a given point. The given point is the center of the circle and the fixed distance is the \emph{radius}. Any segment from the center to a point on the circle is also called a radius. The context should make it clear which meaning applies.


\begin{example}
Use the Pythagorean Theorem to determine if the point $P(10,5)$ lies inside, outside, or on the circle with center $A(2,1)$ and radius $9$. Explain your reasoning. Be sure to include an accurate picture.
\end{example}

\begin{explanation}
The circle is the set of points exactly 9 units from the center $A(2,1)$. The point $P$ will lie inside, outside, or on the circle depending respectively on whether the distance between $A$ and $P$ is less than, equal to, or greater than the radius.

To determine the distance between $A$ and $P$, we use the Pythagorean theorem by drawing a right triangle $\Delta ABP$ with hypotenuse $AP$ and legs parallel to the coordinate axes as in the Figure. The point $B$ has coordinates $(10,1)$, taken from the $x$-coordinate of $P$ (ie. $x=10$) and the $y$-coordinate of $A$ (ie. $y=1$). Then by the Pythagorean Theorem, the distance between $A$ and $P$ is
\[
   \text{dist}(A,P) = \sqrt{AB^2 + BP^2} = \sqrt{(10-2)^2 + (5-1)^2} = \sqrt{8^2 + 4^2} = \sqrt{80} . 
\]
Since this distance is less than the radius of the circle $9 = \sqrt{81}$, the point $P$ lies \emph{inside} the circle.

\iffalse   %%%%%%%%%%%%%%%%%%%%%%%%%%%%%

\begin{figure}[!h]
\centerline{
\includegraphics[height=2 in]{Circle1.eps}   
}
%\caption{The graph of the function $y=f(x)$.}\label{Fi:Shadow}
\end{figure}

\fi   %%%%%%%%%%%%%%%%%%%%%%%%%%%%%%%%%%%%%


\end{explanation}


\begin{example}
Use the distance formula to find an equation of the circle with center $A(2,1)$ and radius $9$. Explain your reasoning. Be sure to include an accurate picture.
\end{example}

\begin{explanation}
The circle is the set of points exactly 9 units from the point $A$. So the point $P(x,y)$ lies on the circle when
\[
     \text{dist}(A,P) = 9 .
\]
By the distance formula (really the Pythagorean Theorem), 
\[
 \text{dist}(A,P) =  \sqrt{(x-2)^2 + (y-1)^2} .
\]
So the point $P(x,y)$ lies on the circle with center $A(1,2)$ and radius $9$ when
\[
     \sqrt{(x-2)^2 + (y-1)^2} = 9 .
\]
This is an equation of the circle with center $A(1,2)$ and radius $9$. Stop here. There is no need to square both sides.

\iffalse   %%%%%%%%%%%%%%%%%%%%%%%%%%%%%

\begin{figure}[!h]
\centerline{
\includegraphics[height=2 in]{Circle1B.eps}   
}
%\caption{The graph of the function $y=f(x)$.}\label{Fi:Shadow}
\end{figure}

\fi

\end{explanation}


\begin{example}
Find the coordinates of the $x$-intercepts of the circle with center $A(2,1)$ and radius $9$. Explain your reasoning. Be sure to include an accurate picture.
\end{example}

\pskip

\begin{explanation}
A point lines on the $x$-axis when its $y$-coordinate is zero. So we can find the $x$-coordinates of the intercepts by substituting $y=0$ into the equation
\[
\sqrt{(x-2)^2 + (y-1)^2} = 9 
\]
of the circle and solving for $x$. So if $y=0$, then
\[
\sqrt{(x-2)^2 + (0-1)^2} = 9,
\]
and
\[
   (x-2)^2 + 1 = 81,
\]
and
\[
   (x-2)^2 = 80.
\]
So
\[
    x-2 = \pm \sqrt{80}
\]
and 
\[
  x = 2\pm \sqrt{80} .
\]

The $x$-intercepts of the circle with center $A(1,2)$ and radius $9$ have coordinates $(2\pm \sqrt{80}, 0)$, or approximately $(10.9,0)$ or $(-6.9,0)$.

\iffalse   %%%%%%%%%%%%%%%%%%%%%%%%%%%%%%%%%

\begin{figure}[!h]
\centerline{
\includegraphics[height=2.5 in]{Circle1C.eps}   
}
%\caption{The graph of the function $y=f(x)$.}\label{Fi:Shadow}
\end{figure}

\fi  %%%%%%%%%%%%%%%%%%%%%%%%%%%%%%%


\end{explanation}



\begin{example}
Find the coordinates of the points where the line $x=-5$ intersects the circle with center $A(2,1)$ passing through the point $Q(10,5)$. Explain your reasoning. Be sure to include an accurate picture.
\end{example}

\begin{explanation}
The circle has radius
\[
   \text{dist}(A,Q) = \sqrt{(10-2)^2  + (5-1)^2} = \sqrt{80} .
\]
So the point $P(x,y)$ lies on the circle when
\[
  \text{dist}(A,P) = \sqrt{80}
\]
or when
\[
   \sqrt{(x-2)^2 +(y-1)^2} = \sqrt{80} .
\]
This is an equation of the circle with center $A(2,1)$ that passes though the point $Q(10,5)$.

To find the coordinates of the points where this circle intersects the line $x=-5$, substitute $x=-5$ into the equation of the circle. Then 
\[
       (-5-2)^2 + (y-1)^2 = 80
\]
and
\[ 
     (y-1)^2 = 31 .
\]
So
\[
     y = 1 \pm \sqrt{31} .
\]
The points where the line $x=-5$ intersects the circle with center $A(2,1)$ that passes through the point $Q(10,5)$ have coordinates $(-5, 1\pm \sqrt{31})$, or approximately $(-5, 6.6)$ or $(-5,-4.6)$.


\iffalse  %%%%%%%%%%%%%%%%%%%%%%%%

\begin{figure}[!h]
\centerline{
\includegraphics[height=1.7 in]{Circle1D.eps}   
}
%\caption{The graph of the function $y=f(x)$.}\label{Fi:Shadow}
\end{figure}

\fi          %%%%%%%%%%%%%%%%%%%%%%%%%%%%%%%%%%

\end{explanation}



\begin{exercise}  
  Which of the following points are twice as far from $(0,0)$ as from $(6,0)$? Check all that apply.  
  \begin{selectAll}  
    \choice{$(2,0)$}  
    \choice[correct]{$(4,0)$}  
    \choice[correct]{$(12,0)$} 
    \choice{$(-6,0)$} 
     \end{selectAll}  
\end{exercise}  


\begin{question} 
Find an equation for the set of points twice as far from $(0,0)$ as from $(6,0)$. 
\[
  \answer{\sqrt{x^2+y^2} = 2 \sqrt{(x-6)^2+y^2}} 
\]
    \end{question}


\begin{exploration}\label{exp:pc1}
The graphs of the beetle's coordinate functions $x=f(t)$ and $y=g(t)$, $-2\leq t \leq 20$, are shown below. Note the units and variable names on the axes. Drag the slider $s$ to see the coordinates of the beetle at different times during its journey.
 
\pdfOnly{
Access Desmos interactives through the online version of this text at
 
\href{https://www.desmos.com/calculator/0iruo192ir}.
}
 
\begin{onlineOnly}
    \begin{center}
\desmos{0iruo192ir}{900}{600}
\end{center}
\end{onlineOnly}
\end{exploration}




\begin{example}
Find the coordinates of all points on the $x$-axis that are equidistant from the points $A(-2,3)$ and $B(6,1)$. Check your answer(s) by computing the appropriate distances.
\end{example}

\begin{explanation}
Let $P$ be a point on the $x$-axis with coordinates $(x,0)$ that is equidistant from $A(-2,3)$ and $B(6,1)$. Then
\[
    \text{dist}(A,P) = \text{dist}(B,P) .
\]
Using the Pythagorean Theorem (or distance formula)
\[
    \text{dist}(A,P) = \sqrt{(x+2)^2 + (0-3)^2}
\]
and
\[
    \text{dist}(B,P) = \sqrt{(x-6)^2 + (0-1)^2} .
\]

Then $P(x,0)$ is equidistant from $A$ and $B$ when
\[
  \sqrt{(x+2)^2 + (0-3)^2} = \sqrt{(x-6)^2 + (0-1)^2} .
\]
Squaring both sides gives
\[
   (x+2)^2 +9 = (x-6)^2 + 1 .
\]
So $P$ is equidistant from $A$ and $B$ when
\[
    x^2 + 4x + 13 = x^2 - 12x + 37, 
\]
or when
\[
     16x = 24
\]
and 
\[
   x=3/2 .
\]

{\bf Conclusion:} There is one point on the $x$-axis that is equidistant from the points $A(-2,3)$ and $B(6,1)$. It has coordinates $(3/2,0)$. 

To check this let's compute the two distances:

\pskip

\begin{align*}
    \text{dist}(A,P)  &= \sqrt{(3/2+2)^2 + (0-3)^2} \\
                             &= \sqrt{(7/2)^2 + 9} \\
                             &= \sqrt{85/4}
\end{align*}

and 

\begin{align*}
    \text{dist}(B,P)  &= \sqrt{(3/2-6)^2 + (0-1)^2} \\
                             &= \sqrt{(9/2)^2 + 1} \\
                             &= \sqrt{85/4} .
\end{align*}

They are indeed equal.


\end{explanation}



\begin{example}
(a) Is the point $Q(5,0)$ twice as far from $E(0,0)$ as it is from $F(15,0)$.

(b) Find the coordinates of all points on the $y$-axis that are three times as far from the point $A(3,-2)$ as from the point $B(1,6)$. %Check your answer(s) by computing the appropriate distances.
\end{example}

\begin{explanation}
(a) This is not really a math question, but an English question. Since
\[
    \text{dist}(Q,E) = 5
\]
and 
\[
    \text{dist}(Q,F) = 10  %= 2 \, \text{dist}(Q,E),
\]
$Q(5,0)$ is twice as far from $F(15,0)$ as it is from $E(0,0)$.

\pskip

{\bf Conclusion:} No, $Q(5,0)$ is {\bf not} twice as far from $E(0,0)$ as it is from $F(15,0)$. It's the other way around; $Q(5,0)$ is twice as far from $F(15,0)$ as it is from $E(0,0)$. 

\pskip

(b) Let $P$ be a point on the $y$-axis with coordinates $(0,y)$ that is three times as  from the point $A(3,-2)$ as it is from the point $B(1,6)$. Then
\[
    \text{dist}(A,P) = 3 \, \text{dist}(B,P) .
\]
Using the Pythagorean Theorem (or distance formula)
\[
    \text{dist}(A,P) = \sqrt{(0-3)^2 + (y+2)^2} = \sqrt{y^2 + 4y + 13}
\]
and
\[
    \text{dist}(B,P) = \sqrt{(0-1)^2 + (y-6)^2} = \sqrt{y^2-12y+37}.
\]
So $P(0,y)$ is  three times as  from $A(3,-2)$ as it is from $B(1,6)$ when
\[
   \sqrt{y^2 + 4y + 13} = 3 \sqrt{y^2-12y+37} .
\]
Squaring both sides gives
\[
  y^2 + 4y + 13 = 9(y^2-12y+37)
\]
and 
\[
        8y^2 - 112y +320 = 0 .
\]
So
\[
     y^2 - 14y + 40 = 0
\]
and
\[
     (y -4 )(y - 10) = 0.
\]
So
\[
     y-4 = 0    \;\;\;  \text{or} \;\;\; y-10=0
\]
and
\[
    y=4  \;\; \; \text{or} \;\;\ y=10 .
\]

\pskip

{\bf Conclusion:} There are two points on the $y$-axis twice as far from $A(3,-2)$ as from $B(1,6)$. They have coordinates $(0,4)$ or $(0,10)$.

\end{explanation}


\begin{example}
Show that the curve with equation
\[
   x^2  - 10x + y^2 + 7y = 5
\]
is a circle. Then determine the coordinates of its center and its radius.
\end{example}


\begin{explanation}
We need to complete the square twice. The expression
\[
     x^2 - 10x
\]
is not a perfect square. But by adding the constant $(-10/2)^2 = (-5)^2 = 25$, we get a new expression that is a perfect square, namely,
\[
     x^2 -10x + (-5)^2 = (x-5)^2 .
\]
We can check this by distributing:
\[
    (x-5)^2 = (x-5)(x-5) = x^2 -5x -5x + 25 = x^2 - 10x + 25 .
\]

Similarly, the expression
\[
   y^2 + 7y
\]
is also not a perfect square. But by adding the constant $(7/2)^2 = 49/4$, we get a new expression that is a perfect square, namely
\[
     y^2 + 7y + (7/2)^2 = (y+7/2)^2 .
\]

Now let's write the equation of the curve as
\[
     (x^2  - 10x + \;\;\;\; ) + (y^2 + 7y + \;\;\;\; ) = 5 .
\]
We'll add $(-5)^2$ and $(7/2)^2$ to both sides, getting
\[
     (x^2  - 10x + (-5)^2 ) + (y^2 + 7y + (7/2)^2 ) = 5 + (-5)^2 + (7/2)^2 ,
\]
or
\[
    (x-5)^2 + (y+7/2)^2 = 5 + 25 + \frac{49}{4} = \frac{169}{4} .
\]
Taking the square root of both sides gives
\[
    \sqrt{(x-5)^2 + (y+7/2)^2} = 13/2 .
\]
The expression $\sqrt{(x-5)^2 + (y+7/2)^2}$ measures the distance from the point $P(x,y)$ to the point $A(5,-7/2)$. Since this distance is equal to the constant $13/2$, the graph of all points $P(x,y)$ whose coordinates satisfy the above equation is a circle with center $A(5,-7/2)$ and radius $13/2$.

\pskip 

{\bf Important Point:} Stick with fractions when completing the square. Do not use decimals.

\end{explanation}

\end{document}














