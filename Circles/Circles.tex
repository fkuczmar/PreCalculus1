\documentclass{ximera}
\title{Circles}


\newcommand{\pskip}{\vskip 0.1 in}

\begin{document}
\begin{abstract}
Circles in the Euclidean plane.
\end{abstract}
\maketitle


\pskip




\begin{example}
Use the Pythagorean Theorem to determine if the point $P(10,5)$ lies inside, outside, or on the circle with center $A(2,1)$ and radius $9$. Explain your reasoning. Be sure to include an accurate picture.
\end{example}

\begin{explanation}
The circle is the set of points exactly 9 units from the center $A(2,1)$. The point $P$ will lie inside, outside, or on the circle depending respectively on whether the distance between $A$ and $P$ is less than, equal to, or greater than the radius.

To determine the distance between $A$ and $P$, we use the Pythagorean theorem by drawing a right triangle $\Delta ABP$ with hypotenuse $AP$ and legs parallel to the coordinate axes as in the Figure. The point $B$ has coordinates $(10,1)$, taken from the $x$-coordinate of $P$ (ie. $x=10$) and the $y$-coordinate of $A$ (ie. $y=1$). Then by the Pythagorean Theorem, the distance between $A$ and $P$ is
\[
   \text{dist}(A,P) = \sqrt{AB^2 + BP^2} = \sqrt{(10-2)^2 + (5-1)^2} = \sqrt{8^2 + 4^2} = \sqrt{80} . 
\]
Since this distance is less than the radius of the circle $9 = \sqrt{81}$, the point $P$ lies \emph{inside} the circle.

\iffalse   %%%%%%%%%%%%%%%%%%%%%%%%%%%%%

\begin{figure}[!h]
\centerline{
\includegraphics[height=2 in]{Circle1.eps}   
}
%\caption{The graph of the function $y=f(x)$.}\label{Fi:Shadow}
\end{figure}

\fi   %%%%%%%%%%%%%%%%%%%%%%%%%%%%%%%%%%%%%


\end{explanation}


\begin{example}
Use the distance formula to find an equation of the circle with center $A(2,1)$ and radius $9$. Explain your reasoning. Be sure to include an accurate picture.
\end{example}

\begin{explanation}
The circle is the set of points exactly 9 units from the point $A$. So the point $P(x,y)$ lies on the circle when
\[
     \text{dist}(A,P) = 9 .
\]
By the distance formula (really the Pythagorean Theorem), 
\[
 \text{dist}(A,P) =  \sqrt{(x-2)^2 + (y-1)^2} .
\]
So the point $P(x,y)$ lies on the circle with center $A(1,2)$ and radius $9$ when
\[
     \sqrt{(x-2)^2 + (y-1)^2} = 9 .
\]
This is an equation of the circle with center $A(1,2)$ and radius $9$. Stop here. There is no need to square both sides.

\iffalse   %%%%%%%%%%%%%%%%%%%%%%%%%%%%%

\begin{figure}[!h]
\centerline{
\includegraphics[height=2 in]{Circle1B.eps}   
}
%\caption{The graph of the function $y=f(x)$.}\label{Fi:Shadow}
\end{figure}

\fi

\end{explanation}


\begin{example}
Find the coordinates of the $x$-intercepts of the circle with center $A(2,1)$ and radius $9$. Explain your reasoning. Be sure to include an accurate picture.
\end{example}

\pskip

\begin{explanation}
A point lines on the $x$-axis when its $y$-coordinate is zero. So we can find the $x$-coordinates of the intercepts by substituting $y=0$ into the equation
\[
\sqrt{(x-2)^2 + (y-1)^2} = 9 
\]
of the circle and solving for $x$. So if $y=0$, then
\[
\sqrt{(x-2)^2 + (0-1)^2} = 9,
\]
and
\[
   (x-2)^2 + 1 = 81,
\]
and
\[
   (x-2)^2 = 80.
\]
So
\[
    x-2 = \pm \sqrt{80}
\]
and 
\[
  x = 2\pm \sqrt{80} .
\]

The $x$-intercepts of the circle with center $A(1,2)$ and radius $9$ have coordinates $(2\pm \sqrt{80}, 0)$, or approximately $(10.9,0)$ or $(-6.9,0)$.

\iffalse   %%%%%%%%%%%%%%%%%%%%%%%%%%%%%%%%%

\begin{figure}[!h]
\centerline{
\includegraphics[height=2.5 in]{Circle1C.eps}   
}
%\caption{The graph of the function $y=f(x)$.}\label{Fi:Shadow}
\end{figure}

\fi  %%%%%%%%%%%%%%%%%%%%%%%%%%%%%%%


\end{explanation}



\begin{example}
Find the coordinates of the points where the line $x=-5$ intersects the circle with center $A(2,1)$ passing through the point $Q(10,5)$. Explain your reasoning. Be sure to include an accurate picture.
\end{example}

\begin{explanation}
The circle has radius
\[
   \text{dist}(A,Q) = \sqrt{(10-2)^2  + (5-1)^2} = \sqrt{80} .
\]
So the point $P(x,y)$ lies on the circle when
\[
  \text{dist}(A,P) = \sqrt{80}
\]
or when
\[
   \sqrt{(x-2)^2 +(y-1)^2} = \sqrt{80} .
\]
This is an equation of the circle with center $A(2,1)$ that passes though the point $Q(10,5)$.

To find the coordinates of the points where this circle intersects the line $x=-5$, substitute $x=-5$ into the equation of the circle. Then 
\[
       (-5-2)^2 + (y-1)^2 = 80
\]
and
\[ 
     (y-1)^2 = 31 .
\]
So
\[
     y = 1 \pm \sqrt{31} .
\]
The points where the line $x=-5$ intersects the circle with center $A(2,1)$ that passes through the point $Q(10,5)$ have coordinates $(-5, 1\pm \sqrt{31})$, or approximately $(-5, 6.6)$ or $(-5,-4.6)$.


\iffalse  %%%%%%%%%%%%%%%%%%%%%%%%

\begin{figure}[!h]
\centerline{
\includegraphics[height=1.7 in]{Circle1D.eps}   
}
%\caption{The graph of the function $y=f(x)$.}\label{Fi:Shadow}
\end{figure}

\fi          %%%%%%%%%%%%%%%%%%%%%%%%%%%%%%%%%%

\end{explanation}



\begin{exercise}  
  Which of the following points are twice as far from $(0,0)$ as from $(6,0)$? Check all that apply.  
  \begin{multipleChoice}  
    \choice{$(2,0)$}  
    \choice[correct]{$(4,0)$}  
    \choice{$(12,0)$} 
    \choice[correct]{$(-6,0$} 
     \end{multipleChoice}  
\end{exercise}  


\begin{question} 
Find an equation for the set of points twice as far from $(0,0)$ as from $(6,0)$. 
\[
  \answer{\sqrt{x^2+y^2} = 2 \sqrt{(x-6)^2+y^2}} 
\]
    \end{question}


\begin{exploration}\label{exp:pc1}
The graphs of the beetle's coordinate functions $x=f(t)$ and $y=g(t)$, $-2\leq t \leq 20$, are shown below. Note the units and variable names on the axes. Drag the slider $s$ to see the coordinates of the beetle at different times during its journey.
 
\pdfOnly{
Access Desmos interactives through the online version of this text at
 
\href{https://www.desmos.com/calculator/0iruo192ir}.
}
 
\begin{onlineOnly}
    \begin{center}
\desmos{0iruo192ir}{900}{600}
\end{center}
\end{onlineOnly}
\end{exploration}


\end{document}














