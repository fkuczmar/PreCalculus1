\documentclass{ximera}
\title{Exponential Growth CW1}

\newcommand{\pskip}{\vskip 0.1 in}

\begin{document}
\begin{abstract}
Exponential functions and their inverses.
\end{abstract}
\maketitle


\section{Growth Factors}
A population that grows \emph{linearly} increases at a constant absolute rate, perhaps at the rate of $20,000$ people/yr. 

A population that grows \emph{exponentially} increases at a constant \emph{relative} rate, perhaps at the rate of $20\%$/yr. To understand what this means, suppose the population was $P$ at the start of the year 2020. Then one year later the population would be $20\%$ greater, or 
\[
   P + 20\% P  = P + 0.2 P = 1.2P .
\]
We call 1.2 the \emph{annual growth factor}. To get the population one year from, we just multiply the current population by this factor of 1.2

\begin{question}  \label{Q00:EwonentialG}
A population grows exponentially, increasing by $25\%$ each year. Suppose the current population is $P$.

(a) What will the population be one year from now?
\[
\answer{1.25P}
\]
 
(b) What was the population one year ago?
\[
\answer{0.8P}
\]

(c) What will the population be three years from now?
\[
\answer{1.953125P}
\]

(d) What is the three-year growth factor?
\[
\answer{1.953125}
\]

(e) What is the relative change in the population over a three-year period?
\[
\text{The population increases by } \answer{95.3125}\% \text{ every three years.}
\]

\end{question}


\section{Point-Slope vs. Point-Growth, Part 1}

\begin{example} \label{E56dw44}
Suppose the population of a (small) colony of bacteria is $40,000$ at 2pm and $50,000$ at 3pm.

\begin{enumerate}

\item Suppose first that between 2pm and 8pm the population is a linear function of time.

\begin{enumerate}
\item What does it mean for the population to be a linear function of time? Explain in everday English. No math terms.

\item Find the population's growth rate.

The population increases at the constant rate of $\answer{10000}$ bac/hour.

\item Let 
\[
      P = f(t) , 2\leq t \leq 8 ,
\]
be a function that expresses the population in terms of the number of hours past noon.

Find an expression for the function $P=f(t)$. Use point-slope!

\end{enumerate}

\item Next suppose that between 2pm and 8pm the population is an exponential function of time.

\begin{enumerate}
\item What does it mean for the population to be an exponential function of time? Explain in everday English. No math terms.

\item Compute the one-hour growth factor. What are its units?%At what rate does the population grow?

\item Interpret the meaning of the one-hour growth factor.

Every hour the population gets multipled by $\answer{1.25}$.

\pskip

Or, every hour the population increases by $\answer{25}\%$.


\item Let 
\[
      P = f(t) , 2\leq t \leq 8 ,
\]
be a function that expresses the population in terms of the number of hours past noon.

\begin{enumerate}
\item Use common sense (do not rely on a formula) to evalute $f(8)$. Show the complete computation. Include units for every number. Explain your reasoning. Click the arrow to the lower right for guidance.
\begin{expandable}
The key idea is to use the one-hour growth factor of $\answer{1.25}$. Because each hour the population gets mutiplied by the factor $\answer{1.25}$ and because the length of the time interval from 2pm to 8pm is
\[
   \Delta t = \answer{8} \text{ hrs} - \answer{2}\text{ hrs} = \answer{6} \text{ hrs} , 
\] 
we need only mutiply the population at 2pm by the factor $\answer{1.25}$ six times. So the population at 8pm is
\[
   f(8) = f(2) (\answer{1.25})^{\answer{6}} = \answer{40000} (\answer{1.25})^{\answer{6}} 
\]
bacteria.
 
\end{expandable}

\item Use the same reasoning to find an expression for the function $P=f(t)$. Explain this reasoning.
\end{enumerate}
\end{enumerate}

\end{enumerate}
\end{example}


\section{Point-Slope vs. Point-Growth, Part 2}

\begin{question}   \label{Q22:sdww}
(a) Suppose that a colony of bacteria has a population of 100,000 at 1pm and a population of 490,000 at 4pm. Find functions 
\[
   P = f(t) \, , 0\leq t \leq 10 ,
\]
and 
\[
   P = g(t) \, , 0\leq t \leq 10 ,
\]
that express the population (measured in thousands of bacteria) in terms of the number of hours past noon assuming the population grows a constant rate (the function $f$) and at a constant relative rate (the function $g$), respectively. Use point-slope for $f$. {\bf Explain the logic behind the expressions for each function.}

Enter these functions in the worksheet below. Compare their behaviors over the given domain.

\pdfOnly{
Access Desmos interactives through the online version of this text at
 
\href{https://www.desmos.com/calculator/bwine7dcmg}.
}
 
\begin{onlineOnly}
    \begin{center}
\desmos{bwine7dcmg}{900}{600}
\end{center}
\end{onlineOnly}

\pskip


\end{question}





\end{document}