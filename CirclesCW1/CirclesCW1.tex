\documentclass{ximera}
\title{Circles, CW1}


\newcommand{\pskip}{\vskip 0.1 in}

\begin{document}
\begin{abstract}
Basic questions about circles.
\end{abstract}
\maketitle


\pskip

\section{Equation of a Circle}

\begin{question} \label{QDEWWD3w339832}

Consider the circle of radius $9$ cm centered at the point $A$ with coordinates $(3,1)$ cm.

\begin{enumerate}
\item Sketch this circle.

\item Describe the circle geometrically. 

\item Use the Pythagorean theorem to determine if the point $Q$ with coordinates $(11,6)$ lies on the circle. Annotate your picture to help with your explanation.

\item Find the coordinates of four points on the circle. Make this easy.

\item Find the exact coordinates of four other points on the circle. Do not use a calculator.

\item Find an equation of the circle. Explain your reasoning. 

Start by saying the point $P$ with coordinates $(x,y)$ (measured in cm) lies on the circle if and only if the distance from $P$ to the center $A$ with coordinates $\answer{(3,1)}$ cm is equal to $\answer{9}$ cm.

So algebraically, the point $P$ with coordinates $(x,y)$ (measured in cm) lies on the circle with center $A(3,1)$ cm and radius $9$ cm if and only if
\[
     \answer{\sqrt{(x-3)^2 + (y-1)^2}}  = 9 .
\] 

\end{enumerate}
\end{question}

\begin{question} \label{Q8F3f33r}
Find an equation of the circle centered at the point $A$ with coordinates $(-2,5)$ that passes through the point $(-4,-7)$. Explain your reasoning. Draw a picture to help with your explanation.
\end{question}


\begin{question}  \label{QKFDfer33}
Find an equation of the smallest circle through the points $A(7,-2)$ and $B(-1,-5)$. Explain your reasoning. 
\end{question}


\section{Closest and Farthest Points on a Circle}

\begin{question}  \label{QKfer4333}
\begin{enumerate}

\item Drag the slider $u$ in Line 2 below to approximate the coordinates of the points on the circle centered at the origin with radius $8$ that are closest and farthest from the point $B$ with coordinates $(12,-8)$.

\begin{onlineOnly}
    \begin{center}
\desmos{1pwptlcvoh}{900}{600}
\end{center}
\end{onlineOnly}

\href{https://www.desmos.com/calculator/1pwptlcvoh}{141: Closest Point on Circle 0}

\item Use algebra, not geometry or vectors, to find the exact coordinates of the points in part (a). Do \emph{not} use a calculator. Explain your reasoning.

\item Use a calculator (or the desmos worksheet above) and your answer from part (b) to approximate the coordinates of these points and compare them with your estimates in part (a).

\end{enumerate}
\end{question}


\begin{question}  \label{QKdferedr4333}
\begin{enumerate}

\item Drag the slider $u$ in Line 2 below to approximate the coordinates of the points on the circle centered at the point $(-3,-2)$ with radius $4$ that are closest and farthest from the point $B$ with coordinates $(1,-10)$.

\begin{onlineOnly}
    \begin{center}
\desmos{zdfejl3mpy}{900}{600}
\end{center}
\end{onlineOnly}

\href{https://www.desmos.com/calculator/zdfejl3mpy}{141: Closest Point on Circle}

\item Use algebra, not geometry or vectors, to find the exact coordinates of the points in part (a). Do \emph{not} use a calculator. Explain your reasoning.

\item Use a calculator (or the desmos worksheet above) and your answer from part (b) to approximate the coordinates of these points and compare them with your estimates in part (a).

\end{enumerate}
\end{question}


\end{document}