\documentclass{ximera}
\title{Quiz Solutions, Winter 2026}

\newcommand{\pskip}{\vskip 0.1 in}

\begin{document}
\begin{abstract}
Quiz solutions.
\end{abstract}
\maketitle


\section{Quiz 2B}

\begin{question} \label{Q2BDrel}

\begin{enumerate}

\item Use the methods of this class to find the coordinates of the point on the circle with radius $5$ centered at the origin that is farthest from the point $P$ with coordinates $(2,-3)$.

\begin{enumerate}
\item Start by drawing a reasonably accurate graph of the circle. Label point $P$ with its name and its coordinates. Label the axes with appropriate variable names.

\item Annotate your graph with the appropriate line(s) to help explain your thinking. 

\item Give a thorough explanation of your thinking in complete sentences.

\item Preface each computation block with a brief explanation of what you are about to compute.

\end{enumerate}
\end{enumerate}


\begin{explanation}
The points on the circle nearest and farthest from $P$ lie on the line through $P$ and the center of the circle. We'll use algebra to find the coordinates of these points.

We start by finding an equation of the line.

The line through the origin and the point $P$ with coordinates $(2,-3)$ has slope
\[
      \frac{\Delta y}{\Delta x} = \frac{-3-0}{2-0} = - \frac{3}{2} .
\]
A point $Q$ with coordinates $(x,y) \neq (0,0)$ lies on this line if and only if
\[
      \frac{y-0}{x-0} = - \frac{3}{2} .
\] 
And so an equation of the line is
\[
  y = - \frac{3}{2}x .
\]

Now for the circle. A point $Q$ with coordinates $(x,y)$ lies on the circle of radius $5$ centered at the origin if and only if the distance from $Q$ to the origin $O$ is equal to $5$. By the Pythogorean theorem
\[
  \text{dist}(Q,O) = \sqrt{(x-0)^2 + (y-0)^2} = \sqrt{x^2 + y^2}.
\]
So the point $Q(x,y)$ lies on this circle if and only if
\[
     \sqrt{x^2 + y^2} = 5.
\]
This is an equation of the circle.

Now to find the coordinates of the points where the line intersects the circle we need to solve the system
\[
\begin{cases}
        y = -\frac{3}{2}x  \\ \\
         \sqrt{x^2 + y^2} =  5 .
\end{cases}
\]

Substituting $y=-3x/2$ into the second equation and squaring both sides gives the equation
\[
     x^2 + \left(  -\frac{3}{2}x\right)^2 = 25.
\]
Then
\[
     x^2 + \frac{9}{4}x^2 = 25
\]
and
\[
\left( 1 + \frac{9}{4} \right) x^2 = 25.
\]
So
\[
   \frac{13}{4}x^2 = 25
\]
and
\[
     x^2 = \frac{100}{13} .
\]
So 
\[
    x = \pm \frac{10}{\sqrt{13}} .
\]

This makes sense. There are two points where the line intersects the circle. You'll need to draw a picture to see that the point $Q$ farthest from $P(2,-3)$ lies in the second quadrant. So $Q$ has $x$-coordinate
\[
     x = - \frac{10}{\sqrt{13}}
\]
and $y$-coordinate
\[
 y = -\frac{3}{2}x = \left( -\frac{3}{2} \right)  \left( - \frac{10}{\sqrt{13}} \right) = \frac{15}{\sqrt{13}} . 
\]

Our conclusion is that the point on the circle of radius $5$ centered at the origin that is closest to the point $(2,-3)$ has coordinates
\[
       \left(     - \frac{10}{\sqrt{13}} ,    \frac{15}{\sqrt{13}} \right) .
\]
\end{explanation}

\end{question}

\end{document}