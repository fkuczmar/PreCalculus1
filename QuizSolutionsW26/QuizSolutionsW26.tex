\documentclass{ximera}
\title{Quiz Solutions, Winter 2026}

\newcommand{\pskip}{\vskip 0.1 in}

\begin{document}
\begin{abstract}
Quiz solutions.
\end{abstract}
\maketitle


\section{Quiz 2B}

\begin{question} \label{Q2BDrel}

\begin{enumerate}

\item Use the methods of this class to find the coordinates of the point on the circle with radius $5$ centered at the origin that is farthest from the point $P$ with coordinates $(2,-3)$.

\begin{enumerate}
\item Start by drawing a reasonably accurate graph of the circle. Label point $P$ with its name and its coordinates. Label the axes with appropriate variable names.

\item Annotate your graph with the appropriate line(s) to help explain your thinking. 

\item Give a thorough explanation of your thinking in complete sentences.

\item Preface each computation block with a brief explanation of what you are about to compute.

\end{enumerate}
\end{enumerate}


\begin{explanation}
The points on the circle nearest and farthest from $P$ lie on the line through $P$ and the center of the circle. We'll use algebra to find the coordinates of these points.

We start by finding an equation of the line.

The line through the origin and the point $P$ with coordinates $(2,-3)$ has slope
\[
      \frac{\Delta y}{\Delta x} = \frac{-3-0}{2-0} = - \frac{3}{2} .
\]
A point $Q$ with coordinates $(x,y) \neq (0,0)$ lies on this line if and only if
\[
      \frac{y-0}{x-0} = - \frac{3}{2} .
\] 
And so an equation of the line is
\[
  y = - \frac{3}{2}x .
\]

Now for the circle. A point $Q$ with coordinates $(x,y)$ lies on the circle of radius $5$ centered at the origin if and only if the distance from $Q$ to the origin $O$ is equal to $5$. By the Pythogorean theorem
\[
  \text{dist}(Q,O) = \sqrt{(x-0)^2 + (y-0)^2} = \sqrt{x^2 + y^2}.
\]
So the point $Q(x,y)$ lies on this circle if and only if
\[
     \sqrt{x^2 + y^2} = 5.
\]
This is an equation of the circle.

Now to find the coordinates of the points where the line intersects the circle we need to solve the system
\[
\begin{cases}
        y = -\frac{3}{2}x  \\ \\
         \sqrt{x^2 + y^2} =  5 .
\end{cases}
\]

Substituting $y=-3x/2$ into the second equation and squaring both sides gives the equation
\[
     x^2 + \left(  -\frac{3}{2}x\right)^2 = 25.
\]
Then
\[
     x^2 + \frac{9}{4}x^2 = 25
\]
and
\[
\left( 1 + \frac{9}{4} \right) x^2 = 25.
\]
So
\[
   \frac{13}{4}x^2 = 25
\]
and
\[
     x^2 = \frac{100}{13} .
\]
So 
\[
    x = \pm \frac{10}{\sqrt{13}} .
\]

This makes sense. There are two points where the line intersects the circle. You'll need to draw a picture to see that the point $Q$ farthest from $P(2,-3)$ lies in the second quadrant. So $Q$ has $x$-coordinate
\[
     x = - \frac{10}{\sqrt{13}}
\]
and $y$-coordinate
\[
 y = -\frac{3}{2}x = \left( -\frac{3}{2} \right)  \left( - \frac{10}{\sqrt{13}} \right) = \frac{15}{\sqrt{13}} . 
\]

Our conclusion is that the point on the circle of radius $5$ centered at the origin that is closest to the point $(2,-3)$ has coordinates
\[
       \left(     - \frac{10}{\sqrt{13}} ,    \frac{15}{\sqrt{13}} \right) .
\]
\end{explanation}
\end{question}

\section{Quiz 3A}

\begin{question} \label{QkDfeeE}
\begin{enumerate}

\item Use the methods of this class to find the coordinates of the point on the circle with radius $3$ centered at the point $A(1,0)$ that is closest to the point $P$ with coordinates $(-2,1)$.

\begin{enumerate}
\item Start by drawing a reasonably accurate graph of the circle. Label point $P$ with its name and its coordinates. Label the axes with appropriate variable names.

\item Annotate your graph with the appropriate line(s) to help explain your thinking. 

\item Give a thorough explanation of your thinking in complete sentences.

\item Preface each computation block with a brief explanation of what you are about to compute.

\item Do not use the quadratic formula.

\end{enumerate}
\end{enumerate}

\begin{explanation}
The key idea is that the closest and farthest points lie on the line through $P$ and the circle's center. To find the coordinates of these points we'll first find equations of the line and circle and then use algebra to solve a system of equations.

\begin{onlineOnly}
    \begin{center}
\desmos{niopnpnuub}{900}{600}
\end{center}
\end{onlineOnly}

\href{https://www.desmos.com/calculator/niopnpnuub}{141: Quiz 3A}

\begin{enumerate}
\item First an equation of the circle. The logic is this. A point $Q$ with coordinates $(x,y)$ lies on the circle of radius $3$ centered at the point $A(1,0)$ if and only if the distance between $Q$ and $A$ is equal to $3$. Using the Pythagorean theorem, the distance beween $Q(x,y)$ and $A(1,0)$ is
\[
    \sqrt{(x-1)^2 + y^2} .
\]
So the point $Q(x,y)$ lies on the circle of radius $3$ centered at $A(1,0)$ if and only if
\[
  \sqrt{(x-1)^2 + y^2} = 3.
\]
This is an equation of the circle.

\item Now an equation of the line through the points $A(1,0)$ and $P(-2,1)$.

The slope of the line (ie. the rate of change of $y$ with respect to $x$) is
\[
 \frac{\Delta y}{\Delta x} = \frac{1-0}{-2-1} = - \frac{1}{3}.
\]

Now a point $Q$ (different from $A$) with coordinates $(x,y)$ lies on the line through $A(1,0)$ with slope $-1/3$ if and only if the slope of the segment $\overline{AQ}$ is equal to $-1/3$. The slope of $\overline{AQ}$ is
\[
     \frac{\Delta y}{\Delta x} = \frac{y-0}{x-1}
\]
and $Q(x,y)$ lies on the line if and only if
\[
  \frac{y-0}{x-1} = - \frac{1}{3} .
\]

So an equation of the line through $A(1,0)$ and $P(-2,1)$ is
\[
        y = -\frac{1}{3}(x-1) .
\]

\item To find the coordinates of the points where the line and circle intersect, we solve the system
\[
 \begin{cases}
       \sqrt{(x-1)^2+y^2} = 3 \\
        y = -\frac{1}{3}(x-1) . 
\end{cases}
\]

Squaring both sides of the first equation and substituting $y=-1/3(x-1)$ gives
\[
        (x-1)^2 + \left( - \frac{1}{3} (x-1)\right)^2 = 9 .
\]
Now be lazy. Don't distribute. Then
\[
 (x-1)^2 + \frac{1}{9}(x-1)^2 = 9
\]
and
\[
  \left(  1 + \frac{1}{9}  \right) (x-1)^2 = 9 .
\]
So
\[
  \frac{10}{9}(x-1)^2 = 9,
\]
and
\[
   (x-1)^2 = \frac{81}{10} .
\]
So the line and circle intersect in the two points with $x$-coordinates
\[
    x = 1 \pm \frac{9}{\sqrt{10}} .
\]
To find the $y$-coordinates of these points, we use the equation of the line 
\[
    y = -\frac{1}{3}(x-1) .
\]
When
\[
  x = 1 + \frac{9}{\sqrt{10}}
\]
\begin{align*}
y &= -\frac{1}{3}(x-1) \\
   &= -\frac{1}{3} \left( \left( 1 + \frac{9}{\sqrt{10}} \right)  - 1\right) \\
   &=  -\frac{1}{3} \left(\frac{9}{\sqrt{10}} \right) \\
   &= -\frac{3}{\sqrt{10}}.
\end{align*}

Similarly, if 
\[
  x = 1 - \frac{9}{\sqrt{10}},
\]
\[
     y = \frac{3}{\sqrt{10}} .
\]

So the line and circle intersect in the points with coordinates
\[
  \left(  1 + \frac{9}{\sqrt{10}} , -\frac{3}{\sqrt{10}} \right)
\]
or
\[
     \left(  1 - \frac{9}{\sqrt{10}} , \frac{3}{\sqrt{10}} \right) .
\]
\end{enumerate}

From the graph above we can see the point on the circle with radius $3$ centered at $(1,0)$ that is closest to the point $(-2,1)$ has coordinates
\[
     \left(  1 - \frac{9}{\sqrt{10}} , \frac{3}{\sqrt{10}} \right) .
\]

\end{explanation}
\end{question}


\end{document}