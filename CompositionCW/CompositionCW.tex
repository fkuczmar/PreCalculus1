\documentclass{ximera}
\title{Composition of Functions, CW}

\newcommand{\pskip}{\vskip 0.1 in}

\begin{document}
\begin{abstract}
Composing functions.
\end{abstract}
\maketitle



\begin{example}  \label{Erer6t364}
Use the graphs of the functions $y=f(x)$ and $y=g(x)$ below to find approximate answers to the following questions. Drag points $A$ and $B$ as needed to help with your approximations. Use set notation to write solution sets when solving an equation.

\begin{onlineOnly}
    \begin{center}
\desmos{githr11qtb}{900}{600}
\end{center}
\end{onlineOnly}

\href{https://www.desmos.com/calculator/githr11qtb}{141: Composition CW1}

\begin{enumerate}

\item Evaluate $f(g(-2))$.

\item Evaluate $g(f(0))$.

\item Solve the equation $g(f(x))=10$. Use the method of $u$-substitution.

\item Solve the equation $f(g(x)) = 10$. Use the method of $u$-substitution.

\item Solve the equation $f(f(x)) = 6$. Use the method of $u$-substitution.

\item Solve the equation $g(g(x)) = 0$. Use the method of $u$-substitution.
\end{enumerate}
\end{example}


\begin{example}  \label{Ex0:Comp55}
Let
\[
    f(x) = \frac{24}{x^2-x-6}
\]
and 
\[
    g(x) = \frac{3x-5}{x-4} .
\]


Answer each of the following questions \emph{without} using the quadratic formula. Use set notation to write solution sets when solving an equation.
\begin{enumerate}

\item Solve the equation $g(f(x))=23/10$. Use the method of $u$-substitution.

\item Solve the equation $g(f(x)) = 17/8$. Use the method of $u$-substitution.

\item Solve the equation $g(g(x)) = 2$. Use the method of $u$-substitution.

\item Find the domain of the composition $g(g(x))$. Use $u$-substitution and write the domain as a set.

\item Find the domain of the composition $g(g(g(x)))$. Use $u$-substitution and write the domain as a set.
\end{enumerate}

\end{example}



\begin{question} \label{Ex:55rtggytt}
Use the graphs of the function $y=f(x)$ and $y=g(x)$ to approximate the answers the following questions. Use the method of $u$-substitution. Drag the sliders to the appropriate positions and show screenshots to help with your explanations. Do \emph{not} make any computations.

Write solutions to equations and inequalities using set-builder notation, \emph{not} interval notation.

\begin{onlineOnly}
    \begin{center}
\desmos{xkgmezvis2}{900}{600}
\end{center}
\end{onlineOnly}


\begin{enumerate}


\item Solve the inequality $f(g(x))\leq 3$.

\item Solve the equation $f(g(x))=x$.

\end{enumerate}
\end{question}


\begin{question}   \label{Q1:Comp}
There are two conditions for $x$ to be in the domain of the composition $f(g(x))$. What are they?
\end{question}


\begin{example}  \label{Ederg4t4r5}
\begin{enumerate}

\item Use the graph of $y=f(x)$ shown below to approximate the solutions to the equation
\[
    f(f(x)) = 5 .
\]

\begin{onlineOnly}
    \begin{center}
\desmos{d6zcmotesc}{900}{600}
\end{center}
\end{onlineOnly}

\href{https://www.desmos.com/calculator/d6zcmotesc}{141: Domain of Composition 5}

\item Suppose that 
\[
       f(x) = -x^2 +10x - 16 \, , \, 2\leq x \leq 8 .
\]
Ignore your work from part (a), start from scatch, and use algebra to find the exact solutions to the equation 
\[
    f(f(x)) = 5.
\]

\item Use the graph of $y=f(x)$ above to approximate the domain of the composition $y=f(f(x))$. Write the domain using set notation.

\item Use algebra to find the exact domain of the composition $y=f(f(x))$.

\end{enumerate}
\end{example}


\begin{question} \label{Q324df43ggr}
\begin{enumerate}
\item Use the graph of $y=f(x)$ shown below to approximate the solutions to the equation
\[
    f(f(x)) = 4 .
\]

\begin{onlineOnly}
    \begin{center}
\desmos{hbegdhlhbb}{900}{600}
\end{center}
\end{onlineOnly}

\href{https://www.desmos.com/calculator/hbegdhlhbb}{141: Domain of Composition 6}

\item Suppose that 
\[
       f(x) = x^{2}+6x+4\  \, , \, -7\leq x\leq 1  .
\]
Ignore your work from part (a), start from scatch, and use algebra to find the exact solutions to the equation 
\[
    f(f(x)) = 4.
\]

\item Find, if possible, a value of $c$ such that the equation
\[
    f(f(x))=c
\]
has \emph{exactly} three solutions. Explain your reasoning.

\end{enumerate}

\end{question}


\begin{example} \label{Ex1:Com334p}
Let 
\[
     f(x) = \frac{x+3}{7x+23}
\]
and
\[
    g(x) = \frac{x}{x-5} .
\]

Use the answer to Question 1 to help you with some of the following questions. Do {\bf not} compute any compositions. Express each domain in set notation.
 
(a) Find the domain of $f$.

(b) Find the domain of $g$.

(c) Find the domain of $g\circ f$.

(d) Find the domain of $f\circ g$.

(e) Find the domain of $g\circ g$.

(f) Find the domain of $g\circ g\circ g$.

\end{example}


\end{document}
