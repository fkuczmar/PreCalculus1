\documentclass{ximera}
\title{Inverse Functions}

\newcommand{\pskip}{\vskip 0.1 in}

\begin{document}
\begin{abstract}
Pairs of inverse functions.
\end{abstract}
\maketitle

The action of putting on your shoes and the action of taking off your shoes are inverses. Each undoes the other. The actions of adding 4 and subtracting 4 are also inverses of each other. If you first add 4 to any number and then subtract 4 from the result, you end up with your original number. And subtracting 4 from a number and then adding 4 to the result brings you back to your original number. 


\begin{example} \label{Exd445t667}
The above example about adding and subtracting 4 was really about functions. Inverse functions come in pairs and each undoes the other. For example, the functions 
\[
    f(x) = x + 4
\]
and 
\[
  g(x) = x-4
\]
are inverses. For example,
\[
    g(f(23)) = g(23+4) =  (23+4) -4 = 23
\]
and
\[
    f(g(23)) = f(23-4) = (23-4) + 4 = 23 .
\]

To show algebraically that the functions $f$ and $g$ are inverses of each other, we should simplify the compositions $g(f(x))$ and $f(g(x))$. For the first composition
\begin{align*}
          g(f(x)) &= g(x+4)  \\
                    &= (x+4) - 4 \\
                    & = x .
\end{align*}
This shows that $g$ undoes $f$. To show that $f$ undoes $g$, we need to show $f(g(x))=x$. Here's the computation:
\begin{align*}
          f(g(x)) &= f(x-4)  \\
                    &= (x-4) + 4 \\
                    & = x .
\end{align*}


\end{example}



When the functions $f$ and $g$ are inverses of each other, we write $g(x) = f^{-1}(x)$ or $f(x) = g^{-1}(x)$. So, for example, for 
\[
   f(x) = x+4 ,
\]
the inverse function is
\[
   f^{-1}(x) = x-4.
\]

{\bf Note:} The notation $f^{-1}(x)$ for the inverse can be misleading. While, for example $3^{-1} = 1/3$ means the reciprocal of $3$, the function $f^{-1}(x)$ is {\bf not} the reciprocal of $f(x)$. So in the example above, the reciprocal
\[
     \frac{1}{f(x)} = \frac{1}{x+4} 
\] 
is not the same as the inverse $f^{-1}(x) = x-4$.


\begin{example}  \label{Ex:4577hhhh}
Find an expression for the inverse of the function
\[
    y= f(x) = 4x -3 
\]
and show algebraically that you are correct.

\begin{explanation}
The function $f$ expresses the output $y=f(x)$ in terms of the input $x$. Our goal in finding an expression for the inverse is to express  $x$ as a function of $y$. That is, we are looking to solve the above equation for $x$ in terms of $y$. This will give us an expression $x=f^{-1}(y)$ for the inverse function. 

One approach is to think about the action of $f$. The function $f$ first multiplies its input by $4$ and then subtracts $3$ from the result. To undo this process we first undo the second action (subtracting $3$) by adding $3$. Then we undo the first action (multiplying by $4$) by multiplying by the reciprocal $1/4$. The result is that
\[
     x =    f^{-1}(y) = \frac{1}{4} \left( x+3 \right) .
\] 

A second, more algebraic approach is to solve the equation $y=4x-3$ for $x$ in terms of $y$ as follows.

Since 
\[
    y = 4x - 3, 
\]
\[
         y + 3 = 4x ,
\]
and
\[
    x = f^{-1}(y) = \frac{1}{4}\left( y + 3 \right) .
\]

Now since $x$ and $y$ are dummy variables (ie. they do not have any practical meaning in this example), we can write the inverse function as 
\[
       y = f^{-1}(x) =  \frac{1}{4}\left( x+3 \right) .
\]

To show algebraically that $f$ and $f^{-1}$ are actually inverses of each other, we need to show that $f^{-1}$ undoes $f$ and that $f$ undoes $f^{-1}$. For the first, we simplify the composition $f^{-1}(f(x))$:
\begin{align*}
        f^{-1}(f(x)) &= f^{-1}(4x-3) \\
                          &= \frac{1}{4}\left( (4x-3)+3 \right) \\
                          &= \frac{1}{4}\left( 4x \right) \\
                          &= x .
\end{align*}

To show that $f$ undoes $f^{-1}$, we simplify the composition $f(f^{-1}(x))$.

\begin{question}  \label{Q343fggg}
Simplify the composition $f(f^{-1}(x))$ showing \emph{all} steps.
\end{question}
\end{explanation}
\end{example}


\begin{question}  \label{Qre345rrree}
Use the two methods of the previous example to find an expression for the inverse of the function 
\[
      y =  g(x) = 5 - \frac{7}{9} \left( x + 53 \right) . 
\] 
Then show algebraically that the functions are inverses of each other.
\end{question}


\begin{example} \label{Ex1:Inverses}
The function
\[
    C = g(F) = \frac{5}{9}\left( F - 32  \right) , F\geq -460 ,
\]
expresses the Celsius temperature in terms of the corresponding Fahrenheit temperature.

Interpret the meaning of each of the following expressions in the context of this particular example. Then simplify each expression.

\pskip

(a) $g(68)$

(b) $g^{-1}(10)$

(c) $g^{-1}(C)$. State the domain of $g^{-1}$.

(d)  $g^{-1}(g(68))$

(e) $g(g^{-1}(10))$

(f) $g^{-1}(g(F))$. Show all work.

(g) $g(g^{-1}(C))$. Show all work.

(h) $g(68) - g(5)$

(i) $g(768) - g(705)$. Comments?

(j) $g^{-1}(g(14)+40)$

(k) $g^{-1}(g(F)+40)$. Comments for this one and the remaining questions?

(l) $g^{-1}(g(F)+40) - F$

(m) $g(g^{-1}(40) - 27)$

(n) $g(g^{-1}(C) - 27)$

(o) $g(g^{-1}(C) - 27) - C$

(p) $g(g^{-1}(C) - T) - C$

\end{example}



\begin{example}  \label{Ex2:Inverse}
Desribe in words the inverses of each of the following actions.

(a) the action of adding 25 to a number

(b) the action of subtracting a number from 25

(c) the action of dividing a number by 25

(d) the action of dividing a non-zero number into 25

(e) the action of subtracting 8 from a number, then dividing the result into 25, and finally adding 8. Is there any restriction on the original number?

(f) the action of subtracting 25 from a number, multiplying the result by $5/7$, and then adding 8. Do {\bf not} use ``divide" in your description of the inverse. Use ``multiply'' instead.

(g) the action of squaring a positive number

(h) the action of squaring a negative number

(i) the action of subtracting the square of a positive number from 25 and then taking the square root of the result

(j) the action of subtracting the square of a negative number from 25 and then taking the square root of the result
\end{example}


\begin{example}  \label{Ex3:Inverse}
The function
\[
    p = f(q) = 6 - \frac{1}{15} \left(  q- 170 \right) , 110 \leq q \leq 200 ,
\]
expresses the price (in dollars/burger) in term of the average number of burgers (measured in burgers/day) sold by Five Guys of Edmonds.

(a) Sketch a grap of the function $p=f(q)$ by hand.

(b) Find the range of $f$.

(c) Explain the meaning of the function $f^{-1}$. What does it take as an input? What does it return as an output.

(d) Solve the appopriate equation to find an expression for $f^{-1}$. Use the appropriate variables for the input and output. Keep in mind part (f) of Example 3 in solving your equation.

(e) Find the domain of $f^{-1}$. Explain your reasoning. Then sketch by hand a graph of $f^{-1}$.

(f) What should you get if you simplified the composition $f^{-1}(f(q))$? Explain your reasoning.

(g) Simplify the composition $f^{-1}(f(q))$ algebraically. Show all steps.

(h) Explain the meaning of the function 
\[
  g(p) = f(f^{-1}(p)+9).
\]
What does it take as an input? What does it return as an output?

(i) Simplify the function $g$ in part (h).

(j) Find the domain of the function $g$ in part (h).

(k) Graph the function $g$ by hand. 


\end{example}




\begin{example} \label{Ex4:Inverses}
This example is about the rate at which a car burns gas, measured in gallons/hour. At any instant, this rate depends on the speed of the car and the gas mileage. For example, suppose that a car gets 30 miles/gallon at a speed of 60 miles/hour. Then at this speed the car burns gas at a rate of
\[
     \frac{60 \text{ miles/hour}}{30 \text{ miles/gal}} = 2 \text{ gal/hour}.
\]

This makes sense both unit-wise and logically. In one hour the car travels 60 miles. And since the car burns one gallon of gas every 30 miles, it burns
\[
    \frac{60 \text{ miles}}{30 \text{ miles/gal}} = 2 \text{ gallons}
\]
of gas every hour.

\pskip

Now suppose the function 
\[
    G = f(v) = 40-\frac{3}{7}(v-50) ,  \,\, 50\leq v \leq 78 ,
\]
expresses the gas mileage of a car (in miles/gal) in terms of its speed (in miles/hour).

(a) At what rate (in gal/hour) does the car burn gas at a speed of $64$ miles/hour?

(b) Use the idea in part (a) to find a function 
\[
     r=h(v) , \,\, 50\leq v \leq 78 ,
\]
that expresses the rate at which the car burns gas (in gal/hour) in terms of its speed (in miles/hour). This is {\bf not} a linear function. You should simplify the function by multiplying the numerator and denominator by the same number to clear the fractions. %This is {\bf not} a linear function.

(c) Find the range of the function $h$.

(d) Explain the meaning of the function $h^{-1}$. What does it take as an input? What does in return as an output?

(e) Find an expression for the function $h^{-1}$. Use the appropriate variables for the input and output.

(f) What is the domain of the function $h^{-1}$?

(g) What should you get if you simplified the composition $h(h^{-1}(r))$? Explain your reasoning.

(h) Simplify the composition $h(h^{-1}(r))$ algebraically. Show all steps.

(i) At what speed does the car burn gas at the rate of $2$ gal/hour?

\end{example}


\begin{example}  \label{Ex:9dfctghgh}
The increasing function
\[
   G = f(v) \,  , \, 20\leq v \leq 55 ,
\]
expresses the gas mileage of a car (in  miles/gal) in terms of its speed (in miles/hr). 

Interpret the \emph{meaning} of each of the following expressions \emph{without} using the graph of the function $G=f(v)$ shown below. Then use the graph to approximate the value of each expression.

(a) $f(40)$

(b) $f^{-1}(40)$

(c) $f(f^{-1}(23))$

(d) $f^{-1}(f(20) + 5)$

(e) $f(f^{-1}(40)-10)$

\begin{onlineOnly}
    \begin{center}
\desmos{vyv8xmc7mg}{900}{600}
\end{center}
\end{onlineOnly}

\pskip

(f) Sketch a graph of the function $f^{-1}$. Label the axes with the appropriate variable names and units. Express the domain of this function in set-builder (\emph{not} interval) notation.

\end{example}

\begin{example}  \label{Ex:oodfdsfo}
The function
\[
   G = f(v) = 62 - \frac{1000}{v+10}\,  , \, 20\leq v \leq 55 ,
\]
expresses the gas mileage of a car (in  miles/gal) in terms of its speed (in miles/hr). 

(a) Find an expression for the function 
\[
    v = f^{-1}(G).
\]
Express the domain of this function in set-builder (\emph{not} interval) notation.

(b) Use algebra to simplify the compositions $f^{-1}(f(v))$ and $f(f^{-1}(G))$. 
  
(c) Suppose that increasing your speed by 10 miles/hour increases  your gas mileage by $5$ miles/gal. Use algebra to determine your speed.


\end{example}



\begin{example} \label{Ex5:Inverses}
The increasing function
\[
     h = f(s) , 0\leq s \leq 19 ,
\]
expresses the altitude (in thousands of feet) in terms of your trip odometer reading (in miles) as you drive along a road in Colorado. 

The decreasing function 
\[
  T = g(h) , 0\leq h \leq 10 ,
\]
expresses the temperature (in $^\circ$C) in terms of the altitude (in thousands of feet) at points along the road.

Translate each of the following from math to English if possible. Then use the graphs of the functions $f$ and $g$ below to evaluate the function and give an answer to the English question in a complete sentence. If a composition does not make sense, say so and explain why. Do {\bf not} use the graphs to help you translate the math into English.

Use the axes to approximate the coordinates of a point. Do {\bf not} click on a point and use the coordinates displayed by Desmos.

\pskip

(a) Evaluate $f^{-1}(6)$.

(b) Evaluate $g^{-1}(6)$.

(c) Evaluate $f^{-1}(g^{-1}(7))$.

(d) Evaluate $g^{-1}(f^{-1}(7))$.

(e) Evaluate $f^{-1}(g^{-1}(5)+4)$.

\begin{exploration}

\pdfOnly{
Access Desmos interactives through the online version of this text at
 
\href{https://www.desmos.com/calculator/vyv8xmc7mg}.
}
 
\begin{onlineOnly}
    \begin{center}
\desmos{vyv8xmc7mg}{900}{600}
\end{center}
\end{onlineOnly}
\end{exploration}

\end{example}


\begin{example} \label{Ex6:Inverses}
(a) Let
\[
  y = f(x) = -82 + \frac{7}{9}(x+113)  \, , -113\leq x \leq 4 .
\] 

i) Describe in words the action of the function $f$. What does it do to the input?

ii) Use your description from part i) to describe the action of $f^{-1}$ {\bf without} referring to division.

iii) Use your description from part ii) to find an expression for $f^{-1}(y)$. This function expresses the input of the function $f$ in terms of its output.

iv) Use the expression for $y=f(x)$ above and algebra to find an expression for $f^{-1}(y)$. Make sure it matches your result from part iii).

v) Find the domain of $f^{-1}$.

vi) Simplify the composition $f^{-1}(f(x))$ algebraically. Show all steps.

vii) Simplify the composition $f(f^{-1}(y))$ algebraically. Show all steps.

\pskip

(b) Consider the action of subtracting a number from 100. Describe in words the inverse of this action. Verify algebraically that your description of the inverse is correct.

(c) The function
\[
    C = g(F) = \frac{5}{9}\left( F - 32  \right) , F\geq -460 ,
\]
expresses the Celsius temperature in terms of the corresponding Fahrenheit temperature.

i) What are the units of $32$ and $5/9$ in the expression for $g$ above? What is the meaning of $5/9$?

ii) Explain the meaning of $g(g(45))$ if possible. Explain your logic.

iii) Explain the meaning of $g^{-1}(g(5)-40)$ if possible. Explain your logic.

iv) Evaluate the composition $g^{-1}(g(5)-40)$. Based on your explanation in part iii), explain how you could have evaluated the compostion with a simpler computation.

v) Explain the  meaning of the function
\[
    W = h(F) = g^{-1}(g(F)-40) .
\]

vi) Use algebra to find a simplified expression for $h(F)$. Explain how you could have arrived at the same result using your description from part v).

vii) Find the domain of the function $h$.



\end{example}


\begin{example} \label{Ex15:Inverses}
The function
\[
   T = f(h) \, , 2\leq h \leq 10 ,
\]
expresses the temperature (in Celsius degrees) in terms of the altitude (measured in thousands of feet) on the east side of Mount Rainier National Park.

\pskip

(a) Interpret the meaning of each of the following expressions if possible.

i) $f^{-1}(6)$

ii) $f(f(6))$

iii) $f^{-1}(f(6))$

iii) $f^{-1}(f(6)-2)$

iv) $f(f^{-1}(6)+2)$

v) $f^{-1}(f(h)-2)$

vi) $f(f^{-1}(T)+2)$

\pskip

(b) Suppose 
\[
     T = f(h) = 14- \frac{3}{5} h \,\,\,  , 2\leq h \leq 10.
\]

\pskip

i) Find an expression for the function $h=f^{-1}(T)$. Include the domain.

ii) Find an expression for the function
\[
    g(h) = f^{-1}(f(h)-2).
\]  
Include the domain.

\pskip

(c) Suppose instead that
\[
   T = f(h) = \frac{2h+10}{h-1}  \,\,\,  , 2\leq h \leq 10.
\]

\pskip

i) Find an expression for the function $h=f^{-1}(T)$. Include the domain.

ii) Use algebra to simplify the compositions $f(f^{-1}(T))$ and $f^{-1}(f(h))$. Show all steps.

\end{example}


\end{document}