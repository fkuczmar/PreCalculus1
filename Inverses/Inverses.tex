\documentclass{ximera}
\title{Inverse Functions}

\newcommand{\pskip}{\vskip 0.1 in}

\begin{document}
\begin{abstract}
Pairs of inverse functions.
\end{abstract}
\maketitle

The action of putting on your shoes and the action of taking off your shoes are inverses. Each undoes the other. The actions of adding 4 and subtracting 4 are also inverses of each other. If you first add 4 to any number and then subtract 4 from the result, you end up with your original number. And subtracting 4 from a number and then adding 4 to the result brings you back to your original number. 

The last example was really about functions. Inverse functions come in pairs and each undoes the other. For example, the functions 
\[
    f(x) = x + 4
\]
and 
\[
  g(x) = x-4
\]
are inverses. For example,
\[
    g(f(23)) = g(23+4) =  (23+4) -4 = 23
\]
and
\[
    f(g(23)) = f(23-4) = (23-4) + 4 = 23 .
\]

\begin{question}   \label{Q1:Inverses}
Simplify the compositions $g(f(x))$ and $f(g(x))$ for the functions $f$ and $g$ above. Explain.
\end{question}

When the functions $f$ and $g$ are inverses of each other, we write $g(x) = f^{-1}(x)$ or $f(x) = g^{-1}(x)$. So, for example, for 
\[
   f(x) = x+4 ,
\]
the inverse function is
\[
   f^{-1}(x) = x-4.
\]

{\bf Note:} The notation $f^{-1}(x)$ for the inverse can be misleading. While, for example $3^{-1} = 1/3$ means the reciprocal of $3$, the function $f^{-1}(x)$ is {\bf not} the reciprocal of $f(x)$. So in the example above, the reciprocal
\[
     \frac{1}{f(x)} = \frac{1}{x+4} 
\] 
is not the same as the inverse $f^{-1}(x) = x-4$.


\begin{example} \label{Ex1:Inverses}
The function
\[
    C = g(F) = \frac{5}{9}\left( F - 32  \right) , F\geq -460 ,
\]
expresses the Celsius temperature in terms of the corresponding Fahrenheit temperature.

Interpret the meaning of each of the following expressions in the context of this particular example. Then simplify each expression.

\pskip

(a) $g(68)$

(b) $g^{-1}(10)$

(c) $g^{-1}(C)$. State the domain of $g^{-1}$.

(d)  $g^{-1}(g(68))$

(e) $g(g^{-1}(10))$

(f) $g^{-1}(g(F))$. Show all work.

(g) $g(g^{-1}(C))$. Show all work.

(h) $g(68) - g(5)$

(i) $g(768) - g(705)$. Comments?

(j) $g^{-1}(g(14)+40)$

(k) $g^{-1}(g(F)+40)$. Comments for this one and the remaining questions?

(l) $g^{-1}(g(F)+40) - F$

(m) $g(g^{-1}(40) - 27)$

(n) $g(g^{-1}(C) - 27)$

(o) $g(g^{-1}(C) - 27) - C$

(p) $g(g^{-1}(C) - T) - C$



\end{example}

\end{document}