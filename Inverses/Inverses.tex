\documentclass{ximera}
\title{Inverse Functions}

\newcommand{\pskip}{\vskip 0.1 in}

\begin{document}
\begin{abstract}
Pairs of inverse functions.
\end{abstract}
\maketitle

The action of putting on your shoes and the action of taking off your shoes are inverses. Each undoes the other. The actions of adding 4 and subtracting 4 are also inverses of each other. If you first add 4 to any number and then subtract 4 from the result, you end up with your original number. And subtracting 4 from a number and then adding 4 to the result brings you back to your original number. 

The last example was really about functions. Inverse functions come in pairs and each undoes the other. For example, the functions 
\[
    f(x) = x + 4
\]
and 
\[
  g(x) = x-4
\]
are inverses. For example,
\[
    g(f(23)) = g(23+4) =  (23+4) -4 = 23
\]
and
\[
    f(g(23)) = f(23-4) = (23-4) + 4 = 23 .
\]

\begin{question}   \label{Q1:Inverses}
Simplify the compositions $g(f(x))$ and $f(g(x))$ for the functions $f$ and $g$ above. Explain.
\end{question}

When the functions $f$ and $g$ are inverses of each other, we write $g(x) = f^{-1}(x)$ or $f(x) = g^{-1}(x)$. So, for example, for 
\[
   f(x) = x+4 ,
\]
the inverse function is
\[
   f^{-1}(x) = x-4.
\]

{\bf Note:} The notation $f^{-1}(x)$ for the inverse can be misleading. While, for example $3^{-1} = 1/3$ means the reciprocal of $3$, the function $f^{-1}(x)$ is {\bf not} the reciprocal of $f(x)$. So in the example above, the reciprocal
\[
     \frac{1}{f(x)} = \frac{1}{x+4} 
\] 
is not the same as the inverse $f^{-1}(x) = x-4$.


\begin{example} \label{Ex1:Inverses}
The function
\[
    C = g(F) = \frac{5}{9}\left( F - 32  \right) , F\geq -460 ,
\]
expresses the Celsius temperature in terms of the corresponding Fahrenheit temperature.

Interpret the meaning of each of the following expressions in the context of this particular example. Then simplify each expression.

\pskip

(a) $g(68)$

(b) $g^{-1}(10)$

(c) $g^{-1}(C)$. State the domain of $g^{-1}$.

(d)  $g^{-1}(g(68))$

(e) $g(g^{-1}(10))$

(f) $g^{-1}(g(F))$. Show all work.

(g) $g(g^{-1}(C))$. Show all work.

(h) $g(68) - g(5)$

(i) $g(768) - g(705)$. Comments?

(j) $g^{-1}(g(14)+40)$

(k) $g^{-1}(g(F)+40)$. Comments for this one and the remaining questions?

(l) $g^{-1}(g(F)+40) - F$

(m) $g(g^{-1}(40) - 27)$

(n) $g(g^{-1}(C) - 27)$

(o) $g(g^{-1}(C) - 27) - C$

(p) $g(g^{-1}(C) - T) - C$

\end{example}



\begin{example}  \label{Ex2:Inverse}
Desribe in words the inverses of each of the following actions.

(a) the action of adding 25 to a number

(b) the action of subtracting a number from 25

(c) the action of dividing a number by 25

(d) the action of dividing a non-zero number into 25

(e) the action of subtracting 8 from a number, then dividing the result into 25, and finally adding 8. Is there any restriction on the original number?

(f) the action of subtracting 25 from a number, multiplying the result by $5/7$, and then adding 8.

(g) the action of squaring a positive number

(h) the action of squaring a negative number

(i) the action of subtracting the square of a positive number from 25 and then taking the square root of the result

(j) the action of subtracting the square of a negative number from 25 and then taking the square root of the result
\end{example}


\begin{example}  \label{Ex3:Inverse}
The function
\[
    p = f(q) = 6 - \frac{1}{15} \left(  q- 170 \right) , 110 \leq q \leq 200 ,
\]
expresses the price (in dollars/burger) in term of the average number of burgers (measured in burgers/day) sold by Five Guys of Edmonds.

(a) Sketch a grap of the function $p=f(q)$ by hand.

(b) Find the range of $f$.

(c) Explain the meaning of the function $f^{-1}$. What does it take as an input? What does it return as an output.

(d) Solve the appopriate equation to find an expression for $f^{-1}$. Use the appropriate variables for the input and output. Keep in mind part (f) of Example 2 in solving your equation.

(e) Find the domain of $f^{-1}$. Explain your reasoning. Then sketch by hand a graph of $f^{-1}$.

(f) What should you get if you simplified the composition $f^{-1}(f(q))$? Explain your reasoning.

(g) Simplify the composition $f^{-1}(f(q))$ algebraically. Show all steps.

(h) Explain the meaning of the function 
\[
   g(p) = f(f^{-1}(p)+9).
\]
What does it take as an input? What does it return as an output?

(i) Simplify the funciton $g$ in part (h).

(j) Find the domain of the function $g$ in part (h).


\end{example}


\begin{example} \label{Ex4:Inverses}
The function 
\[
    G = f(v) = 40-\frac{3}{7}(v-50) ,  \,\, 50\leq v \leq 78 ,
\]
expresses the gas mileage of a car (in miles/gal) in terms of its speed (in miles/hour).

(a) At what rate (in gal/hour) does the car burn gas at a speed of $64$ miles/hour?

(b) Use the idea in part (a) to find a function 
\[
     r=h(v) , \,\, 50\leq v \leq 78 ,
\]
that expresses the rate at which the car burns gas (in gal/hour) in terms of its speed (in miles/hour). This is {\bf not} a linear function.

(c) Find the range of the function $h$.

(d) Explain the meaning of the function $h^{-1}$. What does it take as an input? What does in return as an output?

(e) Find an expression for the function $h^{-1}$. Use the appropriate variables for the input and output.

(f) What is the domain of the function $h^{-1}$?

(g) What should you get if you simplified the composition $h(h^{-1}(r))$? Explain your reasoning.

(h) Simplify the composition $h(h^{-1}(r))$ algebraically. Show all steps.

(i) At what speed does the car burn gas at the rate of $2$ gal/hour?

\end{example}



\begin{example} \label{Ex5:Inverses}

\begin{exploration}

\pdfOnly{
Access Desmos interactives through the online version of this text at
 
\href{https://www.desmos.com/calculator/vyv8xmc7mg}.
}
 
\begin{onlineOnly}
    \begin{center}
\desmos{vyv8xmc7mg}{900}{600}
\end{center}
\end{onlineOnly}
\end{exploration}

\end{example}


\end{document}