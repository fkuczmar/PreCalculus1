\documentclass{ximera}
\title{Functions}

\newcommand{\pskip}{\vskip 0.1 in}

\begin{document}
\begin{abstract}
Using function notation to translate between English and mathematics; reading information from the graphs of functions.
\end{abstract}
\maketitle


\section{The Algebra of Functions}

\begin{question}  \label{Qersr4r}
Let
\[
     f(x) = x^2 + 6x + 1 .
\]

Answer the following questions. When solving a quadratic equation either complete the square or factor. Do \emph{not} use the quadratic formula. Write solutions to equations as solution sets using set-builder notation. No need for explanations. 

(a) Evaluate $f(1)$.

(b) Solve the equation $f(x)=10$.

(c) Solve the equation $f(x)=f(10)$.

(d) Solve the equation $f(x) = 2 f(-1)$.

(e) Simplify the expression $f(w) - f(w-3)$.

(f) Simplify the expression
\[
      \frac{f(w) - f(3)}{w-3} .
\]

(g) Solve the equation
\[
           f(t+3) - f(t) = 10 .
\]

(h) Solve the equation
\[
           \frac{f(x) - f(3)}{x-3} = 10.
\]
\end{question}



\section{Interpreting Graphs, Translating Between Math and English}

\begin{example}  \label{ExFun1}

The function
\[
   g = f(v) , 20 \leq v \leq 75 ,
\]
expresses the gas mileage (in miles/gal) of a car in terms of its speed (in miles/hour). Translate each of the following into mathematics.

\pskip

% f(v)=40-0.02(v-50)^2 

\noindent (a) What is the car's gas mileage at a speed of 40 miles/hour?

\pskip

\emph{Answer:} To determine the car's gas mileage at a speed of $40$ miles/hour, we would evaluate $f(40)$.

\pskip

\noindent (b) At what speeds does the car get 32 miles/gal?

\pskip

\emph{Answer:} To determine the speed at which the car gets $32$ miles/gallon, we would solve the equation
\[
    f(v) = 32 .
\]

\pskip

\noindent (c) At what speeds does the car get at most 32 miles/gal?

%\pskip

%\emph{Answer:} To determine the speed at which the car gets at least $32$ miles/gallon, we would solve the inequality
%\[
%    f(v) \geq 32 .
%\]

%\pskip


\noindent (d) At what speeds does the car get the same gas mileage as it does at a speed of 40 miles/hour?

%\pskip

%\emph{Answer:} To determine the speeds at which the car gets the same gas mileage as it does at a speed of 44 miles/hour we would solve the equation
%\[
%       f(v) = f(44).
%\]

%\pskip

\noindent (e) At what speeds is the gas mileage $3$ miles/gal less than it is at 34 miles/hour?

%\pskip

%\emph{Answer:} To determine the speeds at which the car gets $2$ miles/gal less than it does at a speed of 44 miles/hour we would solve the equation
%\[
%       f(v) = f(40) - 2.
%\]

%\pskip

\noindent (f) At what speeds is the mileage 10 miles/gal greater than it is at 20 miles/hour?

\noindent (g) Find the change in the gas mileage if the speed increases from 60 miles/hour to 70 miles/hour.

\noindent (h) Find the average rate of change in the mileage with respect to speed if the speed increases from 60 miles/hour to 70 miles/hour. Include units.

\noindent (i) At what speeds does an increase of 10 miles/hour in the speed leave the gas mileage unchanged?

\noindent (j) At what speeds does an decrease of 10 miles/hour in the speed increase the mileage by 2 miles/gal?

\end{example}


\begin{example}  \label{ExFun2}

The function
\[
   g = f(v) , 20 \leq v \leq 75 ,
\]
expresses the gas mileage (in miles/gal) of a car in terms of its speed (in miles/hour). Translate each of the following into everyday Engish.

\pskip

\noindent (a) Solve the inequality $f(v)\leq 38$.

\noindent (b) Solve the equation $f(v) = f(45)$.

\noindent (c) Solve the inequality $f(v) \leq f(45)$.

\noindent (d) Solve the inequality $f(v)> f(50)-4$.

\noindent (e) Solve the equation $f(v)=f(v-8)$.

\noindent (f) Evaluate $f(w)$.

\end{example}


\begin{example}  \label{ExFun3}
Use the graph of the function 
\[
    g = f(v) , 20\leq v \leq 75 ,
\]
shown below and the sliders to find approximate answers to each of the following questions. But first translate each English question into mathematics and each mathematical question into English.

(a) What is the car's gas mileage at a speed of 30 miles/hour?

(b) At what speeds does the car get 35 miles/gal?

(c)  At what speeds does the car get at least 35 miles/gal?

(d)  At what speeds does the car get at most 35 miles/gal?

(e) Solve the inequality $f(v)> 50$.

(f) Solve the inequality $f(v)<50$.

(g) At what speeds is the gas mileage $25\%$ less than it is at a speed of 50 miles/hour?



\begin{exploration}\label{exp:Fun3}


\pdfOnly{
Access Desmos interactives through the online version of this text at
 
\href{https://www.desmos.com/calculator/pkn5m5wfky}.
}
 
\begin{onlineOnly}
    \begin{center}
\desmos{pkn5m5wfky}{900}{600}
\end{center}
\end{onlineOnly}
\end{exploration}



\end{example}



\begin{example}  \label{Ex:er044}
The function
\[
    h = f(t) , 0\leq t \leq 7 ,
\]
expresses the height of a balloon (in thousands of feet) in terms of hte number of hours past noon.

(a) \emph{Without} using the graph below, translate each of the following mathematical problems into a question in everyday English in  the context described above.

(b) Then use the graph below to answer each of the everyday English questions from part (a) with a complete sentence. Also answer the original questions mathematically, with either a solution set (parts ii)-ix)) or a function out-put (part i)). Give approximate answers where appropriate. Do \emph{not} use algebra. Briefly explain your reasoning for each.



\pdfOnly{
Access Desmos interactives through the online version of this text at
 
\href{https://www.desmos.com/calculator/ddrjntgogr}.
}
 
\begin{onlineOnly}
    \begin{center}
\desmos{ddrjntgogr}{900}{600}
\end{center}
\end{onlineOnly}

Access this desmos worksheet at \href{https://www.desmos.com/calculator/ddrjntgogr}{141: Balloon}


(i) Evaluate $f(5)$.

(ii) Solve the equations $f(t)=5$ and $f(t)=9$.

(iii) Solve the equation $f(t)=f(1)$.

(iv) Solve the inequalies $f(t)\geq 5$ and $f(t)<10$.

(v) Solve the inequality $f(t)< f(1)$.

(vi) Solve the inequality $f(t)\geq \frac{1}{4}f(3)$.

(vii) Solve the equation $f(t)=f(5) - 4$.

(viii) Solve the equation $f(t)=f(t-4)$.

(ix) Solve the inequality $f(t)<f(t-4)$.

(x) Make up your own question and share it with your neighbor.

\end{example}





\begin{example} \label{Ex:Fun4}

Use the graph of the function 
\[
    g = f(v) , 20\leq v \leq 75 ,
\]
shown below and the sliders to find approximate answers to each of the following questions. But first translate each question into English.
\begin{exploration}\label{exp:Fun4}

\pskip

(a) Solve the equation $f(v) = f(v+23)$.

(b) Solve the equation $f(v) = f(v-23)$.

(c) Solve the equation $f(v) = f(v+10) + 5$.

(d) Solve the inequality $f(v) < f(v+23)$.


\pdfOnly{
Access Desmos interactives through the online version of this text at
 
\href{https://www.desmos.com/calculator/0l3snjdur3}.
}
 
\begin{onlineOnly}
    \begin{center}
\desmos{0l3snjdur3}{900}{600}
\end{center}
\end{onlineOnly}
\end{exploration}

\end{example}





\begin{example} \label{Ex:Fun5}
The function
\[
   T = f(h) , 1\leq h \leq 8,
\]
expresses the temperature (in Fahrenheit degrees) as a function of altitude (in thousands of feet). 

\pskip

Make up at least five questions of the kind in Example 3 for this scenario. Use the graph of the function to find approximate answers/solutions to each question. Be sure to translate your question from math to English or from English to math as appropriate.

\begin{exploration}\label{exp:Fun5}


\pdfOnly{
Access Desmos interactives through the online version of this text at
 
\href{https://www.desmos.com/calculator/f4aqwpw3zo}.
}
 
\begin{onlineOnly}
    \begin{center}
\desmos{f4aqwpw3zo}{900}{600}
\end{center}
\end{onlineOnly}
\end{exploration}

\end{example}




\begin{example} \label{Ex:Fun6}
The function
\[
   T = f(h) , 1\leq h \leq 8,
\]
expresses the temperature (in Fahrenheit degrees) as a function of altitude (in thousands of feet). 

\pskip

Make up at least three questions of the kind in Example 5 for this scenario. Use the graph of the function to find approximate answers/solutions to each question. Be sure to translate your question from math to English first

\begin{exploration}\label{exp:Fun6}


\pdfOnly{
Access Desmos interactives through the online version of this text at
 
\href{https://www.desmos.com/calculator/jx74i9mne0}.
}
 
\begin{onlineOnly}
    \begin{center}
\desmos{jx74i9mne0}{900}{600}
\end{center}
\end{onlineOnly}
\end{exploration}

\end{example}



%%%%%%%%%%%%%%%%%%%%%%%%%%%%%%%%%%%%%%%%%


\section{Graphing Functions}

\begin{exploration}\label{exp:Bugs1}
Play the slider $u$ to activate the motion of the point $P$ below.

(a) Sketch by hand a graph of the function 
\[
   s=f(t), 0\leq t \leq 6 ,
\] 
that expresses the distance (in cm) between $P$ and $A$ as a function of the number of seconds since the motion started.

Then activate the folder on Line 9 by clicking the circle at the left of the line. Play the slider $u$ again and check if your graph is correct.

(b) On the same coordinate system, sketch a graph of the function 
\[
   s=g(t), 0\leq t \leq 6 ,
\] 
that expresses the distance (in cm) between $P$ and $B$ as a function of the number of seconds since the motion started.

Then activate the folder on Line 17 by clicking the circle at the left of the line. Play the slider $u$ again and check if your graph is correct.


\pdfOnly{
Access Desmos interactives through the online version of this text at
 
\href{https://www.desmos.com/calculator/q7w0qmwzdb}.
}
 
\begin{onlineOnly}
    \begin{center}
\desmos{q7w0qmwzdb}{900}{600}
\end{center}
\end{onlineOnly}
\end{exploration}

\begin{exploration}\label{exp:Bugs2}
This is a continuation of the last exploration where the motion of point $P$ is the same as before, but we vary the position of point $A$. The function 
\[
   s=f(t), 0\leq t \leq 6 ,
\] 
still expresses the distance (in cm) between $P$ and $A$ as a function of the number of seconds since the motion started.

Our purpose here is to see how this graph changes as we change the position of $A$. Drag the slider $w$ below to change the position of $A$.

\pdfOnly{
Access Desmos interactives through the online version of this text at
 
\href{https://www.desmos.com/calculator/pfk6p5ptd3}.
}
 
\begin{onlineOnly}
    \begin{center}
\desmos{pfk6p5ptd3}{900}{600}
\end{center}
\end{onlineOnly}
\end{exploration}



\end{document}
